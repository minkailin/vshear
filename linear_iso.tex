\section{Analytical discussion of vertically isothermal
  disks in the framework of the vertically global shearing
  box}\label{analytical}  
We begin with a discussion of vertically isothermal disks
where $\Gamma=1$, then $c_s$ is a constant in the radially-local
approximation. We  first examine isothermal perturbations ($\beta =
0$) before considering the effect of finite thermal relaxation. 
In order to facilitate a fully analytic discussion, we introduce a
number of approximations below that allows a simple governing equation
to be derived and compared with previous works.  

\subsection{Relation to the vertically global
  shearing box} 
We connect the above linearized equations to the vertically global
shearing box (VGSB) formalism developed by \cite{mcnally14}. The VGSB
is an extension of the standard shearing box \citep{goldreich65} to
allow the shear flow in the shearing box to be height-dependent.  
The difference between the radially-local approximation and the VGSB is
that the radial disk structure is ignored in the latter.  That is, the
equilibrium VGSB is homogeneous in the radial direction, as in the
standard shearing box. 


Thus, the linear problem in the VGSB can be derived from
Eq. \ref{lin_mass}---\ref{lin_energy} by setting 
$\hat{g}_c=0$.  
The resulting linear problem is very similar
to that for local axisymmetric waves in accretion disks already
discussed by \cite{lubow93}, but modified by the presence of vertical
shear and thermal relaxation. This is convenient as it allows us to
directly apply their solution methods.    

% but modified by the vertical shear term
% $r\p_z\Omega\delta \delta v_z$. 

After setting $\hat{g}_c=0$ in Eq. \ref{lin_mass}, \ref{lin_vx} and
\ref{lin_energy}, we can eliminate the perturbed velocities between
the linearized equations to obtain a pair or ordinary differential
equations for $W,Q$ as
\begin{align}
  &\frac{\sigma^2}{c_s^2}Q + \frac{\sigma^2k_x^2}{D}W \notag\\ 
  &=\left(\frac{\ii
      k_x r}{D}\frac{d\Omega^2}{dz} -
    \frac{d\ln{\rho}}{dz}\right)\left[\frac{dW}{dz} +
    \frac{d\ln{\rho}}{dz}\left(W-Q\right)\right] \notag\\
  &- \frac{d^2W}{dz^2} - \frac{d\ln{\rho}}{dz}\left(\frac{dW}{dz} -
    \frac{dQ}{dz}\right) - \frac{d^2\ln{\rho}}{dz^2}\left(W-Q\right),\label{ode_w}\\
  &\sigma^2W - \frac{\gamma}{\Gamma}\sigma^2 Q +
  \frac{\ii\sigma}{t_c}\left(W-\frac{Q}{\Gamma}\right)\notag\\
  &=c_s^2\frac{d\ln{\rho}}{dz}\left(\frac{\gamma}{\Gamma} - 1\right) 
  \left[\frac{dW}{dz} + \frac{d\ln{\rho}}{dz}\left(W-Q\right)\right],\label{ode_Q} 
\end{align}
where
\begin{align}
  D \equiv \kappa^2 - \sigma^2,
\end{align}
and we have replaced the notation $\p_z$ by $d/dz$ since we are
considering solutions at a fixed radius. 

The VGSB is a self-consistent framework but extrapolating results 
obtained from the VGSB to the global disk has limitations. This is
because within the VGSB framework, $d\Omega/dz$ can be freely
specified. In effect, by adopting the VGSB framework, we are studying
dynamics in the standard shearing box subject to an additional
azimuthal force $\delta v_zr d \Omega/dz$ with $d\Omega/dz$ as an
input parameter. This is different to the global problem because in
that case, the vertical shear is a result of the background radial
disk structure, which is ignored in the VGSB.    

We further discuss caveats of the VGSB in Appendix \ref{global_corr}.
However, we find that for perturbations with small radial lengthscales
($k_xH\gg 1$) and fast thermal relaxation ($\beta\ll1$), which
characterizes the VSI, the VGSB framework is adequate for the purpose
of a simple analytical discussion. 




\subsection{Governing equation in the VGSB}\label{approx_gov}
We now derive a governing equation with further approximations:
\begin{enumerate}
\item We consider eigenvalues such that
  $|\sigma^2|\ll \kappa^2$. Then we can replace $D=\kappa^2 -\sigma^2\to
  \kappa^2$. % The problem simplifies because the eigenvalue now only
  % appears in the last term in Eq. \ref{iso_ode}.
  We call this the
  low-frequency approximation. 
\item We are mainly interested in the effect of vertical shear, i.e. $q\neq
  0$. This is already captured explicitly in the linearized equations through
  the term $d\Omega/dz$. We therefore simplify
  further by ignoring the vertical dependence of $\Omega$ and $\kappa$
  elsewhere. For clarity, we replace $\kappa^2$ with $\Omega_k^2$. 
  
\end{enumerate}
Making both of these approximations is equivalent to setting 
$D\to\Omega_k^2$.  After making this replacement, we eliminate $W$ and
$Q$ between  Eq. \ref{lin_vz} and Eq. \ref{ode_w}---\ref{ode_Q},
making use of Eq. \ref{vertical_shear}, to
obtain an equation for $\delta v_z$ when $\Gamma=1$,
\begin{align}
  0 =& \frac{d^2\delta v_z}{dz^2} + \left(1 + \frac{\ii k_x c_s^2
      q}{\Omega_k^2r}\right)\frac{d\ln\rho}{dz}\frac{d\delta
    v_z}{dz} \notag\\
  % &+ \left[\sigma^2\left(\frac{k_x^2}{\Omega_k^2} +
  %     \frac{\chi}{c_s^2}\right) + \left(\chi + \frac{\ii k_x c_s^2
  &+ \left[\frac{\sigma^2}{c_s^2}\left(\frac{k_x^2c_s^2}{\Omega_k^2} +
      \chi\right) + \left(\chi + \frac{\ii k_x c_s^2
        q}{\Omega_k^2r}\right)\frac{d^2\ln\rho}{dz^2}\right. \notag\\
  &\phantom{+=}\left.-
    \frac{c_s^2}{\Omega_k^2}\left(\frac{d\ln\rho}{dz}\right)^2\left(k_x^2 -
      \frac{\ii k_x q}{r}\right)
    \left(1-\chi\right) 
  \right]\delta v_z,\label{vertiso_gov}
\end{align}
where
\begin{align}
  \chi \equiv \frac{1-\ii\sigma t_c}{1-\ii\sigma t_c\gamma}.
\end{align}

It is convenient to introduce the dimensionless variables
\begin{align}
  \hat{z} = z/\Hiso,\quad \hat{k}=k_x\Hiso, \quad \hat{\sigma} = \sigma/\Omega_k,
\end{align}
then Eq. \ref{vertiso_gov} becomes
\begin{align}
  0=& \delta v_z ^{\prime\prime} + \left(1 + \ii\epsilon
    q\hat{k}\right)\ln\rho^{\prime}\delta v_z^\prime \notag\\
  &+
  \left[\hat{\sigma}^2\left(\hat{k}^2+\chi\right) +
    \left(  \chi + \ii \epsilon q\hat{k}\right)\ln\rho^{\prime\prime}\right.
  \notag\\
  &\left.- \ln\rho^{\prime
      2}\left(\khat^2 -
      \ii\epsilon
      q\hat{k}\right)\left(1 - \chi\right)\right]\delta v_z,\label{vertiso_gov_nondim}
\end{align}
where $^\prime$ denotes $d/d\zhat$. Eq. \ref{vertiso_gov_nondim} will
be the basis for our discussion of vertically isothermal disks. % ,
% although further simplifications or maniputations will be performed
% depending the type of perturbations being considered. 

In deriving Eq. \ref{vertiso_gov}, the vertical dependence of
$D=\kappa^2(z)-\sigma^2$ was ignored because the replacement
$D\to\Omega_k^2$ was made before eliminating variables in favor of
$\delta v_z$. In Appendix \ref{adia_improve} we show that making this
replacement after eliminating variables, which requires the vertical
derivative of $D$, adds unnecessary complexity for thin disks.  

% We will consider sound-speed profiles with $q<0$ in the global 
% disk. Then $\Omega^2(z)$ decreases away from the midplane,
% and may reach small values for sufficiently large $|z|$, which
% invalidates the low-frequency approximation. In principle, then, the
% above approximations implicitly limits us to consider thin disks ($\epsilon
% \ll 1$) with vertical domain sizes not too large, so that the rotation
% is nearly-Keplerian.  

Note that we can generalize Eq. \ref{vertiso_gov_nondim} to other
(non-Keplerian) rotation profiles by redefining $\khat$
as $k_xc_s/\kappa$, $\hat{\sigma}$ as $\sigma/\kappa$, and 
$\epsilon$ as $c_s/r\kappa$.    

% Consider rotation profiles with radial dependence
% $\propto r^{-m}$. Then the same dimensionless governing equation 
% (Eq. \ref{vertiso_gov_nondim}) holds, provided we redfine $\khat$
% as $k_xc_s/\kappa$, $\hat{\sigma}$ as $\sigma/\kappa$, and reset
% $\epsilon$ to $\epsilon/sqrt{4-2m}$.   


\subsection{Isothermal perturbations}\label{iso_discuss}
Isothermal perturbations correspond to $\beta\equiv0$, 
%relaxation timescale $\be
%or for any adiabatc index $\gamma$ when
%$\beta=0$. 
so $\chi=1$ and Eq. \ref{ode_Q} implies 
\begin{align}
  Q = W, 
\end{align}
 and so 
 \begin{align}
   \ii\hat{\sigma}c_s\delta v_z = W^\prime
   % \quad   \text{(isothermal perturbations)},
 \end{align}
from Eq. \ref{lin_vz}. For isothermal perturbations it is in fact 
simpler to work with $W$ by integrating Eq. \ref{vertiso_gov_nondim}. For later discussion
it is convenient to write the result as 

\begin{align}
  0 = W^{\prime\prime} + \left[\ln\rho^\prime - \ii \epsilon q \hat{k}
    f(\zhat)\right]W^\prime + \hat{\sigma}^2\left(1+\hat{k}^2\right)W,\label{iso_ode1}
\end{align}
where $f(\zhat)$ is defined such that
\begin{align}\label{fz_shear}
  \frac{d\Omega^2}{d\hat{z}} = \epsilon^2q f(\hat{z})\Omega_k^2.
\end{align}

% The linear problem simplifies considerably for the case
% $\Gamma=\gamma=1$.  Then 
% \begin{align}
%   Q=W. 
% \end{align}
% Inserting this into Eq. \ref{ode_w} gives a single equation for the linear stability
% problem for isothermal perturbations in a vertically isothermal disk, 
% \begin{align}\label{iso_ode}
%   \frac{d^2W}{dz^2} + \left(\frac{d\ln{\rho}}{dz} - \frac{\ii k_x
%       r}{D}\frac{d\Omega^2}{dz}\right) \frac{dW}{dz} +
%   \sigma^2\left(\frac{1}{c_s^2} + \frac{k_x^2}{D}\right)W=0, 
% \end{align} 
% subject to appropriate boundary conditions. Eq. \ref{iso_ode} also
% applies to $\gamma>1$ for $t_c\to0$. 

Note that for isothermal perturbations the issue of ignoring the
vertical dependence of $D$ described in \S\ref{approx_gov} and Appendix
\ref{adia_improve} is irrelevant, since
Eq. \ref{iso_ode1} may be derived directly by substituting $Q=W$ in
Eq. \ref{ode_w}, i.e. no derivatives of $D$ needs to be taken. 

%We remark that Eq. \ref{iso_ode1} is also applicable for  $\gamma=\Gamma\neq 1$ in the
%limit $t_c\to\infty$ (i.e. neutrally stratified adiabatic disks),
%since in that case $Q\simeq W$ as well. However, the following
%discussion is only  valid for $\gamma=\Gamma=1$ because $c_s$ is taken
%to be a constant.    


\subsubsection{Necessity of vertical shear for
  instability}\label{integral_relation} 
We can establish a necessary condition for instability by multiplying
Eq. \ref{iso_ode1} by $\rho W^*$ and integrate vertically from
$\zhat=\zhat_1$ to $z=\zhat_2$. We neglect boundary 
terms when integrating by parts, by assuming $W$ or
$W^\prime$ vanishes at the boundaries, or that the boundaries are 
sufficiently far away so that the boundary terms are negligible because of the
decaying background density with increasing height. Then,
\begin{align}
  &\hat{\sigma}^2\left(1+\hat{k}^2\right)\int_{\zhat_1}^{\zhat_2}\rho|W|^2d\zhat \notag\\
  &=\int_{\zhat_1}^{\zhat_2}\rho|W^\prime|^2d\zhat 
  +\ii \epsilon q \hat{k}\int_{\zhat_1}^{\zhat_2}\rho f(\zhat) W^*W^\prime d\zhat.\label{integral_relation1}
\end{align}
It follows that for instability ($\imag\hat{\sigma}>0$), it is necessary to
have $q\neq0$ or more generally $d\Omega^2/dz\neq 0$, i.e. vertical
shear. A similar conclusion can be reached in the equivalent linear  
problem in global cylindrical disk geometry. 

\subsubsection{Maximum growth rate}   
Here we show that the growth rate is limited by the maximum vertical
shear in the domain. Writing $\hat{\sigma} = \hat{\omega} +
\ii\hat{\nu}$ with real $\hat{\omega}$ and $\hat{\nu}$, the real and
imaginary parts of 
Eq. \ref{integral_relation1} are
\begin{align}
  &\left(\hat{\omega}^2-\hat{\nu}^2\right)\left(1+\hat{k}^2\right)
  \int_{\zhat_1}^{\zhat_2}\rho|W|^2d\hat{z} -
  \int_{\zhat_1}^{\zhat_2}\rho|W^\prime|^2d\hat{z}
  \notag\\
  &=\real\left[\ii
    \epsilon\hat{k}q \int_{\zhat_1}^{\zhat_2}\rho
    f(\hat{z}) W^*W^\prime d\zhat\right], \notag\\
  & 2\hat{\omega}\hat{\nu}\left(1+\hat{k}^2\right)
  \int_{\zhat_1}^{\zhat_2}\rho|W|^2d\hat{z}\notag\\
  &=\imag\left[\ii
    \epsilon\hat{k}q \int_{\zhat_1}^{\zhat_2}\rho
    f(\hat{z}) W^*W^\prime d\hat{z}\right].
\end{align}
Adding the square of these equations give
\begin{align}
  &\left[|\hat{\sigma}|^2\left(1+\hat{k}^2\right)
    \int_{\zhat_1}^{\zhat_2}\rho|W|^2d\hat{z} -
    \int_{\zhat_1}^{\zhat_2}\rho\left|W^\prime \right|^2d\hat{z}\right]^2\notag\\
  &+4\hat{\nu}^2\left(1+\hat{k}^2\right) 
  \int_{\zhat_1}^{\zhat_2}\rho
  |W|^2d\hat{z}\int_{\zhat_1}^{\zhat_2}\rho\left|W^\prime \right|^2d\hat{z}\notag\\
  &=\left|\ii
    \epsilon\hat{k}q\int_{\zhat_1}^{\zhat_2}\rho
    f(\hat{z}) W^*W^\prime d\hat{z}\right|^2.
\end{align}
It is clear that
\begin{align}\label{sigma_finite_domain} 
  &4\hat{\nu}^2\left(1+\hat{k}^2\right) 
  \int_{\zhat_1}^{\zhat_2}\rho
  |W|^2d\hat{z}\int_{\zhat_1}^{\zhat_2}\rho\left|W^\prime \right|^2d\hat{z}\notag\\
  &\leq\left|
    \epsilon\hat{k}q\int_{\zhat_1}^{\zhat_2}\rho
    f(\hat{z}) W^*W^\prime d\hat{z}\right|^2.
\end{align}
On the left hand side of this inequality, we apply the Cauchy-Schwarz
inequality to obtain
\begin{align}
  &\left( \int_{\zhat_1}^{\zhat_2}\rho
    |W|\left|W^\prime \right|d\hat{z}\right)^2\leq
  \int_{\zhat_1}^{\zhat_2}\rho 
  |W|^2d\hat{z}\int_{\zhat_1}^{\zhat_2}\rho\left|W^\prime \right|^2d\hat{z}.
\end{align}
On the right hand side of Eq. \ref{sigma_finite_domain} we have
\begin{align}
  \left|\int_{\zhat_1}^{\zhat_2}\rho
    f(\hat{z}) W^*W^\prime d\hat{z}\right|\leq \int_{\zhat_1}^{\zhat_2}\rho
  \left|f(\hat{z})W^*W^\prime \right|d\hat{z} \notag\\
  \leq
  \mathrm{max}\left(|f|\right)\int_{\zhat_1}^{\zhat_2}\rho
  |W|\left|W^\prime \right|d\hat{z},
\end{align}
where $\mathrm{max}(|f|)$ is the maximum value of $|f|$ in
$\zhat\in[\zhat_1,\zhat_2]$. Inserting these inequalities into
Eq. \ref{sigma_finite_domain} gives
\begin{align}\label{max_growth}
  |\hat{\nu}|\leq
  \frac{\epsilon |\hat{k} q|}{2\sqrt{1+\hat{k}^2}}\mathrm{max}(|f|). 
\end{align}
Recalling that $f$ represents vertical shear (Eq. \ref{fz_shear}), it
follows that the maximum possible growth rate of unstable modes,
satisfying the above boundary conditions, is limited by the maximum
value of the vertical shear in the domain considered. 

For the vertically isothermal disk we have $|f(\hat{z})| =
|\hat{z}|\left(1+\epsilon^2\hat{z}^2\right)^{-3/2}$, which maximizes
at $|\zhat|=1/\sqrt{2}\epsilon$. For vertical domains smaller 
than this height, the growth rate is limited by $|f|$ at the vertical
boundaries considered. For $|\epsilon\hat{z}|\ll1$ we have $f\simeq
\hat{z}$, in which case we can expect growth rates to increase at most linearly
with respect to the vertical domain size.   %no faster than linear 



\subsubsection{Thin-disk limit}
We can make further progress for thin disks ($\epsilon\ll1$), 
by expanding the background density and vertical shear profile in powers
of $z/r$. To lowest order we obtain 
\begin{align}
  &\frac{\rho}{\rho_0} \simeq 
  \exp{\left(-\frac{\hat{z}^2}{2}\right)} \equiv w(\hat{z}),\label{thin_dens}\\
  &\frac{d\Omega^2}{d\hat{z}} \simeq \epsilon^2q\Omega_k^2\hat{z}, \label{thin_vshear}
\end{align} 
%With this approximation and those in \S\ref{integral_relation}, 
and the governing equation becomes 
\begin{align}\label{iso_ode3}
  W^{\prime\prime} - \left(1 + \ii q\epsilon
    \hat{k}\right)\hat{z}W^\prime  +
  \hat{\sigma}^2\left(1+\hat{k}^2\right)W = 
  0.
\end{align}
We remark that for large $|\hat{k}|$, Eq. \ref{iso_ode3} is the same as
that derived by \cite{nelson13}, although we have taken a different
route.  

To complete the problem we must specify boundary conditions. The 
maximum vertical domain size should be limited by the thin
disk and low-frequency approximations (i.e. $z\lesssim r$).     
It is, however, common to take $\mathrm{max}|z|\to\infty$ and impose
that the kinetic  energy remain bounded at infinity. This 
is appropriate for the density field because both the true density profile 
and its thin-disk approximation decay rapidly away from the midplane.   

However, the thin-disk representation of vertical shear diverges with 
$\hat{z}$, unlike the true vertical shear profile which eventually
decays.  Thus, taking $\zhat_{1,2}\to \pm\infty$ will permit unphysically large growth
rates (as demonstrated below). Nevertheless, the infinite disk is
useful to consider since it permits an analytical discussion. 

\subsubsection{Stability in the absence of vertical shear}\label{iso_stable}
When $q=0$, Eq. \ref{iso_ode3} is
\begin{align}\label{hermite_ode}
  \left[w(\hat{z})W^\prime \right]^\prime + nW
  w(\hat{z}) =0, 
\end{align}
where we have defined a new eigenvalue
\begin{align}
  n \equiv \hat{\sigma}^2(1+\hat{k}^2). 
\end{align} 
Eq. \ref{hermite_ode} is Hermite's differential equation. If we impose
that the kinetic energy density remain bounded at infinity, then  
\begin{align}
  W \propto \He_n(\hat{z}),
\end{align}
and $n$ is a non-negative integer. This translates to a real
eigenfrequency $\sigma$ such that
\begin{align}
  \left|\hat{\sigma}\right| = \sqrt{n}
  \left(1+\hat{k}^2\right)^{-1/2}. \label{iso_stable_disp}
\end{align}
Since we have assumed $|\sigma^2|\ll \kappa^2\sim \Omega_k^2$, our
analysis is only valid for large wavenumbers such that $\hat{k}^2\gg   
n-1$. Physically, this corresponds to radial length-scales much
smaller than the local disk scale height. 

We remark that for fixed $n$ and $\khat\gg 1$ the dispersion relation 
Eq. \ref{iso_stable_disp} imply $\hat{\omega}\propto 1/\khat$. This inverse relation 
has also been qualitatively observed in numerical simulations performed by 
\cite{stoll14}. 


\subsubsection{Polynomial solutions with vertical shear}\label{iso_poly}
We can seek power-series solution to Eq. \ref{iso_ode3},
\begin{align}
  W(\zhat) = \sum_{l=0}^\infty a_l\zhat^l. 
\end{align}
Then the coefficients must satisfy the recurrence relation
\begin{align}
  (l+2)(l+1)a_{l+2} +
  \left[n - l\left(1+\ii \epsilon q  \hat{k}\right)\right] a_l = 0, 
\end{align}
where $n$ is not necessarily an integer. Indeed, if we demand
a polynomial of degree $L$ as the solution, then the eigenfrequency
must satisfy
\begin{align}\label{sig2_iso}
\hat{\sigma}^2 = L\left(\frac{1+\ii q \epsilon
    \hat{k}}{1+\hat{k}^2}\right).
\end{align}
The eigenfrequency $\hat{\sigma}$ is therefore complex for
$L\geq1$. We are interested in 
unstable modes for which $\imag{\hat{\sigma}}>0$ is the growth rate and is
given by 
\begin{align}\label{simple_growth}
  \hat{\nu} =\sqrt{
   \frac{L}{2\left(1+\hat{k}^2\right)}\left(\sqrt{1+q^2\epsilon^2\hat{k}^2} - 
    1\right)}. 
\end{align}
For fixed $q$ and $\epsilon$, the growth rate vanishes for both
$\hat{k}^2\to0$ and $\hat{k}^2\to\infty$. The maximum growth rate
occurs at the optimum wavenumber $\hat{k}_\mathrm{opt}$
\begin{align}
  |\hat{k}_\mathrm{opt}| = \sqrt{\frac{2+|\epsilon q|}{|\epsilon q|}},
\end{align}
and
\begin{align}
  \mathrm{max}\left(\hat{\nu}\right) =\frac{\sqrt{L}|\epsilon
    q|}{2\sqrt{1+|\epsilon q|}}. \label{iso_max_growth}
\end{align}
Since $L\geq1$, there is no limit to the maximum growth rate. This is
an artifact of the thin-disk approximation because the vertical 
shear $d\Omega^2/dz\propto z$ increases indefinitely with
height. However, we find Eq. \ref{iso_max_growth} compares well with
numerical solutions of the governing equation for small $L$.      

% This expression can be simplified in the limit $|q|\ll 1$, 
%  \begin{align}\label{simple_growth}
%    \frac{\imag(\sigma)}{\Omega_k} = \pm \sqrt{M} 
%    \frac{q\epsilon\hat{k}}{2\sqrt{1+\hat{k}^2}} \quad\quad (|q|\ll 1), 
%  \end{align}
%  with $M$ being an integer. 

% In Appendix \ref{pert_theory} we give an alternative method to
% infer instability in the presence of vertical shear without explicitly
% solving the governing equation. 

