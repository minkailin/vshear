\section{Vertically isothermal disks with 
  isothermal perturbations}
The linear problem simplifies considerably for the case
$\Gamma=\gamma=1$. Then 
\begin{align}
  W=Q. 
\end{align}
%so that
%\begin{align}
 % \delta v_z = \frac{1}{\ii\sigma}\frac{dQ}{dz}. 
%\end{align}
%Inserting this into the expression for $\delta v_x$ and the continuity
%equation, 
Inserting this into Eq. \ref{ode_w} gives a single equation for the linear stability
problem for isothermal perturbations in a vertically isothermal disk, 
\begin{align}\label{iso_ode}
  \frac{d^2Q}{dz^2} + \left(\frac{d\ln{\rho}}{dz} - \frac{\ii k_x
      r}{D}\frac{d\Omega^2}{dz}\right) \frac{dQ}{dz} +
  \sigma^2\left(\frac{1}{c_s^2} + \frac{k_x^2}{D}\right)Q=0, 
\end{align} 
subject to appropriate boundary conditions. 
We remark that Eq .\ref{iso_ode} is also applicable for
$\gamma=\Gamma\neq 1$ in the limit $t_c\to\infty$, since in that case
$W\simeq Q$ as well. 
%The following discussion therefore also holds for
%vertically non-isothermal, adiabatic disks.   

\subsection{Neccessity of vertical shear for instability}\label{integral_relation}
We can establish a neccessary condition for instability with the
following assumptions:

\begin{enumerate}
\item We consider eigenvalues such that
  $|\sigma^2|\ll \kappa^2$. Then we can replace $D=\kappa^2 -\sigma^2\to
  \kappa^2$. The problem simplifies because the eigenvalue now only
  appears in the last term in Eq. \ref{iso_ode}. 
\item We are mainly interested in the effect of vertical shear, i.e. $q\neq
0$. This is already captured explicitly in Eq. \ref{iso_ode} through
the factor $d\Omega^2/dz$. We 
therefore simplify further by ignoring the vertical dependence of
$\kappa^2$. For clarity, we will replace  $\kappa^2$ with $\Omega_k^2$.  
\end{enumerate}

With these approximations we can re-write the governing equation as
\begin{align}\label{iso_ode1}
&\frac{d}{dz}\left(\rho e^{-\ii k_x r
    \Omega^2/\Omega_k^2}\frac{dQ}{dz}\right) +
\sigma^2\left(\frac{1}{c_s^2}+\frac{k_x^2}{\Omega_k^2}\right) \rho Q e^{-\ii k_x r
  \Omega^2/\Omega_k^2} \notag\\
&=0.
\end{align}
% Writing
% \begin{align}
%   V \equiv Q\exp{\left[-\frac{\hat{z}^2}{2}(\ii q \epsilon
%       \hat{k})\right]},
% \end{align}
% we have an alternative form of the governing equation 
% \begin{align}\label{iso_alt}
%   \frac{d}{d\hat{z}}\left[w(\hat{z})\left(\frac{dV}{d\hat{z}} + \ii q \epsilon
%     \hat{k}\hat{z}V\right)\right] + nV w(\hat{z}) =0, 
% \end{align}
% where 
% \begin{align}
%   w(\hat{z}) \equiv \exp{\left(-\frac{\hat{z}^2}{2}\right)},  
% \end{align}
% and we have defined a new eigenvalue
% \begin{align}
%   n \equiv \left(\frac{\sigma}{\Omega_k}\right)^2(1+\hat{k}^2). 
% \end{align}

Next, we multiply Eq. \ref{iso_ode1} by $Q^{*}\exp{\left(\ii k_x r
    \Omega^2/\Omega_k\right)}$, integrate 
vertically and assume boundary terms vanish to obtain
\begin{align}\label{neccessary_cond}
   &\sigma^2\left(\frac{1}{c_s^2}+\frac{k_x^2}{\Omega_k^2}\right)\int_{-\infty}^{\infty}\rho|Q|^2dz \notag\\
  &=\int_{-\infty}^{\infty}\rho\left|\frac{dQ}{dz}\right|^2dz 
  +\frac{\ii k_x r}{\Omega_k^2}\int_{-\infty}^{\infty}\rho\frac{d\Omega^2}{dz}Q^*\frac{dQ}{dz}dz. 
\end{align}
%\begin{align}\label{integral_relation}
 % n \int_{-\infty}^{\infty}w|V|^2d\hat{z} 
 % =&\int_{-\infty}^{\infty}w\left|\frac{dV}{d\hat{z}}\right|^2d\hat{z}\notag\\
 % +&\ii q \epsilon \hat{k}\int_{-\infty}^{\infty}w\hat{z}V\frac{dV^*}{d\hat{z}}d\hat{z}.   
%\end{align}

It follows that for instability ($\imag\sigma>0$), it is neccessary to
have $d\Omega^2/dz\neq 0$, i.e. vertical 
shear. 

% We recognize that in Eq. \ref{integral_relation}, $w(\hat{z})$
% represents the background density profile and
% the factor $\epsilon q \hat{z}$ represents vertical shear. 
% Making the replacement $w\to \rho$ and 
% $\epsilon q \hat{z}\to (r/\Omega_k^2)d\Omega^2/dz$ in
% Eq. \ref{integral_relation}  gives the corresponding integral relation
% without the thin-disk approximation for the the equilibrium density
% and vertical shear profile (Eq. \ref{thin_vshear} ---
% \ref{thin_dens}).  

%Assuming $V$ is either odd or even in
%$\hat{z}$,  then for the second integral on the right
%hand side to be non-zero, it is neccessary to have a vertical shear
%profile that has odd symmetry in the vertical co-ordinate.  

\subsection{Thin-disk limit}
The full expressions for the equilibrium density and vertical shear
depends on $z$ in a non-trivial manner. Thus, to simplify the problem 
we consider thin-disks ($\epsilon\ll1$), 
% Furthermore, the eigenfrequency
% $\sigma^2$ appears in $D$. In order to make progress, in this section
% we make the following approximations. 
so we may expand the background density and vertical shear in powers
of $z/r$. To lowest order we obtain
\begin{align}
  &\rho(z) \simeq \rho_{00} \exp{\left(-\frac{z^2}{2\Hiso^2}\right)},\label{thin_dens}\\
  &\frac{d\Omega^2}{dz} \simeq \frac{q\Omega_k^2z}{r^2}. \label{thin_vshear}
\end{align}

% \emph{Nearly constant $\kappa$}. We are mainly interested in the
% effect of vertical shear, i.e. $q\neq 0$. This is already captured
% explicitly in Eq. \ref{iso_ode}. We therefore simplify further by
% ignoring the vertical dependence of $\kappa^2$. In a thin disk, this
% quantity is nearly Keplerian.

With this approximation and those in \S\ref{integral_relation}, the
governing equation becomes
\begin{align}\label{iso_ode3}
  \frac{d^2Q}{d\hat{z}^2} - \left(1 + \ii q\epsilon
    \hat{k}\right)\hat{z}\frac{dQ}{d\hat{z}} +
  \left(\frac{\sigma}{\Omega_k}\right)^2\left(1+\hat{k}^2\right)Q = 
  0, 
\end{align}
where we have introduced the dimensionless vertical co-ordinate
$\hat{z} \equiv z/\Hiso$ and wavenumber $\hat{k} \equiv k_x \Hiso$. 

In the thin-disk approximation the integral relation
Eq. \ref{neccessary_cond} becomes
\begin{align}\label{integral_relation_thin}
 & \left(\frac{\sigma}{\Omega_k}\right)^2\left(1+\hat{k}^2\right)
  \int_{-\infty}^{\infty}w|Q|^2d\hat{z}  \notag\\ 
  & =\int_{-\infty}^{\infty}w\left|\frac{dQ}{d\hat{z}}\right|^2d\hat{z}
  +\ii q \epsilon
  \hat{k}\int_{-\infty}^{\infty}w\hat{z}Q^*\frac{dQ}{d\hat{z}}d\hat{z}.    
\end{align}
where 
\begin{align}
  w(\hat{z}) \equiv \exp{\left(-\frac{\hat{z}^2}{2}\right)}.  
\end{align}

% the governing equation becomes
% \begin{align}\label{iso_ode2}
%   &\frac{d^2Q}{d\hat{z}^2} - \left[ 1 + \ii q\epsilon
%     \left(k_x\Hiso\right)\frac{\Omega_k^2}{\kappa^2}\right]\hat{z}\frac{dQ}{d\hat{z}}\notag\\
%   &+ \sigma^2\Hiso^2\left(\frac{1}{c_s^2}+\frac{k_x^2}{\kappa^2}\right)Q =
%   0. 
% \end{align}

\subsection{Stability in the absence of vertical shear}
When $q=0$, Eq. \ref{iso_ode3} is
\begin{align}\label{hermite_ode}
  \frac{d}{d\hat{z}}\left[w(\hat{z})\frac{dQ}{d\hat{z}}\right] + nQ
  w(\hat{z}) =0, 
\end{align}
where we have defined a new eigenvalue
\begin{align}
  n \equiv \left(\frac{\sigma}{\Omega_k}\right)^2(1+\hat{k}^2). 
\end{align} 
Eq. \ref{iso_ode3} is Hermite's differential equation. If we impose
that the kinetic energy density remain bounded at infinity, then  
\begin{align}
  Q \propto \He_n(\hat{z}),
\end{align}
and $n$ is a non-negative integer. This translates to a real
eigenfrequency $\sigma$ such that
\begin{align}
  \left|\frac{\sigma}{\Omega_k}\right| = \sqrt{n}
  \left(1+\hat{k}^2\right)^{-1/2}. 
\end{align}
Since we have assumed $|\sigma^2|\ll \kappa^2$, our analysis is only
valid for large wavenumbers such that $\hat{k}^2\gg 
n-1$. Physically, this corresponds to radial length-scales much
smaller than the local disk scale height. 

\subsection{Polynomial solutions with vertical shear}
We can seek power-series solution to Eq. \ref{iso_ode3},
\begin{align}
  Q(\zhat) = \sum_{m=0}^\infty a_m\zhat^m. 
\end{align}
Then the coefficients must satisfy the recurrence relation
\begin{align}
  (m+2)(m+1)a_{m+2} +
  \left[n - m\left(1+\ii q h \hat{k}\right)\right] a_m = 0, 
\end{align}
where $n$ is not neccessarily an integer. Indeed, if we demand
a polynomial of degree $M$ as the solution, then the eigenfrequency
must satisfy
\begin{align}
\left(\frac{\sigma}{\Omega_k}\right)^2 = M\left(\frac{1+\ii q \epsilon
    \hat{k}}{1+\hat{k}^2}\right).
\end{align}
The eigenfrequency $\sigma$ is therefore complex. We are interested in
unstable modes for which $\imag{\sigma}$ is the growth rate and is
given by 
\begin{align}
%   &\left[\frac{\real(\sigma)}{\Omega_k}\right]^2 =
%  \frac{M}{2\left(1+\hat{k}^2\right)}\left(\sqrt{1+q^2h^2\hat{k}^2} + 1\right)\\
   \left[\frac{\imag(\sigma)}{\Omega_k}\right]^2 =
   \frac{M}{2\left(1+\hat{k}^2\right)}\left(\sqrt{1+q^2\epsilon^2\hat{k}^2} - 
    1\right). 
\end{align}
This expression can be simplified in the limit $|q|\ll 1$, 
 \begin{align}\label{simple_growth}
   \frac{\imag(\sigma)}{\Omega_k} = \pm \sqrt{M} 
   \frac{q\epsilon\hat{k}}{2\sqrt{1+\hat{k}^2}} \quad\quad (|q|\ll 1), 
 \end{align}
 with $M$ being an integer. 

% The differential equation, Eq. \ref{iso_ode3}, can be solved by
% \begin{align}
%   Q\propto \hat{z}. 
% \end{align} 
% %Although $Q$ diverges at infinity, this solution is
% %acceptable because the implied kinetic energy densities remain
% %bounded, since the density decays rapidly. 
% In this case the eigenfrequency must satisfy 
% \begin{align}
%   \left(\frac{\sigma}{\Omega_k}\right)^2 = \frac{1+\ii q h
%     \hat{k}}{1+\hat{k}^2}, 
% \end{align}
% and $\sigma$ is generally complex\footnote{
% We note that an even simpler solution to Eq. \ref{iso_ode3} is
% $Q=\mathrm{constant}$, but this implies $\sigma^2=0$ and therefore
% stability, so this case is of no interest.}, with  
% %If we write $\sigma =
% %(\hat{\omega} + \ii \hat{\nu})\Omega_k$ with $\hat{\omega},\hat{\nu}$
% %real, we have
% \begin{align}
%   \left[\frac{\imag(\delta\sigma)}{\Omega_k}\right]^2 =
%   \frac{\sqrt{1+q^2h^2\hat{k}^2} - 1}{2\left(1+\hat{k}^2\right)}. 
% \end{align}
% This expression can be simplified in the limit $|q|\ll 1$. Then we have
% \begin{align}\label{simple_growth}
%   \frac{\imag(\sigma)}{\Omega_k} = \pm
%   \frac{qh\hat{k}}{2\sqrt{1+\hat{k}^2}} \quad\quad (|q|\ll 1).  
%\end{align}

\subsection{Instability in the presence of weak vertical shear}
We can also investigate the destabilizing effect of vertical shear
without explicitly solving the full governing equation,
Eq. \ref{iso_ode3}. We first
consider a system with $q\equiv0$, for which the eigenfunctions are
Hermite polynomials and the real eigenfrequencies are known. We then
perturb this system by introducing a weak vertical shear ($|q|\ll1$)
and linearize the governing equation. This procedure is
\begin{align}   
  &q \to 0 + \delta q\\
  &Q \to \He_n + \delta V \\
  &\sigma \to \sigma + \delta\sigma, 
\end{align}
where for this exercise we restore $\sigma$ as the eigenfrequency. 

Linearizing the integral relation in the thin-disk limit, Eq. \ref{integral_relation_thin}, and taking the
imaginary part, we have
\begin{align}
  \frac{2\sigma\imag(\delta\sigma)}{\Omega_k^2}
  \left(1+\hat{k}^2\right) \int_{-\infty}^{\infty} w(\zhat)
  \He_n^2(\hat{z}) d\hat{z} \notag\\
  = \delta q \epsilon \hat{k} 
  \int_{-\infty}^{\infty}
  w(\zhat)\hat{z}\He_n(\hat{z})\frac{d\He_n}{d\zhat} d\hat{z}
\end{align}
Recognizing $\hat{z} = \He_1$ and using $d\He_n/d\hat{z} = n
\He_{n-1}$ we have
 \begin{align}
   \frac{2\sigma\imag(\delta\sigma)}{\Omega_k^2}
   \left(1+\hat{k}^2\right) \int_{-\infty}^{\infty} w(\zhat)
   \He_n^2(\hat{z}) d\hat{z} \notag\\
   =\delta q \epsilon \hat{k} 
   \int_{-\infty}^{\infty}
   w(\zhat)\He_1(\hat{z})\He_n(\hat{z})n\He_{n-1}(\hat{z})d\hat{z}. 
 \end{align}
% Integrating the right hand side by parts and recognizing $\He_2 =
% \hat{z}^2 -1$, we have
% \begin{align}
%   \frac{2\sigma\imag(\delta\sigma)}{\Omega_k^2}
%   \left(1+\hat{k}^2\right) \int_{-\infty}^{\infty} w(\zhat)
%   \He_n^2(\hat{z}) d\hat{z} \notag\\
%   = \frac{1}{2}\delta q h \hat{k} 
%   \int_{-\infty}^{\infty}
%   w(\zhat)\He_2(\hat{z})\He_n^2(\hat{z})d\hat{z}. 
% \end{align}
Finally, we evaluate the integrals using following properties of Hermite polynomials
\begin{align}
  \int_{-\infty}^{\infty}
  w(\xi) \He_k(\xi)\He_l(\xi) d\xi = \sqrt{2\pi}k!\delta_{kl}, 
\end{align}
and
\begin{align}
  &\int_{-\infty}^{\infty}
  w(\xi) \He_m(\xi)\He_k(\xi)\He_l(\xi) d\xi\notag\\& =
  \frac{\sqrt{2\pi}m!k!l!}{(j-m)!(j-k)!(j-l)!}, 
\end{align}
where $j = (m+k+l)/2$. We then obtain
% \begin{align}
%  \frac{2\sigma\imag(\delta\sigma)}{\Omega_k^2}
%  \left(1+\hat{k}^2\right) = \delta q h \hat{k} \frac{n!}{(n-1)!}.
% \end{align
 \begin{align}
  \frac{2\sigma\imag(\delta\sigma)}{\Omega_k^2}
  \left(1+\hat{k}^2\right) = \delta q \epsilon \hat{k} n.
 \end{align}
Inserting the original eigenfrequency $\sigma = \pm
\sqrt{n}\Omega_k(1+\hat{k}^2)^{-1/2}$ gives
\begin{align}
  \frac{\imag(\delta\sigma)}{\Omega_k} = \pm \frac{1}{2}\delta q \epsilon
  \sqrt{n} \frac{\hat{k}}{\sqrt{1+\hat{k}^2}}. 
\end{align}
This result agrees with Eq. \ref{simple_growth}.  
%for $n=1$, since the
%function $\He_1$ solves the governing equation exactly. 
For $\hat{k}\gg 1$ we have
\begin{align}
  \frac{\imag(\delta\sigma)}{\Omega_k} \simeq \pm \frac{1}{2}\delta q \epsilon
  \sqrt{n} \sgn{\hat{k}}. 
\end{align}
