\section{Analytical discussion of vertically isothermal
  disks in the framework of the vertically global shearing
  box}\label{analytical}  
We begin with a discussion of vertically isothermal disks
where $\Gamma=1$, then $c_s$ is a constant in the radially-local
approximation. % We examine isothermal perturbations ($\beta =
% 0$) before considering the effect of finite thermal relaxation. 
In order to facilitate a fully analytic discussion, we first introduce a
number of approximations below that allows a simple governing equation
to be derived and compared with previous works.  

\subsection{Relation to the vertically global
  shearing box} 
We connect the linearized equations 
Eq. \ref{lin_mass}---\ref{lin_energy} to the vertically global  
shearing box (VGSB) formalism developed by \cite{mcnally14}. The VGSB
is an extension of the standard shearing box \citep{goldreich65} to
allow the shear flow in the shearing box to be height-dependent. This
is an attempt to capture the effect of vertical shear in a local model
of accretoin disks. The difference between the radially-local
approximation and the VGSB is that the radial disk structure is
ignored in the latter. That is, the  equilibrium density and pressure
field in the VGSB is homogeneous in the radial ($x$) direction, as in
the standard shearing box.    

Thus, the linear problem in the VGSB can be derived from
Eq. \ref{lin_mass}---\ref{lin_energy} by setting 
$\hat{g}_c=0$. The resulting problem is very similar 
to that for local axisymmetric waves in accretion disks already
discussed by \cite{lubow93}, but modified by the presence of vertical
shear and thermal relaxation. This is convenient as it allows us to
directly apply some of their solution methods.    

% but modified by the vertical shear term
% $r\p_z\Omega\delta \delta v_z$. 

After setting $\hat{g}_c=0$ in Eq. \ref{lin_mass}, \ref{lin_vx} and
\ref{lin_energy}, we can eliminate the perturbed velocities between
the linearized equations to obtain a pair or ordinary differential
equations for $W,Q$ as
\begin{align}
  &\frac{\sigma^2}{c_s^2}Q + \frac{\sigma^2k_x^2}{D}W \notag\\ 
  &=\left(\frac{\ii
      k_x r}{D}\frac{d\Omega^2}{dz} -
    \frac{d\ln{\rho}}{dz}\right)\left[\frac{dW}{dz} +
    \frac{d\ln{\rho}}{dz}\left(W-Q\right)\right] \notag\\
  &- \frac{d^2W}{dz^2} - \frac{d\ln{\rho}}{dz}\left(\frac{dW}{dz} -
    \frac{dQ}{dz}\right) - \frac{d^2\ln{\rho}}{dz^2}\left(W-Q\right),\label{ode_w}\\
  &\sigma^2W - \frac{\gamma}{\Gamma}\sigma^2 Q +
  \frac{\ii\sigma}{t_c}\left(W-\frac{Q}{\Gamma}\right)\notag\\
  &=c_s^2\frac{d\ln{\rho}}{dz}\left(\frac{\gamma}{\Gamma} - 1\right) 
  \left[\frac{dW}{dz} + \frac{d\ln{\rho}}{dz}\left(W-Q\right)\right],\label{ode_Q} 
\end{align}
where
\begin{align}
  D \equiv \kappa^2 - \sigma^2,
\end{align} 
and we have replaced the notation $\p_z$ by $d/dz$ since we are
considering solutions at a fixed radius. 

The VGSB is a self-consistent framework but extrapolating results 
obtained from the VGSB to the global disk has limitations. This is
because within the VGSB framework, $d\Omega/dz$ can be freely
specified. In effect, by adopting the VGSB framework, we are studying
dynamics in the standard shearing box subject to an additional
azimuthal force $\delta v_zr d \Omega/dz$ with $d\Omega/dz$ as an
input parameter. For consistency with the global disk, we choose
$d\Omega/dz$ in the VGSB to be that due to the thermal structure in
the global disk (Eq. \ref{vertical_shear}). However, 
% This is different to the global problem because in
% that case, the vertical shear is a result of the 
the background radial disk structure is ignored elsewhere in the VGSB
(which is done by setting $\hat{g}_c=0$).      

We further discuss caveats of the VGSB in Appendix \ref{global_corr}.
We find that for perturbations with small radial lengthscales
($k_xH\gg 1$) and fast thermal relaxation ($\beta\ll1$), which
characterizes the VSI, the VGSB framework is adequate for the purpose
of a simple analytical discussion. 

\subsection{Low-frequency and nearly-Keplerian
  approximation}\label{approx_gov} 
We now derive a governing equation with further approximations:
\begin{enumerate}
\item We consider eigenvalues such that
  $|\sigma^2|\ll \kappa^2$. Then we can replace $D=\kappa^2 -\sigma^2\to
  \kappa^2$. % The problem simplifies because the eigenvalue now only
  % appears in the last term in Eq. \ref{iso_ode}. %stoll kley see low
  % freq 
  We call this the low-frequency approximation. 
\item We are mainly interested in the effect of vertical shear, i.e. $q\neq
  0$. This is already captured explicitly in the linearized equations through
  the term $d\Omega/dz$. We therefore simplify 
  further by ignoring the vertical dependence of $\Omega$ and $\kappa$
  elsewhere. For clarity, we replace $\kappa^2$ with
  $\Omega_k^2$. We call this the nearly-Keplerian approximation. 
  
\end{enumerate}
Making both of these approximations is equivalent to setting 
$D\to\Omega_k^2$.  After making this replacement, we eliminate $W$ and
$Q$ between  Eq. \ref{lin_vz} and Eq. \ref{ode_w}---\ref{ode_Q},
making use of Eq. \ref{vertical_shear}, to
obtain an equation for $\delta v_z$ for $\Gamma=1$,
\begin{align}
  0 =& \frac{d^2\delta v_z}{dz^2} + \left(1 + \frac{\ii k_x c_s^2
      q}{\Omega_k^2r}\right)\frac{d\ln\rho}{dz}\frac{d\delta
    v_z}{dz} \notag\\
  % &+ \left[\sigma^2\left(\frac{k_x^2}{\Omega_k^2} +
  %     \frac{\chi}{c_s^2}\right) + \left(\chi + \frac{\ii k_x c_s^2
  &+ \left[\frac{\sigma^2}{c_s^2}\left(\frac{k_x^2c_s^2}{\Omega_k^2} +
      \chi\right) + \left(\chi + \frac{\ii k_x c_s^2
        q}{\Omega_k^2r}\right)\frac{d^2\ln\rho}{dz^2}\right. \notag\\
  &\phantom{+=}\left.-
    \frac{c_s^2}{\Omega_k^2}\left(\frac{d\ln\rho}{dz}\right)^2\left(k_x^2 -
      \frac{\ii k_x q}{r}\right)
    \left(1-\chi\right) 
  \right]\delta v_z,\label{vertiso_gov}
\end{align}
where
\begin{align}
  \chi \equiv \frac{1-\ii\sigma t_c}{1-\ii\sigma t_c\gamma}.
\end{align}

It is convenient to introduce the dimensionless variables
\begin{align}
  \hat{z} = z/H,\quad \hat{k}=k_xH, \quad \hat{\sigma} =\hat{\omega} +
  \ii\hat{\nu}= \sigma/\Omega_k,
\end{align}
where $\hat{\omega}$ and $\hat{\nu}$ are real. 
Eq. \ref{vertiso_gov} becomes 
\begin{align}
  0=& \delta v_z ^{\prime\prime} + \left(1 + \ii\epsilon
    q\hat{k}\right)\ln\rho^{\prime}\delta v_z^\prime \notag\\
  &+
  \left[\hat{\sigma}^2\left(\hat{k}^2+\chi\right) +
    \left(  \chi + \ii \epsilon q\hat{k}\right)\ln\rho^{\prime\prime}\right.
  \notag\\
  &\left.- \ln\rho^{\prime
      2}\left(\khat^2 -
      \ii\epsilon
      q\hat{k}\right)\left(1 - \chi\right)\right]\delta v_z,\label{vertiso_gov_nondim}
\end{align}
where $^\prime$ denotes $d/d\zhat$. Eq. \ref{vertiso_gov_nondim} will
be the basis for our discussion of vertically isothermal disks. 

In deriving Eq. \ref{vertiso_gov}, the vertical dependence of
$D=\kappa^2(z)-\sigma^2$ was ignored because the replacement
$D\to\Omega_k^2$ was made before eliminating variables in favor of
$\delta v_z$. In Appendix \ref{adia_improve} we show that making this
replacement after eliminating variables, which involves $dD/dz$,
introduces terms $O(\epsilon^2)$. Since we are interested in 
thin disks ($\epsilon\ll 1 $), e.g. protoplaneary disks where 
$\epsilon\lesssim 0.1$ \citep{chiang10}, these terms introduce
unnecessary complexity for the present discussion. However, this
approximation will be relaxed in our numerical study.  


Note that we can generalize Eq. \ref{vertiso_gov_nondim} non-Keplerian
rotation profiles by redefining $\khat$ as $k_xc_s/\kappa$,
$\hat{\sigma}$ as $\sigma/\kappa$, and  $\epsilon$ as $c_s/r\kappa$.     





      

% This expression can be simplified in the limit $|q|\ll 1$, 
%  \begin{align}\label{simple_growth}
%    \frac{\imag(\sigma)}{\Omega_k} = \pm \sqrt{M} 
%    \frac{q\epsilon\hat{k}}{2\sqrt{1+\hat{k}^2}} \quad\quad (|q|\ll 1), 
%  \end{align}
%  with $M$ being an integer. 

% In Appendix \ref{pert_theory} we give an alternative method to
% infer instability in the presence of vertical shear without explicitly
% solving the governing equation. 

