\subsection{Reduced equation for vertically isothermal disks}\label{sec:simplified}
%We begin with a discussion of vertically isothermal disks
%where $\Gamma=1$, then $c_s$ is a constant in the radially-local
%approximation. 
%we first introduce a
%number of approximations below that allows a simple governing equation
%to be derived and compared with previous works. 
%However, all of
%the approximations described below are relaxed in our numerical
%study.  

Our simplified model starts with  Eqs.\ \ref{lin_all} 
and makes the following additional simplifications:

\begin{enumerate}
 
 \item We set $\Gamma = 1$, focusing on vertically isothermal disks.
  
\item  We set $\zeta = 0$,  neglecting terms with an explicit dependence on the 
radial structure of the equilibrium disk.  Vertical shear, which implicitly depends on the radial
temperature gradient, is retained.  This fully-radially-local approximation is also made in the
 `vertically global shearing box' of \citetalias{mcnally14}. 

\item We make the low frequency approximation, assuming $|\upsilon^2|\ll \kappa^2,\,\Omega^2$. 
%keeping only the lowest order terms in $|\upsilon|/\kappa$.  
Similar to the incompressible \citepalias{goldreich67} or anelastic \citepalias{nelson13,barker15} approximations,
the low frequency approximation filters acoustic waves in favor of
inertial-gravity waves \citep{lubow93}.  

\item We make the Keplerian approximation, setting $\Omega \rightarrow \OmK$ 
and $\kappa \rightarrow \OmK$, but retaining the vertical dependence 
in the crucial vertical shear term,
  $\p_z\Omega$. 
  
\item We consider thin disks with $h\ll 1$ 
  %and take the $z \ll r$ limit to 
  %obtain 
  and use the Gaussian approximation, Eq.\ \ref{rhoisothin}, for the equilibrium density field.  

\end{enumerate}
% These approximations lead to Eq. \ref{full_ode} which, for a thin disk
% $h\ll 1$, simplifies further to  
% %The result for $\Gamma=1$ is
% \begin{align}
%   0 =& \frac{d^2\delta v_z}{dz^2} + \left(1 + \frac{\ii k_x c_s^2
%       q}{\OmK^2r}\right)\frac{d\ln\rho}{dz}\frac{d\delta
%     v_z}{dz} \notag\\
%   % &+ \left[\upsilon^2\left(\frac{k_x^2}{\OmK^2} +
%   %     \frac{\chi}{c_s^2}\right) + \left(\chi + \frac{\ii k_x c_s^2
%   &+ \left[\frac{\upsilon^2}{c_s^2}\left(\frac{k_x^2c_s^2}{\OmK^2} +
%       \chi\right) + \left(\chi + \frac{\ii k_x c_s^2
%         q}{\OmK^2r}\right)\frac{d^2\ln\rho}{dz^2}\right. \notag\\
%   &\phantom{+=}\left.-
%     \frac{c_s^2}{\OmK^2}\left(\frac{d\ln\rho}{dz}\right)^2\left(k_x^2 -
%       \frac{\ii k_x q}{r}\right)
%     \left(1-\chi\right) 
%   \right]\delta v_z,\label{vertiso_gov}
% \end{align}
% where
% \begin{align}
%   \chi \equiv \frac{1-\ii\upsilon t_c}{1-\ii\upsilon t_c\gamma}.
% \end{align}

 In terms of the dimensionless variables
\begin{align}\label{nondim}
  \hat{z} = z/H,\quad \hat{k}=k_xH, \quad \hat{\upsilon} =\hat{\omega} +
  \ii\hat{\sigma}= \upsilon/\OmK,
\end{align}
where $\hat{\omega}$ and $\hat{\sigma}$ are real, the above
approximations lead to a single second order ODE, 
% Eq. \ref{vertiso_gov} becomes 
% \begin{align}
%   0=& \delta v_z ^{\prime\prime} + \left(1 + \ii h
%     q\hat{k}\right)\ln\rho^{\prime}\delta v_z^\prime \notag\\
%   &+
%   \left[\hat{\upsilon}^2\left(\hat{k}^2+\chi\right) +
%     \left(  \chi + \ii h q\hat{k}\right)\ln\rho^{\prime\prime}\right.
%   \notag\\
%   &\left.- \ln\rho^{\prime
%       2}\left(\khat^2 -
%       \ii h
%       q\hat{k}\right)\left(1 - \chi\right)\right]\delta v_z,\label{vertiso_gov_nondim}
% \end{align}
% where $^\prime$ denotes $d/d\zhat$.
%\subsection{Thin-disk approximation}\label{analytic_relax}
%The governing equation, Eq. \ref{vertiso_gov_nondim}, is then   
\begin{align}
  \delta v_z^{\prime\prime} - z A\delta v_z^\prime +
  (B - C\zhat^2)\delta v_z = 0,\label{nearly_iso_explicit}
\end{align}
where $^\prime$ denotes $d/d\zhat$ and 
\begin{subequations}\begin{align}
  &A \equiv 1 + \ii h q \hat{k},\\
  &B \equiv \hat{\upsilon}^2\left(\chi + \hat{k}^2\right) -
  \left(\chi + \ii h q \hat{k}\right),\\
  &C \equiv \left(1-\chi\right)\left(\hat{k}^2 - \ii
    h q\hat{k}\right), 
\end{align}\end{subequations}
where
\begin{align}\label{chi}
\chi = \frac{1-\ii\hat{\upsilon}\beta}{1-\ii\hat{\upsilon}\beta\gamma}.
\end{align}
The derivation of Eq. \ref{nearly_iso_explicit}  is detailed in
Appendix \ref{adia_improve}. 
 
%We remark that Eq. \ref{nearly_iso_explicit} is in the same 
%form as Eq. 41 in \cite{lubow93}, which describe adiabatic axisymmetric waves in
%a vertically isothermal disk without vertical shear. The two equations
%become identical after  setting $q=0$ (no vertical shear) and
%$\chi=1/\gamma$ (adiabatic flow) in our Eq. \ref{nearly_iso_explicit},
%and making the low-frequency, Keplerian approximation in
%\citeauthor{lubow93} after a change of variables.    

 In Appendix \ref{global_corr}  we discuss the limits of these approximations.
% and show that they are valid in the parameter space of interest for the VSI.
In particular we point out that the fully-radially-local approximation is only valid for
short cooling times, $\beta \lesssim O(1)${\bf, which is the 
  regime in which the VSI operates, as demonstrated below. 
  On the other hand, for 
  $\beta\gtrsim O(1)$, this approximation can introduce artificial
  instability due to the non-self-consistent neglect of global radial gradients (see \S\ref{analytic_adia}).}

%vertically isothermal disk lacks a surface 

% \subsection{Relation to the vertically global
%   shearing box}\label{vgsb_approx} 
% We connect the linearized equations 
% Eq. \ref{lin_mass}---\ref{lin_energy} to the vertically global  
% shearing box (VGSB) formalism developed by \cite{mcnally14}. The VGSB
% is an extension of the standard shearing box \citep{goldreich65} to 
% background shear flows that are height dependent. 
% As in the  standard shearing box, in the VGSB the equilibrium disk
% structure is assumed to be homogeneous in the radial ($x$) direction.  

% Thus, the linear problem in the VGSB can be derived from
% Eq. \ref{lin_mass}---\ref{lin_energy} by setting 
% $\zeta=0$. The resulting problem is very similar 
% to that for local axisymmetric waves in accretion disks already
% discussed by \cite{lubow93}, but modified by the presence of vertical
% shear and thermal relaxation. This is convenient as it allows us to
% directly apply some of their solution methods.    

% After setting $\zeta=0$ in Eq. \ref{lin_mass}, \ref{lin_vx} and
% \ref{lin_energy}, we can eliminate variables 
% to obtain a pair or ordinary differential
% equations for $W,Q$ as
% \begin{align}
%   &\frac{\upsilon^2}{c_s^2}Q + \frac{\upsilon^2k_x^2}{D}W \notag\\ 
%   &=\left(\frac{\ii
%       k_x r}{D}\frac{d\Omega^2}{dz} -
%     \frac{d\ln{\rho}}{dz}\right)\left[\frac{dW}{dz} +
%     \frac{d\ln{\rho}}{dz}\left(W-Q\right)\right] \notag\\
%   &- \frac{d^2W}{dz^2} - \frac{d\ln{\rho}}{dz}\left(\frac{dW}{dz} -
%     \frac{dQ}{dz}\right) - \frac{d^2\ln{\rho}}{dz^2}\left(W-Q\right),\label{ode_w}\\
%   &\upsilon^2W - \frac{\gamma}{\Gamma}\upsilon^2 Q +
%   \frac{\ii\upsilon}{t_c}\left(W-\frac{Q}{\Gamma}\right)\notag\\
%   &=c_s^2\frac{d\ln{\rho}}{dz}\left(\frac{\gamma}{\Gamma} - 1\right) 
%   \left[\frac{dW}{dz} + \frac{d\ln{\rho}}{dz}\left(W-Q\right)\right],\label{ode_Q} 
% \end{align}
% where
% \begin{align}
%   D \equiv \kappa^2 - \upsilon^2,
% \end{align} 
% and we have replaced the notation $\p_z$ by $d/dz$ since we are
% considering solutions at a fixed radius. 

% Extrapolating results obtained from the VGSB to the global disk has
% limitations. This is because within the VGSB framework, $d\Omega/dz$
% can be freely specified. 
% We choose $d\Omega/dz$ to be that due to the thermal structure in 
% the global disk (Eq. \ref{vertical_shear}). However, background radial
% gradient terms are ignored elsewhere in the VGSB (which is done by
% setting $\zeta=0$).       


% We further discuss caveats of the VGSB in Appendix
% \ref{global_corr}. We find that for fast thermal relaxation
% ($\beta\ll1$) and perturbations with small radial
% lengthscales ($k_xH\gg 1$), which characterizes the VSI, the VGSB
% framework is adequate for the purpose of a simple analytical
% treatment.  
