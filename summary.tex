\section{Summary and discussion}\label{summary}
%what have we done, relation to previous work 
In this paper we have studied the linear axisymmetric vertical shear
instability (VSI) in astrophysical disks that arises from the
height-dependence of the equilibrium angular velocity profile
$\Omega(r,z)$ of a baroclinic disk. The VSI was analyzed and
explained earlier in the local limit by \cite{urpin98} and
\cite{urpin03}; and simulated by \cite{arlt04} and more recently by
\cite{nelson13}.     

We generalized previous vertically-global linear stability calculations
of the VSI \citep{nelson13,mcnally14}, which assumed isothermal
perturbations, to include an energy equation. Our effort may also be
regarded as an extension of \cite{urpin03}, which included an energy
equation, by relaxing the Boussinesq and vertically-local
approximations adopted in that analysis.    

We added a source term in the energy equation which relaxes the disk
temperature to its initial value on a timescale $t_c=\beta\Omega_k^{-1}$. This
form of thermal relaxation was also employed in the global numerical
simulations of \cite{nelson13}, and allows the thermodynamic response
of the disk to be controlled.  

In accordance with the above studies, which show that the VSI occurs for 
perturbations with small radial lengthscales, we applied the 
radially-local approximation in the linearized fluid equations. For
analytic discussion only, we further simplified the linear problem by
ignoring the background radial disk structure. This is equivalent to adopting
the `vertically-global shearing box' (VGSB) formalism developed by
\cite{mcnally14}, who showed the linear VSI studied by
\cite{nelson13} in a locally isothermal disk can be captured by the
VGSB. We find the VGSB framework is appropriate for $\beta\lesssim1$. 
This is convenient because previous works show that rapid thermal
relaxation ($\beta\ll1$) is generally required for the VSI. However,
the VGSB framework is not valid in the adiabatic limit. 
   
%OK framework for the regime of interest 
% The presence of a
% vertical shear, but 
% is not consistent with the
% global 

We mostly focused on vertically isothermal disks ($\Gamma=1$) because
it permits a largely analytical discussion. Our analysis is an
extension of the linear models considered in \cite{lubow93} by
adding vertical shear and (rapid) thermal relaxation, but in the limit of low-frequency
perturbations in nearly-Keplerian disks. These approximations were
made for simplicity, but also motivated by previous work
\citep{nelson13}.   

For isothermal perturbations ($\beta=0$) we explicitly
demonstrated instability  associated with vertical shear, and that the 
VSI growth rate is limited by the maximum vertical shear within the 
vertical domain of interest. The fundamental VSI mode is characterized
by one node in the density perturbation at the mid-plane
(i.e. anti-symmetric about $z=0$), which corresponds to a vertical
velocity perturbation  $\delta v_z\sim\mathrm{constant}$ near
$z=0$. 

% In the opposite limit of adiabatic perturbations ($\gamma>1$
% and $\beta\to\infty$), we find VSI growth rates are dramatically
% reduced on account of a stabilizing 
% vertical entropy gradient. The VSI is not completely
% surpressed because vertical shear dominates over buoyancy 
% sufficiently close the disk midplane. However, perturbations grow
% slowly and is constrained to regions immediately about the disk
% midplane. Thus, the adiabatic VSI is likely unimportant in practice. 

For non-isothermal  perturbations with thermal relaxation ($\gamma>1$,
$\beta>0$) , the general dispersion relation (Eq. \ref{relax_disp}) is
complicated and requires a numerical treatment. However, for the 
fundamental VSI some simplification is  possible. In particular, we
quantified the critical thermal relaxation timescale
$\beta_\mathrm{crit}=\epsilon|q|/(\gamma-1)$, below which small-scale
perturbations may be regarded as isothermal and the fundamental VSI
operates.  We expect $\beta<\beta_\mathrm{crit}$ to be a sufficient
condition for the VSI to operate. We believe this explicit expression for
$\beta_\mathrm{crit}$ is the most useful result from our
investigation.  

We also solved the linear stability problem numerically using a
pseudo-spectral approach. For numerical calculations we accounted for
the background radial disk structure for consistency. We focused
on the fundamental mode to compare with our analytical
discussion, and because \cite{nelson13} found that this mode 
eventually dominate their simulations. Our numerical solutions are
mostly consistent with our analytic expectations, including the 
dependence of $\beta_\mathrm{crit}$ on disk parameters. 

% as given
% by Eq. \ref{iso_vsi_cond}, provides a good measure of the required
% th 
%  except for perturbations with small radial
% wavenumber ($\khat\to0$) where the low-frequency approximation is
% inappropriate. The radially local model is also not valid for such modes.   

We briefly studied vertically non-isothermal disks using our linear 
code. We confirm the finding by \cite{nelson13} that in the absence
of vertical entropy gradients, the VSI operates and is insensitive to
thermal relaxation. When a stable stratification is introduced, the
rapid reduction in VSI growth rates by a finite thermal relaxation
timescale is qualitatively similar to that observed for vertically
isothermal disks. Based on numerical results, we conjecture that the
critical thermal relaxation timescale is given by
$\beta_\mathrm{crit}=\epsilon|q|/(\gamma-\Gamma)$ in vertically 
non-isothermal disks.  

%application to MMSN 

We applied our linear results to assess the applicability of the VSI in
protoplanetary disks, specifically the Minimum Mass Solar Nebulae (MMSN),
using disk models described in \cite{chiang10}. We find the MMSN is
unstable to the fundamental VSI at $10$s of AU, with characteristic
growth times of $10$s of orbits. However, we note the strong
temperature dependence of the physical thermal relaxation timescale
in the disk model ($\beta$ decreases as $T^{-5}$ for perturbations with small
radial scale). Thus, increasing the disk temperature even slightly can significantly 
shorten the typical growth times for the VSI. This is also attributed
to a larger value of $\beta_\mathrm{crit}\propto\epsilon$, since increasing $T$ also
increases the aspect-ratio $\epsilon$. 


%relation with kley simulations


\subsection{Caveats and outlooks} 
We highlight below some improvements to the present  
linear models in future works:  

\emph{Realistic thermal evolution.} We have treated the thermal
relaxation time $t_c$ as a constant for simplicity. However, a
physical expression for $t_c$ will depend on 
$z$. For example, the Newtonian cooling timescale
(\S\ref{newton_cool}) will depend on $z$ for vertically non-isothermal
disks, and through the opacity since 
$\kappa_d=\kappa_d(\rho,T)$ in general. 
A proper treatment for radiative diffusion, on the other hand, will
ultimately involve the Laplacian operator. In this case additional
boundary conditions must be supplemented. We will consider this in a
future work.  

%need to compare kley sims 

\emph{Additional physics.} Depending on where the VSI operates in a
disk, additional effects may become important. For example, in the
outer parts of a protoplanetary disk, where we estimate the VSI 
operates, disk self-gravity (SG) may play a role. Specifically, the vertical
shear profile depends on the vertical density stratification
(Eq. \ref{vertical_shear}) which can be enhanced by vertical SG. This
may promote the VSI. Inclusion of the Poisson equation should
therefore be considered.  

We have neglected viscosity. \cite{nelson13} have shown that an
alpha viscosity of $\lesssim O(10^{-4})$ is required for the VSI to
operate. It is of interest to quantify the effect of viscosity
on the linear VSI. This would help identify, more accurately, the
appropriate regions in a protoplanetary disk that are sufficiently
laminar to develop the VSI.  This problem is, however, complicated by
the fact that proper inclusion of viscosity introduces viscous heating
in the energy equation. 


\emph{Radial dependence.} We have adopted the radially-local approximation 
throughout this paper. This is based on previous works, including
nonlinear simulations, which show that the VSI is relevant for
perturbations with small radial lengthscales. Although this simplifies
the linear problem to ordinary differential equations, this approach
introduces the wavenumber $k_x$ as a free parameter. The effect
of radial boundary conditions cannot be explored. Future
generalizations of the linear problem should retain the radial
dependence properly and solve the two-dimensional $(r,z)$ partial 
differential equation.     

% Extrapolating our results from the shearing box to a global disk  
% has an inconsistency. Namely, the equilibrium temperature in the
% box only depends on height (if at all). However, it is precisely the
% radial temperature (or entropy) gradient in the global disk that
% gives rise to vertical shear. We do not expect this to be a serious issue for 
% perturbations with small radial wavelengths, 
% but will need to be resolved by adopting global disk models  
% if one wishes to generalize the VSI to large-scale perturbations. 


% \emph{Theoretical generalizations.} Our analytical discussion
% (\S\ref{analytical}) can be extended in several ways. One could
% relax the low-frequency approximation to improve the agreement between
% numerical and analytical eigenfrequencies at small $\khat$. However,
% such modes are radially more global, and would need a global
% cylindrical disk model to study them. Another minor improvement is to 


%analytical extensions -> nonisothermal, all frequency, non-keplerian,
%diffusion 
%viscosity self-gravity 
%disk models: polytropic connected with isothermal 
