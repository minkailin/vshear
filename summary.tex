\section{Summary and discussion}\label{summary}
In this paper we have studied the axisymmetric vertical shear 
instability (VSI) in astrophysical disks that arise from the 
height dependence of the equilibrium angular velocity profile,  
$\p_z\Omega\neq 0$. Vertical shear is generally
expected in astrophysical disks due to baroclinicity ($\nabla
P\times\nabla \rho \neq 0$). This may be due
to a radial variation in the disk temperature, or more generally in
the disk entropy. However, astrophysical disks are also typically
sub-adiabatically stratified for stability against vertical
convection ($N_z^2\geq 0$). In this case, the VSI requires a short
thermal or cooling timescale $t_c$ in order to operate.  

Physically, a short thermal timescale is needed to reduce the  
stabilizing effect of buoyancy forces. Because vertical shear is weak
in thin disks, very rapid cooling, corresponding to $t_c\Omega\ll 1$, 
is generally needed for the VSI, otherwise buoyancy is strongly
stabilizing. Our primary goal was to quantify this requirement,
thereby provide a simple way to assess whether or not the VSI is
relevant in realistic disks.  

To do so, we generalized previous vertically global, radially local linear 
calculations of the VSI \citepalias{nelson13,mcnally14,barker15}, which
assumed isothermal perturbations, to include an energy equation with 
a source term in that relaxes the disk 
temperature to its initial value on a timescale
$t_c$. 
% This `thermal relaxation' model allows
% the thermodynamic response of the disk to be controlled by varying
% $t_c$, and has been employed in previous numerical simulations of the
% VSI \citepalias{nelson13}.  
%In accordance with the above studies, which show that the VSI occurs for 
%perturbations with small radial lengthscales, we applied the 
%radially local approximation in the linearized fluid equations. 
We mostly focused on vertically isothermal disks as this is the
relevant case for protoplanetary disks (PPDs) passively 
irradiated by its central star \citep{chiang97}. It also permits a
simplified analytical discussion by extending the 
linear models considered in \cite{lubow93} by adding vertical shear
and (rapid) cooling. We obtained explicit solutions 
describing the VSI with thermal relaxation in the thin disk limit.  

For a vertically isothermal, sub-adiabatically stratified disk we
determined the thermal timescale   
\begin{align*}
  t_\mathrm{crit} = \frac{ h |q|}{\gamma-1}\OmK^{-1}, 
\end{align*}
below which the VSI operates effectively. 
%While $t_\mathrm{crit}$ was
%derived for the lowest order VSI mode --- the `fundamental  
%corrugation mode' --- we find from numerical calculations that
%$t_\mathrm{crit}$ can be used to characterize the general
%stabilization of the VSI by a finite thermal timescale. %That is, all
%VSI growth rates are reduced significantly as $\beta$ is increased
%from 
%zero to $\beta_\mathrm{crit}$.    
We have thus provided a quantitative measure for the thermal
timescale requirement for the VSI in astrophysical 
disks. The thermal timescale must be shorter than the disk
dynamical timescale by $O( h)$, which is 
quite short for thin disks ($ h\ll 1$), because vertical shear is weak.    

We applied our linear models to assess the applicability of the VSI in
PPDs. We considered the Minimum Mass Solar Nebula 
(MMSN) described in \cite{chiang10} and   
modeled the thermal timescale $t_c$ based on realistic
disk properties and dust opacity. 
%This results in $t_c$ being a
% function of position and the perturbation lengthscale of interest. 
We find the MMSN is unstable to the VSI 
from $\sim 5$AU to $\sim50$AU with characteristic growth times of
$\sim 30$ orbits. It is possible to have the VSI at smaller
radii, but only on radial lengthscales signficantly shorter
than the disk scaleheight. %However, such small-scale perturbations may 
%be subject to viscous decay in a real disk.   

Although $t_c$ is not vertically constant in the 
MMSN as assumed in our analysis, we find that evaluating the condition
$t_c< t_\mathrm{crit}$ at the disk midplane gives an indication of the
radii at which the VSI operates in the MMSN. This simple criterion is
thus a practical first step in estimating the relevance of the VSI in
realistic astrophysical disks without resorting to linear
calculations or non-linear numerical simulations.   

\subsection{Caveats and outlooks} 
We highlight below some outstanding issues that should be addressed in
future works. 

\subsubsection{Viscosity}\label{caveats_visc}
% Depending on where the VSI operates in a
% disk, additional effects may become important. For example, in the
% outer parts of a PPD, where we estimate the VSI 
% operates, disk self-gravity (SG) may play a role. Vertical
% SG may favor the VSI by increasing $|\p_z\Omega|$ through
% enhancing the density stratification 
% (Eq. \ref{vertical_shear}). %sg in pert? no because kx>>1. but maybe for kx~1 

We have neglected viscosity. 
% while \citetalias{barker15} find that inclusion of viscosity is needed for 
% convergence when studying surface modes in radially global disk
% models. %but not body modes? 
% Viscosity may also affect the
% thermodynamic requirement for the VSI, because viscous heating
% needs to be self-consistently included in the energy equation. This
% should be investigated in future works.   
An important effect of viscosity is setting a minimum wavelength (or 
maximum wavenumber) for which the VSI is relevant. Disturbances with
large $\khat\equiv k_xH$ may be subject to viscous decay. Adopting the standard
precription for the kinematic viscosity $\nu = \alpha c_s H$
\citep{shakura73}, where $\alpha$ is the dimensionless viscosity
parameter, the viscous timescale for a perturbation lengthscale  
$l\sim 1/k_x$ is $t_\mathrm{visc} = 1/\alpha\khat^2\Omega $. 
Setting $t_\mathrm{visc}$ to be longer than the
minimum VSI growth timescale $1/h\Omega$ implies 
$\khat^2\lesssim h/\alpha$. Inserting $h \sim 0.05$,
we find only modes with $\khat\lesssim 20$ are relevant for
$\alpha\sim 10^{-4}$ or $\khat\lesssim 70$ for $\alpha\sim
10^{-5}$. \citepalias[These small viscosity values are required for
the VSI to operate, see][]{nelson13}.

%\citetalias{nelson13} find that a
%dimensionless viscosity of $\alpha\lesssim O(10^{-4})$ is required for the
%VSI. 

The above argument justifies our focus on moderate
wavenumbers, $\khat=O(10)$, which precludes extremely small-scale
disturbances such as surface modes. Nevertheless, future calculations
should explicitly account for viscosity. In particular, viscous
heating may influence the thermodynamic dependence of the VSI. 

% needs to be self-consistently included in the energy equation. 






% It is of interest to quantify the effect of viscosity 
% on the linear VSI. This would help identify, more accurately, the
% appropriate regions in a protoplanetary disk that are sufficiently
% laminar to develop the VSI.  This problem is, however, complicated by
% the fact that proper inclusion of viscosity introduces a meridional
% flow in the equilibrium state, as well as viscous heating
% in the energy equation. 


\subsubsection{Convective overstability}
We have focused on parameter regimes relevant to the VSI,
namely perturbations with small radial lengthscales 
($k_xH\gtrsim 1$) in disks with short thermal timescales
($t_c\Omega\ll 1$). This excludes the `convective overstability'
recently discussed by \cite{klahr14} and \cite{lyra14} in vertically
unstratified disk models.   

The convective overstability corresponds to slowly growing epicyclic motions.
It is most effective for $t_c\Omega \sim 1$ and disturbances with
$|k_x|\ll |k_z|$, where $k_z$ is a vertical wavenumber. These
conditions are mutually exclusive with the VSI.  
The maximum growth rate for convective overstability is given by 
\begin{align}
  \sigma_\mathrm{conv,max} = -\frac{N^2_r}{4\Omega}\label{max_conv_gen}
\end{align}
\citep{lyra14}, where $N_r^2 = -(c_p\rho)^{-1}\p_rP\p_rS$ is the
square of the radial buoyancy frequency
(cf. Eq. \ref{nzsq_def}). Thus, growth requires $N_r^2<0$.  
%Evaluating Eq. \ref{max_conv_gen} at the midplane of our disk models,
%we find  
%\begin{align}
%  \sigma_\mathrm{conv,max} =
%  -\frac{ h^2\OmK}{4\gamma\Gamma}\left(p+q\right)
%  \left[\left(\gamma-1\right)p-q\right]. \label{max_conv}
%\end{align}
In fact, for the MMSN considered in this work, convective
overstability does not apply because 
$N_r^2>0$.   

For more general disk structures convective overstability is
possible, in principle. The models adopted for our numerical study in
\S\ref{numerical} do satisfy $N_r^2<0$, but convective overstability
require perturbations with long radial lengthscales,
$k_xH\ll 1$, for which radially global disk models would be more 
self-consistent. 

\subsubsection{Improved thermal evolution} 
We assumed non-adiabatic effects can be represented by a thermal
relaxation timescale $t_c$. For  
application to PPDs we connected $t_c$ to the disk
properties such as surface density, temperature and opacity, making
$t_c$ depend on position and the perturbation 
lengthscales. One might improve the realism further by generalizing
the opacity model adopted here to be density (and hence height)
dependent.  
%here to be density-dependent,  i.e. $\kappa_d =
%% \kappa_d(\rho, T)$. 
However, going beyond thermal relaxation requires a
proper treatment of radiative diffusion. 
% done roughly in this work -
% a proper study 
% involves a Laplacian operator and additional boundary
% conditions must be supplemented. 
We will consider these improvements in a
follow-up study. %additional heat source 
 



