\section{Summary and discussion}\label{summary}
%what have we done, relation to previous work 
In this paper we have studied the linear axisymmetric vertical shear
instability (VSI) in astrophysical disks that arises from the
height-dependence of the equilibrium angular velocity profile
$\Omega=\Omega(r,z)$ in a baroclinic disk. We examined how
the linear instability depends on disk thermodynamics and applied our
results to assess the relevance of the VSI in realistic protoplanetary
disks.  

% The VSI was analyzed earlier in 
% the local limit by \cite{urpin98} and 
% \cite{urpin03} and simulated by \cite{arlt04}. Recent non-linear
% numerical simulations by \cite{nelson13} and \cite{stoll14} indicate
% the VSI generally requires a short thermal timescale. 

We generalized previous vertically global linear stability calculations
of the VSI \citep{nelson13,mcnally14}, which assumed isothermal
perturbations, to include an energy equation. Our effort may also be
regarded as an extension of \cite{urpin03}, which did include an
energy equation, by relaxing the Boussinesq and vertically local
approximations adopted in that analysis.      

We included a source term in the energy equation which relaxes the disk
temperature to its initial value on a timescale
$t_c=\beta\Omega_k^{-1}$. This approach to model thermal relaxation was also
employed in the global numerical simulations of \cite{nelson13}, and
allows the thermodynamic response of the disk to be controlled by
varying the parameter $\beta$. However, it is possible to model
$\beta$ self-consistently by relating it to thermal timescales
expected in realistic disks.  

In accordance with the above studies, which show that the VSI occurs for 
perturbations with small radial lengthscales, we applied the 
radially local approximation in the linearized fluid equations. For
analytic discussion only, we further simplified the linear problem by
ignoring the background radial disk structure. This is equivalent to adopting
the `vertically-global shearing box' (VGSB) formalism developed by
\cite{mcnally14}, who showed the linear VSI studied by
\cite{nelson13} in a locally isothermal disk can be captured by the
VGSB. We find the VGSB framework is appropriate for non-isothermal
perturbations provided $\beta\lesssim1$.  
This is convenient because rapid thermal relaxation ($\beta\ll1$) is
generally required for the VSI.  However, the VGSB framework is not
appropriate in the adiabatic limit ($\beta \gg 1$). 
   
%OK framework for the regime of interest 
% The presence of a
% vertical shear, but 
% is not consistent with the
% global 

We mostly focused on vertically isothermal disks ($\Gamma=1$), since
this is the relevant case for protoplanetary disks that are passively irradiated by its central star 
\citep{chiang97}. It also permits a largely analytical discussion.
Our analysis is an extension of the linear models considered in
\cite{lubow93} by  adding vertical shear and (rapid) thermal
relaxation, but in the limit of low-frequency perturbations in
nearly-Keplerian disks. These approximations were made for simplicity,
but also motivated by recent numerical simulations
\citep{nelson13}. We obtained explicit solutions for the VSI with
thermal relaxation in the thin disk limit. 

% For isothermal perturbations ($\beta=0$) we explicitly
% demonstrated instability  associated with vertical shear, and that the 
% VSI growth rate is limited by the maximum vertical shear within the 
% vertical domain of interest. 
% For perturbations with finite thermal relaxation, the general
% dispersion relation (Eq. \ref{relax_disp}) is quite
% complicated. However, for the fundamental VSI some simplification is
% possible. In particular, 

The fundamental VSI mode is characterized by a vertical velocity
perturbation with zero nodes and $\delta v_z\sim\mathrm{constant}$
near the mid-plane, and by one node in the density perturbation at the
midplane  (i.e. anti-symmetric about $z=0$).  These have been referred
to as `corrugation modes' by \cite{nelson13}, and were found to
dominate their numerical simulations, as well as that of
\cite{stoll14}. Furthermore, when computing unstable modes for a
protoplanetary disk model with thermal relaxation based on realistic
disk properties, we find the fundamental mode to be the dominant mode,
 and it is robust against boundary conditions. The fundamental mode is
 thus likely the most relevant VSI mode in protoplanetary disks.  

We quantified the critical thermal relaxation timescale 
$\beta_\mathrm{crit}=\epsilon|q|/(\gamma-1)$, below which small-scale
perturbations may be regarded as isothermal and the fundamental VSI
operates. %  We expect $\beta<\beta_\mathrm{crit}$ to be a sufficient
% condition for the VSI to operate.
We have therefore provided a quantitative measure for the `fast'
thermal relaxation requirement for the VSI in astrophysical disks. 

We also solved the linear stability problem numerically using a
pseudo-spectral approach. For numerical calculations we relax all the
approximations made in our analysis, and accounted for
the background radial disk structure for consistency.  Our
numerical solutions are mostly consistent with our analytic 
expectations, including the dependence of $\beta_\mathrm{crit}$ on
disk parameters.  %for regimes of interest

% as given
% by Eq. \ref{iso_vsi_cond}, provides a good measure of the required
% th 
%  except for perturbations with small radial
% wavenumber ($\khat\to0$) where the low-frequency approximation is
% inappropriate. The radially local model is also not valid for such modes.   

We briefly studied vertically non-isothermal disks using our linear 
code. We confirm the finding by \cite{nelson13} that in the absence
of vertical entropy gradients, the VSI is insensitive to
thermal relaxation. When a stable stratification is introduced, the
rapid reduction in VSI growth rates by a finite thermal relaxation
timescale is qualitatively similar to that observed for vertically
isothermal disks. Based on numerical results, we conjecture that the
critical thermal relaxation timescale is given by 
$\beta_\mathrm{crit}=\epsilon|q|/(\gamma-\Gamma)$ in vertically 
non-isothermal disks.  

%application to MMSN 

We applied our linear models to assess the applicability of the VSI in
protoplanetary disks, using a model for the the Minimum Mass Solar Nebulae
(MMSN) described in \cite{chiang10}. We  
modeled the thermal relaxation timescale $\beta$ based on realistic
disk properties and dust opacity, which results in a $\beta$ that
depends on both the perturbation lengthscale as well as the height
from the midplane. We find the MMSN is unstable to the VSI
from $\sim 5$AU to $\sim50$AU with characteristic growth times of
$\sim 30$ orbits. Although $\beta$ is not vertically constant in the
MMSN as assumed in our analysis, we find that evaluating $\beta <
\beta_\mathrm{crit}$ at the disk midplane gives a rough measure of the
radii at which the VSI operates in the MMSN.  This simple criteron may
be a useful first step to estimate the relevance of the VSI in
realistic astrophysical disks without resorting to non-linear
numerical simulations.  

%  However, we note the strong
% temperature dependence of the physical thermal relaxation timescale
% in the disk model ($\beta$ decreases as $T^{-5}$ for perturbations with small
% radial scale). Thus, increasing the disk temperature even slightly can significantly 
% shorten the typical growth times for the VSI. This is also attributed
% to a larger value of $\beta_\mathrm{crit}\propto\epsilon$, since increasing $T$ also
% increases the aspect-ratio $\epsilon$. 

%relation with kley simulations

\subsection{Caveats and outlooks} 
We highlight below some extensions to the present  
linear models in future works:  

\emph{Improved thermal evolution.} We assumed non-adiabatic effects
can be represented by a thermal relaxation timescale $t_c$. For 
application to protoplanetary disks we connected $t_c$ to the disk
properties, but one can improve the realism further by generalizing the
opacity model adopted here to be density-dependent,  i.e. $\kappa_d =
\kappa_d(\rho, T)$ as described in \cite{bell94}. %two temperature
% structure  
However, going beyond the framework of thermal
relaxation requires a proper treatment for radiative diffusion, which
involves the Lapacian operator and additional boundary conditions
must be supplemented. We will consider these improvements in a
follow-up study. 
 
\emph{Additional physics.} Depending on where the VSI operates in a
disk, additional effects may become important. For example, in the
outer parts of a protoplanetary disk, where we estimate the VSI 
operates, disk self-gravity (SG) may play a role. Specifically, the vertical
shear profile depends on the vertical density stratification
(Eq. \ref{vertical_shear}) which can be enhanced by vertical SG. This
may promote the VSI. Inclusion of the Poisson equation should
therefore be considered.  

We have neglected viscosity. \cite{nelson13} have shown that an
alpha viscosity of $\lesssim O(10^{-4})$ is required for the VSI to
operate. It is of interest to quantify the effect of viscosity
on the linear VSI. This would help identify, more accurately, the
appropriate regions in a protoplanetary disk that are sufficiently
laminar to develop the VSI.  This problem is, however, complicated by
the fact that proper inclusion of viscosity introduces a meridional
flow in the equilibrium state, as well as viscous heating
in the energy equation. 


\emph{Radial dependence.} Based on previous works, including
non-linear simulations, which show that the VSI is relevant for
perturbations with small radial lengthscales, we have adopted the
radially-local approximation throughout this paper.  Although this
simplifies the linear problem to ordinary differential equations, this
approach introduces the wavenumber $k_x$ as a free parameter. The
effect of radial boundary conditions cannot be explored. Future 
generalizations of the linear problem should retain the radial
dependence properly and solve the two-dimensional $(r,z)$ partial 
differential equation eigenvalue problem. 

% Extrapolating our results from the shearing box to a global disk  
% has an inconsistency. Namely, the equilibrium temperature in the
% box only depends on height (if at all). However, it is precisely the
% radial temperature (or entropy) gradient in the global disk that
% gives rise to vertical shear. We do not expect this to be a serious issue for 
% perturbations with small radial wavelengths, 
% but will need to be resolved by adopting global disk models  
% if one wishes to generalize the VSI to large-scale perturbations. 


% \emph{Theoretical generalizations.} Our analytical discussion
% (\S\ref{analytical}) can be extended in several ways. One could
% relax the low-frequency approximation to improve the agreement between
% numerical and analytical eigenfrequencies at small $\khat$. However,
% such modes are radially more global, and would need a global
% cylindrical disk model to study them. Another minor improvement is to 


%analytical extensions -> nonisothermal, all frequency, non-keplerian,
%diffusion 
%viscosity self-gravity 
%disk models: polytropic connected with isothermal 
