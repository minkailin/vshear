\section{Summary and discussion}
%what have we done, relation to previous work 
In this paper we have studied the linear axisymmetric vertical shear
instability (VSI) in astrophysical disks that arises from the
height-dependence of the equilibrium angular velocity profile
$\Omega(r,z)$ of a baroclinic  disk. The VSI was analysed and
explained earlier in the local limit by \cite{urpin98} and
\cite{urpin03}, simulated by \cite{arlt04} and more  recently by
\cite{nelson13}.     

We generalized previous vertically-global linear stability calculations
of the VSI \citep{nelson13,mcnally14}, which assumed
isothermal perturbations, to include an energy equation. Our effort
may also be regarded as an extension of \cite{urpin03}, which 
included an energy equation, by relaxing the Boussinesq and  
vertically-local approximations adopted in that analysis.  

In accordance with the above studies, which show that the VSI   
applies to perturbations with small radial lengthscales, we retained
the radially-local approximation by using the `vertically-global
shearing box' (VGSB) formalism developed by \cite{mcnally14}, who 
showed the linear isothermal VSI studied in \cite{nelson13}, can
be captured by the VGSB. We added 
an energy source term in the VGSB equations which relaxes the disk
temperature to its initial value on a timescale
$t_c=\beta\Omega_k^{-1}$. This form of thermal relaxation was also 
employed in the global numerical simulations of \cite{nelson13}, and allows
the thermodynamic response of the disk to be varied in a controlled 
manner. 

We mostly focused on vertically isothermal disks ($\Gamma=1$) because
it permits a largely analytical discussion. Our analysis is an
extension of the linear models considered in \cite{lubow93} by
including vertical shear, but in the limit of low-frequency
perturbations in  nearly-Keplerian disks. These approximations were
made for simplicity, but also motivated by previous work
\citep{nelson13}.   

For isothermal perturbations ($\gamma=1$ or $\beta=0$) we explicitly
demonstrated instability 
associated with vertical shear, and that the VSI growth rate is
limited by the maximum vertical shear within the vertical domain of
interest. In the opposite limit of adiabatic perturbations ($\gamma>1$
and $\beta\to\infty$), we find VSI growth rates are dramatically
reduced on account of a stabilizing 
vertical entropy gradient. The VSI is not completely
surpressed because vertical shear dominates over buoyancy 
sufficiently close the disk midplane. However, perturbations grow
slowly and is constrained to regions immediately about the disk
midplane. Thus, the adiabatic VSI is likely unimportant in practice.  

For non-isothermal ($\gamma\neq1$) perturbations with intermediate
thermal relaxation timescales, the general dispersion relation
(Eq. \ref{relax_disp}) is complicated and requires a numerical
study. However, for the fundamental VSI, characterized by  $\delta
v_z\sim\mathrm{constant}$ near $z=0$, some simplification is
possible. In particular, we quantified the thermal relaxation
timescale $\beta_\mathrm{crit}=\epsilon|q|/(\gamma-1)$, below
which small-scale perturbations ($\khat\gg 1$) may be regarded as isothermal
and the standard VSI operates. We believe this explicit expression for
$\beta_\mathrm{crit}$ is the most useful result from our
investigation.


 







%summary of results 






%application to MMSN 

\subsection{Caveats and outlooks} 