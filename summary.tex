\section{Caveats and Neglected Effects} 
%We highlight below some outstanding issues that should be addressed in
%future works. 

\subsection{Viscosity}\label{caveats_visc}
% Depending on where the VSI operates in a
% disk, additional effects may become important. For example, in the
% outer parts of a PPD, where we estimate the VSI 
% operates, disk self-gravity (SG) may play a role. Vertical
% SG may favor the VSI by increasing $|\p_z\Omega|$ through
% enhancing the density stratification 
% (Eq. \ref{vertical_shear}). %sg in pert? no because kx>>1. but maybe for kx~1 

Our neglect of viscosity is valid in a laminar disk, because molecular viscosity is so small.
However turbulence could act as an effective viscosity which preferentially damps 
small scale modes.   Since the goal of VSI is to drive turbulent transport,
the most relevant modes for sustaining the VSI should be able to overcome
turbulent damping.

To roughly estimate the wavelengths which are prone to damping, we
consider 
%An important effect of viscosity is setting a minimum wavelength (or 
%maximum wavenumber) for which the VSI is relevant. Disturbances with
%large $\khat\equiv k_xH$ may be subject to viscous decay. Adopting 
the standard
precription for the kinematic viscosity $\nu = \alpha c_s H$
\citep{shakura73}, in terms of the dimsionless $\alpha$ 
parameter. The viscous timescale for a perturbation lengthscale  
$l\sim 1/k_x = H/\khat$ is $t_\mathrm{visc} = 1/\alpha\khat^2\Omega $. 
For significant growth, $t_\mathrm{visc}$ should be longer than the
characterstic VSI growth timescale $1/(h|q|\Omega)$, i.e.\ 
$\khat^2\lesssim |q|h/\alpha$. With $h \sim 0.05$, $q = -1$  and
 $\alpha$ from simulations \citepalias{nelson13},
we estimate that growth requires $\khat\lesssim 20$ for
$\alpha\sim 10^{-4}$ or $\khat\lesssim 70$ for $\alpha\sim
10^{-5}$. 

This argument justifies our focus on moderate
wavenumbers, $\khat=O(10)$. Future work should consider
the viscous effects in more detail, to better understand
how the VSI operates in nature and in simulations.

%suggestion: Keep this idea to yourself, since you don't have a result yet.
% In particular, viscous
%heating may influence the thermodynamic dependence of the VSI. 


\subsection{Convective overstability}
The VSI bears some similarity to the `convective overstability'
 (CONO, \citealp{klahr14, lyra14}).   Both instabilities rely on thermal relaxation to 
overcome the stability of adiabatic perturbations, i.e.\ the Solberg-Hoiland
criteria (see \S\ref{solberg}).

The key differences are that the CONO neglects vertical buoyancy
and requires an imaginary horizontal buoyancy,
\begin{align}\label{Nr}
N_r^2 \equiv -{1 \over \rho C_P}{\p P \over \p r} {\p S \over \p r} < 0\, .
\end{align}
Since vertical buoyancy is an important stabilitizing influnce for the VSI, it should
be considered in future studies of CONO.

The sign of $N_r^2$ depends on disk parameters.  To demonstrate that the required parameters 
are somewhat extreme, we consider the midplane of vertically isothermal disks with 
$\Sigma \propto r^n$ so that $n = p+q/2 + 3/2$. Using Eqs. \ref{dSdr} and \ref{Nr} with $\gamma = 1.4$,
$N_r^2 < 0$ requires $n> 3(1/2 + q)$.  For standard disk temperature laws this requirement implies
a flat or rising surface density profile, e.g.\ $n = 3/14$ for $q = -3/7$ or $n = 0$ for $q = -1/2$.
Thus CONO is most like to operate at special locations, like the outside edges of 
disk gaps or holes and/or shadowed regions where $q < -1/2$.

Perhaps due to these physical differences, CONO operates in different regimes of parameter
space than the VSI.  CONO grows best for longer wavelengths $\khat \lesssim 1$ and longer 
cooling times $\beta \sim 1$. 

Future work should aim to understand these related instabilities in the same framework, thereby 
explaining the key differences.




%We have focused on parameter regimes relevant to the VSI,
%namely perturbations with small radial lengthscales 
%($k_xH\gtrsim 1$) in disks with short thermal timescales
%($t_c\Omega\ll 1$). This excludes the `convective overstability'
%recently discussed by \cite{klahr14} and \cite{lyra14} in vertically
%unstratified disk models.   
%
%The convective overstability corresponds to slowly growing epicyclic motions.
%It is most effective for $t_c\Omega \sim 1$ and disturbances with
%$|k_x|\ll |k_z|$, where $k_z$ is a vertical wavenumber. These
%conditions are mutually exclusive with the VSI.  
%The maximum growth rate for convective overstability is given by 
%\begin{align}
%  \sigma_\mathrm{conv,max} = -\frac{N^2_r}{4\Omega}\label{max_conv_gen}
%\end{align}
%\citep{lyra14}, where $N_r^2 = -(c_p\rho)^{-1}\p_rP\p_rS$ is the
%square of the radial buoyancy frequency
%(cf. Eq. \ref{nzsq_def}). Thus, growth requires $N_r^2<0$.  
%%Evaluating Eq. \ref{max_conv_gen} at the midplane of our disk models,
%%we find  
%%\begin{align}
%%  \sigma_\mathrm{conv,max} =
%%  -\frac{ h^2\OmK}{4\gamma\Gamma}\left(p+q\right)
%%  \left[\left(\gamma-1\right)p-q\right]. \label{max_conv}
%%\end{align}
%In fact, for the MMSN considered in this work, convective
%overstability does not apply because 
%$N_r^2>0$.   
%
%For more general disk structures convective overstability is
%possible, in principle. The models adopted for our numerical study in
%\S\ref{numerical} do satisfy $N_r^2<0$, but convective overstability
%require perturbations with long radial lengthscales,
%$k_xH\ll 1$, for which radially global disk models would be more 
%self-consistent. 

\subsection{Radiative transfer} 
Our relatively simple treatment of cooling with an idealized dust opacity
could certainly be generalized in future works.  In hotter disks, gas phase opacities
must be considered.   In cold disks there are many choices for a dust opacity which 
varies with grain sizes and compositions. Dust properties could change the viability 
of the VSI for better or worse.  The radiative transfer itself
could be calculated with higher levels of sophistication, as is already being done in
numerical simulations \citep{stoll14}.


\section{Summary and discussion}\label{summary}
In this paper we study the vertical shear instability (VSI) with a focus on 
the radiative cooling requirements.  In turn we address the viability of the VSI 
as an angular momentum transport mechanism in protoplanetary disks (PPDs).

Our linear axisymmetric analysis of the VSI 
considers (uniquely to our knowledge) finite cooling times in a vertically global model.  
Our main analytical finding, which we confirm numerically, is the critical cooling 
timescale above which VSI growth is suppressed, Eqs.\ \ref{prelim_bcrit} and \ref{iso_vsi_cond}.
Short cooling times are needed to weaken the stabilizing effects of vertical buoyancy, thereby allowing 
the free energy in vertical shear to drive instability.

Our main focus is irradiated, vertically isothermal disks which have strong vertical buoyancy.
The critical cooling time is thus short, shorter than the orbital time by a factor of the disk
aspect ratio.  This finding is consistent with, and helps explain, the results of 3D numerical simulations \citepalias{N13}.

In applying our results to PPDs, we pay particular attention to the transition from 
optically thick radiative diffusion to optically thin Newtonian cooling.  The largest obstacle to VSI occurs in 
high density inner disk, $\lesssim 1$ -- 5 AU, where radiative diffusion times are slow.  Shorter wavelength disturbances
 cool faster, though only if they remain optically thick.  In the inner disk cooling can only be fast enough
 if wavelengths are too short to drive significant transport.  The best hope for the VSI in inner disk is a low surface density.

...

In this paper we have studied the axisymmetric vertical shear 
instability (VSI) in astrophysical disks that arise from the 
height dependence of the equilibrium angular velocity profile,  
$\p_z\Omega\neq 0$. Vertical shear is generally
expected in astrophysical disks due to baroclinicity ($\nabla
P\times\nabla \rho \neq 0$). This may be due
to a radial variation in the disk temperature, or more generally in
the disk entropy. However, astrophysical disks are also typically
sub-adiabatically stratified for stability against vertical
convection ($N_z^2\geq 0$). In this case, the VSI requires a short
thermal or cooling timescale $t_c$ in order to operate.  

Physically, a short thermal timescale is needed to reduce the  
stabilizing effect of buoyancy forces. Because vertical shear is weak
in thin disks, very rapid cooling, corresponding to $t_c\Omega\ll 1$, 
is generally needed for the VSI, otherwise buoyancy is strongly
stabilizing. Our primary goal was to quantify this requirement,
thereby provide a simple way to assess whether or not the VSI is
relevant in realistic disks.  

To do so, we generalized previous vertically global, radially local linear 
calculations of the VSI \citepalias{nelson13,mcnally14,barker15}, which
assumed isothermal perturbations, to include an energy equation with 
a source term in that relaxes the disk 
temperature to its initial value on a timescale
$t_c$. 
% This `thermal relaxation' model allows
% the thermodynamic response of the disk to be controlled by varying
% $t_c$, and has been employed in previous numerical simulations of the
% VSI \citepalias{nelson13}.  
%In accordance with the above studies, which show that the VSI occurs for 
%perturbations with small radial lengthscales, we applied the 
%radially local approximation in the linearized fluid equations. 
We mostly focused on vertically isothermal disks as this is the
relevant case for protoplanetary disks (PPDs) passively 
irradiated by its central star \citep{chiang97}. It also permits a
simplified analytical discussion by extending the 
linear models considered in \cite{lubow93} by adding vertical shear
and (rapid) cooling. We obtained explicit solutions 
describing the VSI with thermal relaxation in the thin disk limit.  

For a vertically isothermal, sub-adiabatically stratified disk we
determined the thermal timescale   
\begin{align*}
  t_\mathrm{crit} = \frac{ h |q|}{\gamma-1}\OmK^{-1}, 
\end{align*}
below which the VSI operates effectively. 
%While $t_\mathrm{crit}$ was
%derived for the lowest order VSI mode --- the `fundamental  
%corrugation mode' --- we find from numerical calculations that
%$t_\mathrm{crit}$ can be used to characterize the general
%stabilization of the VSI by a finite thermal timescale. %That is, all
%VSI growth rates are reduced significantly as $\beta$ is increased
%from 
%zero to $\beta_\mathrm{crit}$.    
We have thus provided a quantitative measure for the thermal
timescale requirement for the VSI in astrophysical 
disks. The thermal timescale must be shorter than the disk
dynamical timescale by $O( h)$, which is 
quite short for thin disks ($ h\ll 1$), because vertical shear is weak.    

We applied our linear models to assess the applicability of the VSI in
PPDs. We considered the Minimum Mass Solar Nebula 
(MMSN) described in \cite{chiang10} and   
modeled the thermal timescale $t_c$ based on realistic
disk properties and dust opacity. 
%This results in $t_c$ being a
% function of position and the perturbation lengthscale of interest. 
We find the MMSN is unstable to the VSI 
from $\sim 5$AU to $\sim50$AU with characteristic growth times of
$\sim 30$ orbits. It is possible to have the VSI at smaller
radii, but only on radial lengthscales signficantly shorter
than the disk scaleheight. %However, such small-scale perturbations may 
%be subject to viscous decay in a real disk.   

Although $t_c$ is not vertically constant in the 
MMSN as assumed in our analysis, we find that evaluating the condition
$t_c< t_\mathrm{crit}$ at the disk midplane gives an indication of the
radii at which the VSI operates in the MMSN. This simple criterion is
thus a practical first step in estimating the relevance of the VSI in
realistic astrophysical disks without resorting to linear
calculations or non-linear numerical simulations.   

 



