\section{Caveats and Neglected Effects}\label{caveats} 
%We highlight below some outstanding issues that should be addressed in
%future works. 

\subsection{Viscosity}\label{caveats_visc}
% Depending on where the VSI operates in a
% disk, additional effects may become important. For example, in the
% outer parts of a PPD, where we estimate the VSI 
% operates, disk self-gravity (SG) may play a role. Vertical
% SG may favor the VSI by increasing $|\p_z\Omega|$ through
% enhancing the density stratification 
% (Eq. \ref{vertical_shear}). %sg in pert? no because kx>>1. but maybe for kx~1 

Our neglect of viscosity is valid in a laminar disk, because molecular viscosity is so small.
However, turbulence could act as an effective viscosity which preferentially damps 
small scale modes.   Since the goal of VSI is to drive turbulent transport,
the most relevant modes for sustaining the VSI should be able to overcome
turbulent damping.

To roughly estimate the wavelengths which are prone to damping, we
consider 
%An important effect of viscosity is setting a minimum wavelength (or 
%maximum wavenumber) for which the VSI is relevant. Disturbances with
%large $\khat\equiv k_xH$ may be subject to viscous decay. Adopting 
the standard
prescription for the kinematic viscosity $\nu = \alpha c_s H$
\citep{shakura73}, in terms of the dimensionless $\alpha$ 
parameter. The viscous timescale for a perturbation lengthscale  
$l\sim 1/k_x = H/\khat$ is $t_\mathrm{visc} = 1/\alpha\khat^2\Omega $. 
For significant growth, $t_\mathrm{visc}$ should be longer than the
characteristic VSI growth timescale $1/(h|q|\Omega)$, i.e.\ 
$\khat^2\lesssim |q|h/\alpha$. With $h \sim 0.05$, $q = -1$  and
 $\alpha$ from simulations \citepalias{nelson13},
we estimate that growth requires $\khat\lesssim 20$ for
$\alpha\sim 10^{-4}$ or $\khat\lesssim 70$ for $\alpha\sim
10^{-5}$. 

This argument justifies our focus on moderate
wavenumbers, $\khat=O(10)$. Future work should consider
the viscous effects in more detail, to better understand
how the VSI operates in nature and in simulations.

%suggestion: Keep this idea to yourself, since you don't have a result yet.
% In particular, viscous
%heating may influence the thermodynamic dependence of the VSI. 


\subsection{Convective overstability}
The VSI bears some similarity to the `convective overstability'
 (CONO, \citealp{klahr14, lyra14}).   Both instabilities rely on thermal relaxation to 
overcome the stability of disks to adiabatic perturbations, i.e.\ the Solberg-Hoiland
criteria (see \S\ref{solberg}).

The key differences are that the CONO neglects vertical buoyancy
and requires an imaginary horizontal buoyancy,
\begin{align}\label{Nr}
N_r^2 \equiv -{1 \over \rho C_P}{\p P \over \p r} {\p S \over \p r} < 0\, .
\end{align}
Since vertical buoyancy is an important stabilizing influence for the VSI, it should
be considered in future studies of CONO.

The sign of $N_r^2$ depends on disk parameters.  To demonstrate that the parameters needed for $N_r^2 <0$
are somewhat extreme, or at least non-standard, we consider the midplane of vertically isothermal disks with 
$\Sigma \propto r^n$ so that $n = p+q/2 + 3/2$. Using Eqs. \ref{dSdr} and \ref{Nr} with $\gamma = 1.4$,
$N_r^2 < 0$ requires $n> 3(1/2 + q)$.  For standard disk temperature laws, this requirement implies
a flat or rising surface density profile, e.g.\ $n = 3/14$ for $q = -3/7$ or $n = 0$ for $q = -1/2$.
Since standard $\Sigma$ profiles decline with radius, CONO is most like to operate at 
special locations, like the outside edges of disk gaps or holes and/or shadowed regions where $q < -1/2$.

Perhaps due to these physical differences, CONO operates in different regimes of parameter
space than the VSI.  The CONO grows best for longer wavelengths $\khat \lesssim 1$ and longer 
cooling times $\beta \sim 1$. 

Future work should aim to understand these related instabilities in the same framework, thereby 
explaining the key differences.




%We have focused on parameter regimes relevant to the VSI,
%namely perturbations with small radial lengthscales 
%($k_xH\gtrsim 1$) in disks with short thermal timescales
%($t_c\Omega\ll 1$). This excludes the `convective overstability'
%recently discussed by \cite{klahr14} and \cite{lyra14} in vertically
%unstratified disk models.   
%
%The convective overstability corresponds to slowly growing epicyclic motions.
%It is most effective for $t_c\Omega \sim 1$ and disturbances with
%$|k_x|\ll |k_z|$, where $k_z$ is a vertical wavenumber. These
%conditions are mutually exclusive with the VSI.  
%The maximum growth rate for convective overstability is given by 
%\begin{align}
%  \sigma_\mathrm{conv,max} = -\frac{N^2_r}{4\Omega}\label{max_conv_gen}
%\end{align}
%\citep{lyra14}, where $N_r^2 = -(c_p\rho)^{-1}\p_rP\p_rS$ is the
%square of the radial buoyancy frequency
%(cf. Eq. \ref{nzsq_def}). Thus, growth requires $N_r^2<0$.  
%%Evaluating Eq. \ref{max_conv_gen} at the midplane of our disk models,
%%we find  
%%\begin{align}
%%  \sigma_\mathrm{conv,max} =
%%  -\frac{ h^2\OmK}{4\gamma\Gamma}\left(p+q\right)
%%  \left[\left(\gamma-1\right)p-q\right]. \label{max_conv}
%%\end{align}
%In fact, for the MMSN considered in this work, convective
%overstability does not apply because 
%$N_r^2>0$.   
%
%For more general disk structures convective overstability is
%possible, in principle. The models adopted for our numerical study in
%\S\ref{numerical} do satisfy $N_r^2<0$, but convective overstability
%require perturbations with long radial lengthscales,
%$k_xH\ll 1$, for which radially global disk models would be more 
%self-consistent. 

\subsection{Radiative transfer} 
Our relatively simple treatment of cooling with an idealized dust opacity
could certainly be generalized in future works.  In hotter disks, gas phase opacities
must be considered.   In cold disks, there are many choices for the dust opacity, which 
varies with grain sizes and compositions. Changing dust properties would alter 
the viability of the VSI for better or worse.  The radiative transfer itself
could be calculated with higher levels of sophistication, as is already being done in
numerical simulations of the VSI \citep{stoll14}.


\section{Summary and discussion}\label{summary}
In this paper we study the vertical shear instability (VSI) with a focus the 
role of radiative cooling.  In turn we assess the viability of the VSI 
as an angular momentum transport mechanism in protoplanetary disks (PPDs).

Our linear, axisymmetric analysis of the VSI 
considers (uniquely to our knowledge) finite cooling times in a vertically global model.  
In order for vertical shear to drive the VSI, short cooling times are needed to weaken 
the stabilizing effects of vertical buoyancy. Our main analytical finding, which we confirm numerically, 
is the critical cooling  timescale above which VSI growth is suppressed, Eqs.\ \ref{prelim_bcrit} and \ref{iso_vsi_cond}.


Our main focus is irradiated, vertically isothermal disks which have strong vertical buoyancy.
The critical cooling time is thus short, shorter than the orbital time by a factor of the disk
aspect ratio.  This finding is consistent with, and helps explain, the results of recent numerical simulations \citepalias{nelson13}.
We briefly consider  vertically non-isothermal disks in \S\ref{nonvertiso}.

In applying our results to PPDs, we pay particular attention to the transition from 
optically thick radiative diffusion to optically thin Newtonian cooling.  The largest obstacle to VSI occurs in 
high density inner disk, $\lesssim 1$ -- 5AU, where radiative diffusion times are slow.  Shorter wavelength disturbances
 cool faster, so long as they remain optically thick.  In the inner disk, however, our VSI cooling criterion requires 
 wavelengths that are too short to drive significant transport.  The best hope for the VSI in inner disk is a low surface density, which speeds radiative diffusion by lengthening the photon mean free path.  This option naturally begs 
 the question of what accretion mechanism lowered the surface density in the first place. 
 
 In the outer disk, the VSI tends to cool in the optically thin limit.  The main issue is whether the opacity 
 is high enough for optically thin cooling to be sufficiently fast.  Our standard opacity assumes a Solar abundances of small dust.
 In this case, the VSI can operate from $\sim 5$ -- 100AU.  
A factor of 10 reduction in the opacity, for instance by locking
 small dust into planetesimals and planetary cores \citep{youdin13}, makes cooling times too slow for VSI growth.  
 In this case cooling is too slow not just in the outer disk, but into  $< 1$ AU.
 
 An enhanced opacity makes optically thin cooling faster and radiative diffusion slower.  This shift favors VSI growth 
 in the outer disk at the expense of the inner disk, pushing the inner limit of VSI growth further out.
 The standard choice --- corresponding to Solar abundances in a standard MMSN disk --- allows the VSI 
 to grow over the widest range of relevant disk radii.
 
 Our detailed study of the spectrum of VSI modes confirms that some artificial `surface modes' are 
 triggered by imposed vertical boundaries \citepalias{nelson13, barker15}.  Domain size
 convergence studies are thus essential.  Fortunately, our results show that longer cooling times 
  stifle the growth of surface modes.  Thus, at least in some cases, more realistic radiative transfer also produces more reliable dynamics.
 
 The VSI deserves further study as a viable mechanism to drive at least low levels of accretion in cold disks.
 



