\section{Summary and discussion}\label{summary}
In this paper we have studied the  axisymmetric vertical shear
instability (VSI) in astrophysical disks that arise from the
height dependence of the equilibrium angular velocity profile
$\Omega=\Omega(r,z)$ in a baroclinic disk. We examined how
the linear instability depends on disk thermodynamics and applied our
results to assess the relevance of the VSI in realistic protoplanetary
disks.  

We generalized previous vertically global linear 
calculations of the VSI \citep{nelson13,mcnally14}, which assumed isothermal
perturbations, to include an energy equation. Our effort may also be
regarded as an extension of \cite{urpin03}, which did include an
energy equation, but adopted the Boussinesq and vertically local
approximations. We considered vertically global and fully
compressible perturbations. 

We included a source term in the energy equation which relaxes the disk
temperature to its initial value on a timescale
$t_c=\beta\Omega_k^{-1}$. This approach allows the thermodynamic
response of the disk to be controlled by varying the parameter
$\beta$, and has been employed in previous numerical simulations of
the VSI \citep{nelson13}. However, it is possible to 
relate  $\beta$ to physical thermal timescales expected in realistic disks.  

In accordance with the above studies, which show that the VSI occurs for 
perturbations with small radial lengthscales, we applied the 
radially local approximation in the linearized fluid equations. For
analytic discussion only, we further simplified the linear problem by
ignoring the background radial disk structure. This is equivalent to adopting
the `vertically global shearing box' (VGSB) formalism developed by
\cite{mcnally14}. 
We find the VGSB framework is appropriate for analyzing the VSI because 
rapid thermal relaxation ($\beta\ll1$) is 
generally required for the instability.   However, the VGSB framework is not
appropriate in the adiabatic limit ($\beta \gg 1$). 
   

We mostly focused on vertically isothermal disks ($\Gamma=1$) as 
this is the relevant case for protoplanetary disks that are passively irradiated by its central star 
\citep{chiang97}. It also permits a largely analytical discussion.
Our analysis is an extension of the linear models considered in
\cite{lubow93} by  adding vertical shear and (rapid) thermal
relaxation. We obtained explicit solutions describing the VSI with 
thermal relaxation in the thin disk limit. 

The fundamental VSI mode is characterized by a symmetric vertical
velocity perturbation with zero nodes and $\delta
v_z\sim\mathrm{constant}$ near the mid-plane, and by an anti-symmetric
density perturbation with one node at $z=0$.  These have been referred
to as `corrugation modes' by \cite{nelson13}, and were found to
dominate their numerical simulations, as well as that of
\cite{stoll14}. Furthermore, when computing unstable modes for a
protoplanetary disk model with thermal relaxation based on realistic
disk properties, we find the fundamental mode to be the dominant
mode. The fundamental mode is
thus likely the most relevant VSI mode in protoplanetary disks.  

We determined the critical thermal relaxation timescale 
$\beta_\mathrm{crit}=\epsilon|q|/(\gamma-1)$, below which small-scale
perturbations may be regarded as isothermal and the VSI 
operates. We have therefore provided a quantitative measure for the `fast'
thermal relaxation requirement for the VSI in astrophysical disks. 

We also solved the linear stability problem numerically using a
pseudo-spectral approach. For numerical calculations we relax all the
approximations made in our analysis, and accounted for
the background radial disk structure.  Our
numerical results are mostly consistent with our analytic 
expectations, including the dependence of $\beta_\mathrm{crit}$ on
disk parameters.  We find the fundamental mode is robust against
boundary conditions, and higher order modes are more rapidly
stabilized with increasing $\beta$. 

We briefly studied vertically non-isothermal disks. We confirm the finding by \cite{nelson13} that in the absence
of vertical entropy gradients, the VSI does not require rapid thermal
relaxation.  When a stable stratification is introduced, the
rapid reduction in VSI growth rates by a finite thermal relaxation
timescale is qualitatively similar to that observed for vertically
isothermal disks. Based on numerical results, we conjecture that the
critical thermal relaxation timescale is given by 
$\beta_\mathrm{crit}=\epsilon|q|/(\gamma-\Gamma)$ in vertically 
non-isothermal disks.  

%application to MMSN 

We applied our linear models to assess the applicability of the VSI in
protoplanetary disks. We considered the Minimum Mass Solar Nebula 
(MMSN) described in \cite{chiang10} and   
modeled the thermal relaxation timescale $\beta$ based on realistic
disk properties and dust opacity. This results in a $\beta$ function that
depends on both the perturbation lengthscale as well as the height 
from the midplane. We find the MMSN is unstable to the VSI 
from $\sim 5$AU to $\sim50$AU with characteristic growth times of
$\sim 30$ orbits. Although $\beta$ is not vertically constant in the
MMSN as assumed in our analysis, we find that evaluating the condition
$\beta <\beta_\mathrm{crit}$ at the disk midplane gives a rough measure of the
radii at which the VSI operates in the MMSN. This simple criterion is
thus a useful first step to estimate the relevance of the VSI in
realistic astrophysical disks without resorting to non-linear
numerical simulations.  

\subsection{Caveats and outlooks} 
We highlight below some extensions to the present  
linear models in future works.  

\subsubsection{Convective overstability}
We have focused on parameter regimes relevant to the VSI,
namely $\khat\gtrsim 1$ and $\beta\ll 1$. This excludes the
`convective overstability' recently discussed by \cite{klahr14} and
\cite{lyra14} in vertically unstratified disk models. 

The convective overstability corresponds to slowly growing epicyclic motions.
It is most effective for $\beta\sim 1$ and disturbances with $|k_x|\ll |k_z|$, where
$k_z$ is a vertical wavenumber. These conditions are mutually exclusive with the VSI. 
The maximum growth rate for convective overstability is given by 
\begin{align}
  \nu_\mathrm{conv,max} = -\frac{N^2_r}{4\Omega}\label{max_conv_gen}
\end{align}
\citep{lyra14}, where $N_r^2 = -(c_p\rho)^{-1}\p_rP\p_rS$ is the
square of the radial buoyancy frequency
(cf. Eq. \ref{nzsq_def}). Thus, growth requires $N_r^2<0$.  

Evaluating Eq. \ref{max_conv_gen} at the midplane of our disk models,
we find  
\begin{align}
  \nu_\mathrm{conv,max} =
  -\frac{\epsilon^2\Omega_k}{4\gamma\Gamma}\left(p+q\right)
  \left[\left(\gamma-1\right)p-q\right]. \label{max_conv}
\end{align}
In fact, for the MMSN considered in this work, convective
overstability does not apply because 
$N_r^2>0$.   

For more general disk structures convective overstability is
possible in principle. The models adopted for our numerical study in
\S\ref{numerical} do satisfy $N_r^2<0$, but convective overstability
require perturbations with long radial lengthscales,
$\khat\ll 1$, for which radially global disk models would be more 
self-consistent. 

\subsubsection{Improved thermal evolution} 
We assumed non-adiabatic effects
can be represented by a thermal relaxation timescale $t_c$. For 
application to protoplanetary disks we connected $t_c$ to the disk
properties. One can improve the realism further by generalizing the 
opacity model adopted here to be density (and hence height) dependent. 
%here to be density-dependent,  i.e. $\kappa_d =
%% \kappa_d(\rho, T)$. 
However, going beyond thermal relaxation requires a
proper treatment of radiative diffusion. The linear problem then
involves a Laplacian operator and additional boundary
conditions must be supplemented. We will consider these improvements in a
follow-up study. %additional heat source 
 
\subsubsection{Additional physics} 
Depending on where the VSI operates in a
disk, additional effects may become important. For example, in the
outer parts of a protoplanetary disk, where we estimate the VSI 
operates, disk self-gravity (SG) may play a role. Vertical
self-gravity may favor the VSI by increasing $|\p_z\Omega|$ through
enhancing the density stratification 
(Eq. \ref{vertical_shear}). %sg in pert? no because kx>>1. but maybe for kx~1 

We have neglected viscosity. \cite{nelson13} find that an 
alpha viscosity $\lesssim O(10^{-4})$ is required for the VSI. 
It is of interest to quantify the effect of viscosity 
on the linear VSI. This would help identify, more accurately, the
appropriate regions in a protoplanetary disk that are sufficiently
laminar to develop the VSI.  This problem is, however, complicated by
the fact that proper inclusion of viscosity introduces a meridional
flow in the equilibrium state, as well as viscous heating
in the energy equation. 


