\section{Full governing equation for vertically isothermal disks}\label{adia_improve}
% \section{Improving the nearly-Keplerian approximation for adiabatic
%   disks}\label{adia_improve}
In \S\ref{approx_gov} we made the replacement $D\to\Omega_k^2$ before
eliminating variables to obtain a single equation for $\delta v_z$,
Eq. \ref{vertiso_gov}.  This procedure ignores the vertical dependence of
$D=\kappa^2(z) - \sigma^2$. We show here that this has no significant
consequence for thin disks. 

In the global disk with $\Gamma=1$ we have
\begin{align}\label{dkappa2}
  \frac{\p\kappa^2}{\p z} = 4 \frac{\p\Omega^2}{\p z} + r\frac{\p}{\p
    r}\frac{\p\Omega^2}{\p z} = -
  \frac{\p\ln\rho}{\p z}\frac{qc_s^2}{r^2} \left(2 + q +
    \frac{z^2/r^2-2}{z^2/r^2+1}\right). 
\end{align}
For the local problem we may then write
\begin{align}
  \frac{dD}{dz}  = - \frac{d\ln\rho}{dz}\frac{qc_s^2}{r^2}F(z;q),
\end{align}
where the function $F$ corresponds to the bracket in
Eq. \ref{dkappa2}. This function varies from $F=q$ at $z=0$ to $F\to
3+q$ as $|z|\to\infty$, i.e. its magnitude is of order unity. 

 Eliminating $W$ and $Q$ from Eq. \ref{ode_w}---\ref{ode_Q} and
 Eq. \ref{lin_vz} for vertically isothermal disks, retaining the
 vertical dependence of $D$ and including thermal relaxation, we
 obtain 
\begin{align}
  0 =& \frac{d^2\delta v_z}{dz^2} + \left[1 + \frac{\ii k_x c_s^2
      q}{Dr} - \frac{k_x^2c_s^2}{\left(k_x^2c_s^2 + \chi
        D\right)}\frac{qc_s^2F}{Dr^2}\right]\frac{d\ln\rho}{dz}\frac{d\delta
    v_z}{dz} \notag\\
  &+ \left\{\sigma^2\left(\frac{k_x^2}{D} +
      \frac{\chi}{c_s^2}\right) + \left(\chi + \frac{\ii k_x c_s^2
        q}{Dr}\right)\frac{d^2\ln\rho}{dz^2} -
    \frac{c_s^2}{D}\left(\frac{d\ln\rho}{dz}\right)^2\left(k_x^2 -
      \frac{\ii k_x q}{r}\right)
   \left[\left(1-\chi\right) +
     \frac{\chi}{\left(k_x^2c_s^2 + \chi D\right)}\frac{qc_s^2 F}{r^2}\right] 
   \right\}\delta v_z,
\end{align}
where we recall $\chi = \left(1-\ii\sigma t_c\right)/\left(1-\ii\sigma
t_c \gamma\right)$. Making the low-frequency approximation with the
replacement $D\to \Omega_k^2$ gives, in terms of dimensionless
variables,
\begin{align}
   &\delta v_z ^{\prime\prime} + \left[1 + \ii\epsilon q\hat{k} -
    \frac{ \hat{k}^2}{
      \left(\hat{k}^2+\chi\right)}q \epsilon^2F\right]\ln\rho^{\prime}\delta v_z^\prime +
  \left\{\left(\chi + \ii \epsilon q
      \hat{k}\right)\ln\rho^{\prime\prime} - \ln\rho^{\prime
      2}\left(\khat^2 -
      \ii\epsilon
      q\hat{k}\right)\left[1 - \chi +
      \frac{\chi}{\left(\hat{k}^2+\chi\right)}q\epsilon^2F\right]\right\}\delta v_z \notag\\&=
  -\hat{\sigma}^2\left(\hat{k}^2+\chi\right)\delta v_z\label{adia_diso3}.
\end{align}   
Eq. \ref{adia_diso3} differs from Eq. \ref{vertiso_gov_nondim} by terms
proportional to $\epsilon^2$. For a thin disk, $\epsilon\ll1$, so
neglecting these terms has no qualitative effect on the discussion in
\S\ref{analytic_adia} and \S\ref{analytic_relax}.  This issue does not
arise for isothermal perturbations discussed in \S\ref{iso_discuss}
since in that case we work with a governing equation for $W$ instead. 

\section{Alternative inference of instability in the presence of weak
  vertical shear}\label{pert_theory}
We can also investigate the destabilizing effect of vertical shear
without explicitly solving the full governing equation,
Eq. \ref{iso_ode3}. We first
consider a system with $q\equiv0$, for which the eigenfunctions are
Hermite polynomials and the real eigenfrequencies are known. We then
perturb this system by introducing a weak vertical shear ($|q|\ll1$)
and linearize the governing equation. This procedure is
\begin{align}   
  q \to 0 + \delta q,\quad
  W \to \He_n + \delta W,\quad
  \hat{\sigma} \to \hat{\sigma} + \delta\hat{\sigma}. 
\end{align}
%where for this exercise we restore $\sigma$ as the eigenfrequency. 

Linearizing the integral relation Eq. \ref{integral_relation1} in the
thin-disk limit, and taking the
imaginary part, we have
\begin{align}
  2\hat{\sigma}\imag(\delta\hat{\sigma})
  \left(1+\hat{k}^2\right) \int_{-\infty}^{\infty} w(\zhat)
  \He_n^2(\hat{z}) d\hat{z}
  = \delta q \epsilon \hat{k} 
  \int_{-\infty}^{\infty}
  w(\zhat)\hat{z}\He_n(\hat{z})\He_n^\prime(\zhat) d\hat{z}
\end{align}
Recognizing $\hat{z} = \He_1(\hat{z})$ and using $\He_n^\prime = n
\He_{n-1}$ we have
 \begin{align}
   2\hat{\sigma}\imag(\delta\hat{\sigma})
   \left(1+\hat{k}^2\right) \int_{-\infty}^{\infty} w(\zhat)
   \He_n^2(\hat{z}) d\hat{z}
   =\delta q \epsilon \hat{k} 
   \int_{-\infty}^{\infty}
   w(\zhat)\He_1(\hat{z})\He_n(\hat{z})n\He_{n-1}(\hat{z})d\hat{z}. 
 \end{align}

Finally, we evaluate the integrals using the results
\begin{align}
  \int_{-\infty}^{\infty}
  w(\xi) \He_k(\xi)\He_l(\xi) d\xi = \sqrt{2\pi}k!\delta_{kl} \quad
% \end{align}
\text{and} \quad
% \begin{align}
  \int_{-\infty}^{\infty}
  w(\xi) \He_m(\xi)\He_k(\xi)\He_l(\xi) d\xi =
  \frac{\sqrt{2\pi}m!k!l!}{(j-m)!(j-k)!(j-l)!}, 
\end{align}
where $j = (m+k+l)/2$. We then obtain
 \begin{align}
   2\hat{\sigma}\imag(\delta\hat{\sigma})
  \left(1+\hat{k}^2\right) = \delta q \epsilon \hat{k} n.
 \end{align}
Inserting the original eigenfrequency $\hat{\sigma} = \pm
\sqrt{n}(1+\hat{k}^2)^{-1/2}$ gives
\begin{align}
  \imag(\delta\hat{\sigma})= \pm \frac{1}{2}\delta q \epsilon
  \sqrt{n} \frac{\hat{k}}{\sqrt{1+\hat{k}^2}}. 
\end{align}
This result agrees with Eq. \ref{simple_growth} in the limit
$|q|\to0$. 
%for $n=1$, since the
%function $\He_1$ solves the governing equation exactly. 
For $\hat{k}\gg 1$ we have
\begin{align}
  \imag(\delta\hat{\sigma})\simeq \pm \frac{1}{2}\delta q \epsilon
  \sqrt{n} \sgn{\hat{k}}. 
\end{align}




% Differentiating Eq. \ref{adia_iso1} properly gives
% \begin{align}
%   \left(D + \gamma k_x^2 c_s^2\right)\frac{d\Delta}{dz}=& D
%   \frac{d^2\delta v_z}{dz^2} + k_x^2c_s^2\left[\frac{\gamma}{\left(D + \gamma k_x^2 c_s^2\right)}\frac{dD}{dz} + \ii
%     \frac{d\ln\rho}{dz}\left(\ii +
%       \frac{q}{k_xr}\right)\right]\frac{d\delta v_z}{dz}\notag\\
%   &+ \ii k_x^2 c_s^2 \left(\ii +
%       \frac{q}{k_xr}\right)\left[\frac{d^2\ln\rho}{dz^2} -
%       \frac{1}{\left(D + \gamma
%           k_x^2c_s^2\right)}\frac{dD}{dz}\frac{d\ln\rho}{dz}\right]\delta
%     v_z. \label{adia_diso1}
% \end{align}
% The terms that were ignored in deriving Eq. \ref{adia_iso3} are those
% proportional to $dD/dz$ in Eq. \ref{adia_diso1}. Now, in the global
% disk with $\Gamma=1$ we have
% \begin{align}\label{dkappa2}
%   \frac{\p\kappa^2}{\p z} = 4 \frac{\p\Omega^2}{\p z} + r\frac{\p}{\p
%     r}\frac{\p\Omega^2}{\p z} = -
%   \frac{\p\ln\rho}{\p z}\frac{qc_s^2}{r^2} \left(2 + q +
%     \frac{z^2/r^2-2}{z^2/r^2+1}\right). 
% \end{align}
% For the local problem we may then write
% \begin{align}
%   \frac{dD}{dz}  = - \frac{d\ln\rho}{dz}\frac{qc_s^2}{r^2}F(z;q),
% \end{align}
% where the function $F$ corresponds to the bracket in
% Eq. \ref{dkappa2}. This function varies from $F=q$ at $z=0$ to $F\to
% 3+q$ as $|z|\to\infty$, i.e. its magnitude is of order unity. 
% % We estimate the
% % importance of these terms by noting that 
% % \begin{align}
% %   \frac{dD}{dz}\equiv \frac{d\kappa^2}{dz} \simeq \frac{d\Omega^2}{dz}
% %   = - \frac{d\ln\rho}{dz}\frac{qc_s^2}{r^2},
% % \end{align}
% % for thin disks. 
% Eq. \ref{adia_diso1} becomes
% \begin{align}
% \left(D + \gamma k_x^2 c_s^2\right)\frac{d\Delta}{dz}=& D
%   \frac{d^2\delta v_z}{dz^2} - k_x^2c_s^2\frac{d\ln\rho}{dz}\left[ 1 - 
%       \frac{\ii q}{k_xr}  +  \frac{\gamma q c_s^2F}{r^2\left(D +
%         \gamma k_x^2 c_s^2\right)}\right]\frac{d\delta
%     v_z}{dz}\notag\\ 
%   &- k_x^2 c_s^2 \left(1 - 
%       \frac{\ii q}{k_xr}\right)\left[\frac{d^2\ln\rho}{dz^2} + 
%       \frac{qc_s^2F}{r^2\left(D + \gamma
%           k_x^2c_s^2\right)}\left(\frac{d\ln\rho}{dz}\right)^2\right]\delta
%     v_z. \label{adia_diso2}
% \end{align}
% Since $D\sim \Omega_k^2$ in the low-frequency limit, we 
% see from Eq. \ref{adia_diso2} that the neglected terms are
% $O(\epsilon^2)$. We can make this explicit by combining
% Eq. \ref{adia_diso1}---\ref{adia_diso2} and Eq. \ref{adia_iso2}, then
% set $D\to\Omega_k^2$. In terms of non-dimensional variables, the result
% is  
% \begin{align}
%    &\delta v_z ^{\prime\prime} + \left(1 + \ii\epsilon q\hat{k} -
%     \frac{\gamma q \epsilon^2\hat{k}^2F}{1+\gamma
%       \hat{k}^2}\right)\ln\rho^{\prime}\delta v_z^\prime +
%   \left[\left(\frac{1}{\gamma} + \ii \epsilon q
%       \hat{k}\right)\ln\rho^{\prime\prime} - \hat{k}^2\left(1 -
%       \frac{\ii\epsilon
%         q}{\hat{k}}\right)\left(\frac{\gamma-1}{\gamma} +
%       \frac{q\epsilon^2F}{1+\gamma\hat{k}^2}\right)\ln\rho^{\prime
%       2}\right]\delta v_z \notag\\&=
%   -\hat{\sigma}^2\left(\frac{1}{\gamma} + \hat{k}^2\right)\delta v_z\label{adia_diso3}.
% \end{align}   
% Eq. \ref{adia_diso3} differs from Eq. \ref{adia_iso3} by terms
% proportional to $\epsilon^2$. For a thin disk, $\epsilon\ll1$, so
% neglecting these terms has no qualitative effect on the discussion in
% \S\ref{analytic_adia}. 

\section{Coefficients for the dispersion relation for perturbations
with thermal relaxation}\label{relax_coeff}
The coefficients of the dispersion relation Eq. \ref{relax_disp} is
given by:
\begin{align}
  &c_0 = M(M+1)\widetilde{A}^2,\\
  &c_1 = \ii\beta\left\{\left(1-\gamma\right)\left[1 +
      \khat^2\left(1+2M\right)^2 - 4 \ii\epsilon q\khat M (M+1)\right] 
    - 2\widetilde{A}^2\gamma M (M+1)\right\},\\
  &c_2 = \left(\khat^2 + 1\right)\widetilde{A} + \beta^2\left\{(1-\gamma)\left[1
      + \gamma \khat^2(1+2M)^2 - 4\ii\epsilon q \khat \gamma M(M+1)
    \right]
    -\gamma^2 \widetilde{A}^2 M(M+1)
  \right\},\\
  &c_3 = \beta\left\{\epsilon q \khat + \gamma \left[\ii + \epsilon q
      k \left(1+2\khat^2\right)\right] - 3\ii - 2\ii
    \khat^2\right\},\\
  &c_4 =
  \beta^2\left(1+\gamma\khat^2\right)\left[\gamma\left(1-\ii\epsilon q
    \khat\right)-2\right] - \left(1+\khat^2\right)^2,\\
&c_5 = 2\ii\beta\left(1+\khat^2\right)\left(1+\gamma\khat^2\right),\\
&c_6 = \beta^2\left(1+\gamma\khat^2\right)^2.
\end{align}

\section{Fiducial model for a protoplanetary disk}\label{mmsn}
For results application in \S\ref{application} we use the disk model
described in \cite{chiang10}. This disk model orbits a Solar-mass star and 
has the surface density distribution
\begin{align}\label{mmsn_sigma}
  \Sigma = 2200
  \hat{\Sigma}\left(\frac{r}{\mathrm{AU}}\right)^{-3/2}\mathrm{g}\,\mathrm{cm}^{-2},  
\end{align}
and the temperature profile
\begin{align}\label{mmsn_temp}
  T = 120\hat{T}\left(\frac{r}{\mathrm{AU}}\right)^{-3/7} \mathrm{K},
\end{align}
which implies $q=-3/7$. In the above expressions, $\hat{\Sigma}$ and
$\hat{T}$ are dimensionless coefficients used to scale the model
relative to the MMSN. Their nomial values are unity. By assuming a
vertically isothermal disk, we deduce the disk aspect-ratio 
\begin{align}\label{mmsn_epsilon}
  \epsilon =
  3.36\times10^{-2}\left(\frac{\hat{T}}{\mu}\right)^{1/2}\left(\frac{r}{\mathrm{AU}}\right)^{2/7}, 
\end{align}
and the mid-plane density distribution 
\begin{align}
%  \rho_0 = 2.7\times10^{-9}
%  \hat{\Sigma}\left(\frac{r}{\mathrm{AU}}\right)^{-39/14}\mathrm{g}\,\mathrm{cm}^{-3},  
\rho_0 = 1.7\times10^{-9}
  \hat{\Sigma}\left(\frac{\hat{T}}{\mu}\right)^{-1/2}\left(\frac{r}{\mathrm{AU}}\right)^{-39/14}\mathrm{g}\,\mathrm{cm}^{-3},
\end{align}
which implies $p=-39/14$. 
%0.033576258
In addition, we
use the opacity model {\bf(reference?)}
 \begin{align}
   \kappa_d &= 2 \hat{\kappa}_d \left(\frac{T}{100\mathrm{K}}\right)^2
   \mathrm{cm}^2\,\mathrm{g}
    =
   2.88\hat{\kappa}_d\hat{T}^2\left(\frac{r}{\mathrm{AU}}\right)^{-6/7}\mathrm{cm}^2\,\mathrm{g}^{-1},   
 \end{align}
where the second equality follows from the model temperature profile
above, and $\hat{\kappa}_d$ has similar meaning as $\hat{\Sigma}$ and
$\hat{T}$. 
