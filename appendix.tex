\section{Alternative inference of instability in the presence of weak
  vertical shear}\label{pert_theory}
We can also investigate the destabilizing effect of vertical shear
without explicitly solving the full governing equation,
Eq. \ref{iso_ode3}. We first
consider a system with $q\equiv0$, for which the eigenfunctions are
Hermite polynomials and the real eigenfrequencies are known. We then
perturb this system by introducing a weak vertical shear ($|q|\ll1$)
and linearize the governing equation. This procedure is
\begin{align}   
  q \to 0 + \delta q,\quad
  W \to \He_n + \delta W,\quad
  \hat{\sigma} \to \hat{\sigma} + \delta\hat{\sigma}. 
\end{align}
%where for this exercise we restore $\sigma$ as the eigenfrequency. 

Linearizing the integral relation Eq. \ref{integral_relation1} in the
thin-disk limit, and taking the
imaginary part, we have
\begin{align}
  2\hat{\sigma}\imag(\delta\hat{\sigma})
  \left(1+\hat{k}^2\right) \int_{-\infty}^{\infty} w(\zhat)
  \He_n^2(\hat{z}) d\hat{z}
  = \delta q \epsilon \hat{k} 
  \int_{-\infty}^{\infty}
  w(\zhat)\hat{z}\He_n(\hat{z})\He_n^\prime(\zhat) d\hat{z}
\end{align}
Recognizing $\hat{z} = \He_1(\hat{z})$ and using $\He_n^\prime = n
\He_{n-1}$ we have
 \begin{align}
   2\hat{\sigma}\imag(\delta\hat{\sigma})
   \left(1+\hat{k}^2\right) \int_{-\infty}^{\infty} w(\zhat)
   \He_n^2(\hat{z}) d\hat{z}
   =\delta q \epsilon \hat{k} 
   \int_{-\infty}^{\infty}
   w(\zhat)\He_1(\hat{z})\He_n(\hat{z})n\He_{n-1}(\hat{z})d\hat{z}. 
 \end{align}

Finally, we evaluate the integrals using the results
\begin{align}
  \int_{-\infty}^{\infty}
  w(\xi) \He_k(\xi)\He_l(\xi) d\xi = \sqrt{2\pi}k!\delta_{kl} \quad
% \end{align}
\text{and} \quad
% \begin{align}
  \int_{-\infty}^{\infty}
  w(\xi) \He_m(\xi)\He_k(\xi)\He_l(\xi) d\xi =
  \frac{\sqrt{2\pi}m!k!l!}{(j-m)!(j-k)!(j-l)!}, 
\end{align}
where $j = (m+k+l)/2$. We then obtain
 \begin{align}
   2\hat{\sigma}\imag(\delta\hat{\sigma})
  \left(1+\hat{k}^2\right) = \delta q \epsilon \hat{k} n.
 \end{align}
Inserting the original eigenfrequency $\hat{\sigma} = \pm
\sqrt{n}(1+\hat{k}^2)^{-1/2}$ gives
\begin{align}
  \imag(\delta\hat{\sigma})= \pm \frac{1}{2}\delta q \epsilon
  \sqrt{n} \frac{\hat{k}}{\sqrt{1+\hat{k}^2}}. 
\end{align}
This result agrees with Eq. \ref{simple_growth} in the limit
$|q|\to0$. 
%for $n=1$, since the
%function $\He_1$ solves the governing equation exactly. 
For $\hat{k}\gg 1$ we have
\begin{align}
  \imag(\delta\hat{\sigma})\simeq \pm \frac{1}{2}\delta q \epsilon
  \sqrt{n} \sgn{\hat{k}}. 
\end{align}

\section{Improving the nearly-Keplerian approximation for adiabatic
  disks}\label{adia_improve}
In \S\ref{analytic_adia} we made the replacement $D\to\Omega_k^2$
before combining Eq. \ref{adia_iso1}---\ref{adia_iso2} to derive
Eq. \ref{adia_iso3}. This procedure ignores the vertical dependence of
$D$. We show here that this has no significant consequence for thin
disks. 

Differentiating Eq. \ref{adia_iso1} properly gives
\begin{align}
  \left(D + \gamma k_x^2 c_s^2\right)\frac{d\Delta}{dz}=& D
  \frac{d^2\delta v_z}{dz^2} + k_x^2c_s^2\left[\frac{\gamma}{\left(D + \gamma k_x^2 c_s^2\right)}\frac{dD}{dz} + \ii
    \frac{d\ln\rho}{dz}\left(\ii +
      \frac{q}{k_xr}\right)\right]\frac{d\delta v_z}{dz}\notag\\
  &+ \ii k_x^2 c_s^2 \left(\ii +
      \frac{q}{k_xr}\right)\left[\frac{d^2\ln\rho}{dz^2} -
      \frac{1}{\left(D + \gamma
          k_x^2c_s^2\right)}\frac{dD}{dz}\frac{d\ln\rho}{dz}\right]\delta
    v_z. \label{adia_diso1}
\end{align}
The terms that were ignored in deriving Eq. \ref{adia_iso3} are those
proportional to $dD/dz$ in Eq. \ref{adia_diso1}. We estimate the
importance of these terms by noting that 
\begin{align}
  \frac{dD}{dz}\equiv \frac{d\kappa^2}{dz} \simeq \frac{d\Omega^2}{dz}
  = - \frac{d\ln\rho}{dz}\frac{qc_s^2}{r^2},
\end{align}
for thin disks. Then Eq. \ref{adia_diso1} becomes
\begin{align}
\left(D + \gamma k_x^2 c_s^2\right)\frac{d\Delta}{dz}=& D
  \frac{d^2\delta v_z}{dz^2} - k_x^2c_s^2\frac{d\ln\rho}{dz}\left[ 1 - 
      \frac{\ii q}{k_xr}  +  \frac{\gamma q c_s^2}{r^2\left(D +
        \gamma k_x^2 c_s^2\right)}\right]\frac{d\delta
    v_z}{dz}\notag\\ 
  &- k_x^2 c_s^2 \left(1 - 
      \frac{\ii q}{k_xr}\right)\left[\frac{d^2\ln\rho}{dz^2} + 
      \frac{qc_s^2}{r^2\left(D + \gamma
          k_x^2c_s^2\right)}\left(\frac{d\ln\rho}{dz}\right)^2\right]\delta
    v_z. \label{adia_diso2}
\end{align}
Since $D\sim \Omega_k^2$ in the low-frequency limit, we 
see from Eq. \ref{adia_diso2} that the neglected terms are
$O(\epsilon^2)$. We can make this explicit by combining
Eq. \ref{adia_diso1}---\ref{adia_diso2} and Eq. \ref{adia_iso2}, then
set $D\to\Omega_k^2$. In terms of non-dimensional variables, the result
is  
\begin{align}
   &\delta v_z ^{\prime\prime} + \left(1 + \ii\epsilon q\hat{k} -
    \frac{\gamma q \epsilon^2\hat{k}^2}{1+\gamma
      \hat{k}^2}\right)\ln\rho^{\prime}\delta v_z^\prime +
  \left[\left(\frac{1}{\gamma} + \ii \epsilon q
      \hat{k}\right)\ln\rho^{\prime\prime} - \hat{k}^2\left(1 -
      \frac{\ii\epsilon
        q}{\hat{k}}\right)\left(\frac{\gamma-1}{\gamma} +
      \frac{q\epsilon^2}{1+\gamma\hat{k}^2}\right)\ln\rho^{\prime
      2}\right]\delta v_z \notag\\&=
  -\hat{\sigma}^2\left(\frac{1}{\gamma} + \hat{k}^2\right)\delta v_z\label{adia_diso3}.
\end{align}   
Eq. \ref{adia_diso3} differs from Eq. \ref{adia_iso3} by terms
proportional to $\epsilon^2$. For a thin disk, $\epsilon\ll1$, so
neglecting these terms has no qualitative effect on the discussion in
\S\ref{analytic_adia}. 