\section{Analytic results with finite cooling times}\label{analytical}  

%\subsection{Isothermal perturbations}\label{iso_pert}
Previous analytic studies of the VSI have largely focused on isothermal
perturbations, with infinitely rapid cooling, as discussed in the introduction.  We further explore
this $\beta = 0$ limit in Appendix \ref{iso_discuss} both to further develop intuition for this 
idealized case and to establish a connection with previous 
works.  However our main interest is the effect of finite cooling times, which
we explore analytically in this section.  

In \S\ref{relax_pert} we show that even an infinitesimal increase in the cooling time,
starting from $\beta = 0$, slows the growth rate of the VSI.  In \S\ref{disp_relax} we 
derive exact solutions to the simplified VSI model developed above (Eq.\ \ref{nearly_iso_explicit}).
In \S\ref{iso_vsi_beta_crit} we reach our main result, the maximum cooling time above which 
VSI growth is suppressed.


%Isothermal perturbations correspond to $\beta = 0$ or $\gamma=1$ so 
%that $\chi = 1$. This case has been investigated by \citetalias{nelson13},
%\citetalias{mcnally14} and \citetalias{barker15}.  
%
%In Appendix \ref{iso_discuss}, we collect some useful results for
%isothermal  perturbations. This includes a necessary condition for
%instability (there must be vertical shear) and an upper bound to the
%instability growth rate (it is limited by the maximum vertical shear
%rate in the disk). Neither of these actually require the thin-disk
%form of the density field described above.  
%We also develop solutions in the thin-disk
%limit, but without assuming low frequency from the onset. 
%The methods used in Appendix \ref{iso_discuss} and results obtained
%therein motivates the solution procedure employed in the following
%sections, and are provided for ease of reference. We refer the reader
%to \citetalias{barker15} for a more detailed treatment of isothermal 
%perturbations. %which also account for more general boundary
%               %conditions than that considered below 

\subsection{Introducing a small but finite
  cooling time}\label{relax_pert}
We are interested in finite thermal relaxation
timescales $\beta > 0$, but it is instructive to
first ask  the more analytically tractable question: how 
do the eigenfrequencies and eigenfunctions change when we change
$\beta$ from zero to a small but finite value? For 
sufficiently small $\beta$ we expect the solution to only differ
slightly from a case with $\beta\equiv 0$. We thus perturb a solution
for $\beta=0$ to see the effect of finite cooling.  

For definiteness, let us consider the simple solution 
\begin{align}
  \beta\equiv 0, \quad \delta v_z = 1,\quad \hat{\upsilon}^2 = \frac{1 +
  \ii h q \hat{k}}{1+\hat{k}^2}, \label{pert_basic} 
\end{align}
which solves Eq. \ref{nearly_iso_explicit} since $\delta
v_z=$constant and $\chi=1$ so that $B=C=0$. 
This is the lowest order VSI mode, the `fundamental corrugation mode',
where the vertical velocity is constant throughout the disk
column. Hereafter we shall simply call it the fundamental
mode. 

The fundamental mode has been observed to dominate numerical simulations
\citep[\citetalias{nelson13}]{stoll14}, and are the ones we find to typically 
dominate in numerical calculations with increasing $\beta$
(\S\ref{therm_relax_eff}) for moderate    
values of $\khat$, as well as in PPDs 
with a realistic estimate for thermal timescales
(\S\ref{application}). We will see later that consideration of low
order modes in fact provides a useful way to characterize the effect
of increasing the thermal timescale on the VSI. 

We linearize Eq. \ref{nearly_iso_explicit} about the above
solution for $\beta\equiv0$ and write 
\begin{align}\label{nearly_iso_pert}
  \beta \to 0 + \delta\beta,\, \delta v_z\to \delta v_z+\delta
  v_{z1},\,\hat{\upsilon} \to \hat{\upsilon} + \delta\hat{\upsilon}, 
\end{align}
which implies 
\begin{align}
  \chi \to 1 + \delta\chi = 1 + \ii \hat{\upsilon}\left(\gamma-1\right)\delta\beta,
\end{align}
with $\delta v_z$ and $\hat{\upsilon}$ given by Eq. \ref{pert_basic}. We
may then seek  
\begin{align}
  \delta v_{z1} = a \zhat^2 + b,
\end{align}
where $a$, $b$ are constants. 

We insert Eq. \ref{nearly_iso_pert} into
Eq. \ref{nearly_iso_explicit}, keeping only first order terms, and
solve for $\delta\hat{\upsilon}$ using the above expressions for $\delta\chi$
and $\hat{\upsilon}$. We find imaginary part of $\delta\hat{\upsilon}$
is 
% yield the requirements, after
% balancing constants and coefficients of $\zhat^2$, 
% \begin{align}
%   0 &= 2a + 2\hat{\upsilon}\left(1+\khat^2\right)\delta\hat{\upsilon} +
%   \left(\hat{\upsilon}^2 -1\right)\delta\chi, \\
%   0&= 2a\left(1+\ii h q \khat\right) - \left(\khat^2 - \ii h
%     q \khat\right)\delta\chi.
% \end{align}
% This gives 
% \begin{align}\label{nearly_iso_dsig}
%   \delta \hat{\upsilon} =
%   -\frac{\ii\left(\gamma-1\right)\hat{k}^2\left(\ii h
%       q - \hat{k}\right)^2}{2\left(1+\ii h q \hat{k}\right)\left(1+\hat{k}^2\right)^2}\delta\beta.
% \end{align}
% The imaginary part of Eq. \ref{nearly_iso_dsig} is
\begin{align}
  \delta\hat{\sigma} =
  -\frac{\left(\gamma-1\right)\hat{k}^2 \left(\hat{k}^2 -
      2h^2q^2\hat{k}^2 - h^2q^2\right)}{2\left(1+h^2 q^2
      \hat{k}^2\right)\left(1+\hat{k}^2\right)^2}\delta\beta.  
\end{align}
Since $h \ll 1$, introducing finite cooling $\delta\beta>0$
implies $\delta\hat{\sigma} < 0$, i.e. stabilization, since $\gamma>1$. 
% Interestingly, for both
% $\hat{k}\to0$ and $\hat{k}\to\infty$ we find $\delta\hat{\nu}\to0$, so
% stabilization by small finite cooling is ineffective at very large or
% very small scales. 

A finite cooling time allow buoyancy
forces to stabilize vertical motions in sub-adiabatically stratified
disks. The $\zhat^2$ dependence in $\delta v_{z1}$ makes  
sense because it becomes significant at large $|\zhat|$, i.e. away from
the midplane where the effect of buoyancy first appears as $\beta$ is
increased from zero (since $N_z^2\propto z^2$ for a thin disk, see
Eq. \ref{nzsq_def}).   

\subsection{Explicit solutions and dispersion relation}\label{disp_relax}
We now solve Eq. \ref{nearly_iso_explicit} explicitly. We first write  
\begin{align}
  \delta v_z(\zhat) =
  g(\zhat)\exp\left(\frac{\lambda\zhat^2}{2}\right), \label{adia_ansatz}
\end{align}
where $\lambda$ is a constant to be chosen for convenience. Inserting
Eq. \ref{adia_ansatz} into Eq. \ref{nearly_iso_explicit} gives
\begin{align}
  0 = g^{\prime\prime} - \hat{z}\left(A - 2\lambda\right)g^\prime + \left(B +
    \lambda\right)g
  +\left(\lambda^2 - \lambda A - C\right)\zhat^2 g.
\end{align}
We choose $\lambda$ to make the coefficient of $\zhat^2g$
vanish, and impose the vertical kinetic energy density
$\rho|\delta v_z|^2\propto |g|^2 \exp{\left(\real\lambda -
    1/2\right)\zhat^2}$ to remain finite as $|\zhat|\to\infty$. 
Then assuming $g(\zhat)$ is a polynomial, we require  
\begin{align}
  \real\lambda < \frac{1}{2}, 
\end{align}
which amounts to choosing 
\begin{align}
  \lambda = \frac{1}{2}\left(A - \sqrt{A^2 + 4C}\right).  
\end{align} 


Eq. \ref{nearly_iso_explicit} becomes 
\begin{align}
  0 = g^{\prime\prime} - \hat{z}\left(A - 2\lambda\right)g^\prime +
  \left(B + \lambda\right)g.
\end{align}
We seek polynomial solutions 
\begin{align}
  g(\zhat) = \sum_{m=0}^M b_m \zhat^m,
\end{align}
which requires
\begin{align}
  B(\hat{\upsilon}) + \lambda(\hat{\upsilon}) =
  M\left[A-2\lambda(\hat{\upsilon})\right].\label{adia_disp0} 
\end{align}
For ease of analysis, we re-arrange Eq. \ref{adia_disp0} and square it
to obtain a polynomial in $\hat{\upsilon}$,  
\begin{align}
  0 = \sum_{l=0}^{6}c_l\hat{\upsilon}^l,\label{relax_disp}
\end{align}
where the coefficients $c_l$ are given in Appendix \ref{relax_coeff}.
The dispersion relation $\hat{\upsilon}=\hat{\upsilon}(\khat)$ is
complicated and generally requires a numerical solution. However,
simple results may be obtained in limiting cases which we consider
below.  

\subsection{The critical cooling time derived}\label{iso_vsi_beta_crit}
Here we estimate the maximum thermal relaxation timescale 
$\beta_c$ that allows growth of the VSI. In Appendix \ref{disp_neut_limit}
we derive the following relation between $\beta_c$ and the wave
frequency $\hat{\omega}_c$, \emph{assuming} marginal stability to the
VSI:  
\begin{subequations}\begin{align}
    \left(\hat{\omega}_c\khat\right)^4  =& \left(\hat{\omega}_c\khat\right)^2 
  + M(M+1)\left(1-h^2q^2\khat^2\right) ,\label{relax_cond_simp1}  \\
  \left(\hat{\omega}_c\khat\right)^2 = & \frac{\beta_c}{h q} (\gamma-1)(1+2M)^2
  \left(\hat{\omega}_c\khat\right) \label{relax_cond_simp2}\\
  &- 2M(M+1) . \notag
\end{align}\end{subequations}
These equations consider $\khat^2\gg 1$ and %, $M\lesssim O(1)$; 
assumes $|\hat{\omega}\khat| $ is $O(1)$ and $\beta_c\ll 1$.   Note that 
Eq. \ref{relax_cond_simp1} reduces to the  
dispersion relation for low-frequency inertial waves in the absence of
vertical shear (see Appendix \ref{stable_novshear} for details). 
%marginal stability requires hqk \leq 1
 
We are most interested in the longest cooling time which allows growth
for any $M$. In Appendix \ref{max_cool}, we find that if
$|hq\khat|\leq 1$, so that marginal stability exists, then $\beta_c$
decreases with increasing $M$.    
%\begin{align}
%  |hq\khat|^2 \leq 1 \Rightarrow \frac{d\beta_c}{dM} < 0 \quad \forall
%  M\geq 0. \label{bcrit_gen}
%\end{align}
%Eq. \ref{bcrit_gen} indicates that higher $M$ modes have lower $\beta_c$,
%i.e.\ require faster cooling for VSI growth. This trend is confirmed
%numerically in \S\ref{therm_relax_eff}.  
The VSI criterion is then set by $\beta_c$ for the
fundamental mode ($M = 0$), for which the exact solution to 
Eq. \ref{relax_cond_simp1} --- \ref{relax_cond_simp2} is
$\hat{\omega}_c\khat = \beta_c(\gamma-1)/hq = -1$. We thus find the
cooling criterion for VSI growth is    
 \begin{align}\label{iso_vsi_cond}
   \beta <   \beta_\mathrm{crit}  \equiv
   \frac{h|q|}{\gamma-1} . 
 \end{align}
 The thin disk approximation, $h \ll 1$, indicates that $\beta_\mathrm{crit} \ll
 1$, as assumed to obtain Eq. \ref{relax_cond_simp1}---\ref{relax_cond_simp2}. 
 We heuristically explain $\beta_\mathrm{crit}$ in \S\ref{vsi_require}, and numerically
 test its validity in \S\ref{bcrit_num_test}.  
 


% This is clarified in Appendix
% \ref{bcrit_alt}.  
 



% For $M=0$, it is straight foward to solve Eq. \ref{relax_cond_simp1} 
% --- \ref{relax_cond_simp2} exactly and obtain the maximum cooling time
% for the fundamental mode, $\beta_\mathrm{crit}$
% (Eq. \ref{iso_vsi_cond}), which shows that $\beta_\mathrm{crit}\ll 1$
% because we consider thin disks.  

% Eq. \ref{relax_cond_simp1} admits the following   
% approximate solution for the frequency 
% if $\mathrm{min}(1,M)\times|hq\khat|^2\lesssim 1$, 
% \begin{align}
%   % \left(\hat{\omega}_c\khat\right)^2 \simeq (M+1)\left[1  + O( M|h q \khat|^2)\right]  \label{marg_freq_approx}
%   \left(\hat{\omega}_c\khat\right)^2 \simeq (M+1)\left(1 -
%     \frac{M|h q \khat|^2}{2M+1}\right).  \label{marg_freq_approx}
% \end{align}  
% (Equality holds for $M=0$.)  
% Note that Eq. \ref{marg_freq_approx} reduces to the dispersion
% relation for low-frequency inertial waves in the absence of vertical
% shear (see Appendix \ref{stable_novshear} for details). 
% If we further assume $|hq\khat|^2\ll 1$, we obtain from
% Eq. \ref{relax_cond_simp2}---\ref{marg_freq_approx} that  
% \begin{align}\label{bcrit_gen} 
%   \beta_c &\simeq  \frac{h|q|}{(\gamma - 1)}\frac{\sqrt{1+M}}{(1+2M)}.   
% %  \frac{\sqrt{1+M}}{(1+2M)}
% %  \left[1 + \frac{M(2M-1)|h q   \khat|^2}{2(2M+1)^2}\right].
% %  &\lesssim \beta_\mathrm{crit}%\notag
% %>>>>>>> 8816838e4a06974fc61d74b84a2474adaeee3ebc
% \end{align}
% Eq. \ref{bcrit_gen} indicates that higher $M$ modes have lower $\beta_c$,
% i.e.\ require faster cooling for VSI growth. 
%\footnote{
%In fact, this conclusion does not require $\theta\equiv|hq\khat|^2$ to be extremely 
%small. By retaining terms proportional to
% $\theta$, we find $\beta_c\to \beta_c \times 
%  \left[1 + M(2M-1)\theta/2(2M+1)^2 \right]$. Since $\theta\lesssim 1$ by assumption, 
% $\mathrm{max}(\beta_c)$ still occurs at $M=0$.}.     
%(and the trend holds even when the restrictive assumption
%on $|h q \hat{k}|$ is dropped). 
