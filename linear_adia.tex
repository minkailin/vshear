\section{Analytic results with finite cooling times}\label{analytical}  

%\subsection{Isothermal perturbations}\label{iso_pert}
Previous analytic studies of the VSI have largely focused on isothermal
perturbations, with infinitely rapid cooling, as discussed in the introduction.  We further explore
this $\beta = 0$ limit in Appendix \ref{iso_discuss} both to further develop intuition for this 
idealized case and to establish a connection with previous 
works.  However our main interest is the effect of finite cooling times, which
we explore analytically in this section.  

In \S\ref{relax_pert} we show that even an infinitesimal increase in the cooling time,
starting from $\beta = 0$, slows the growth rate of the VSI.  In \S\ref{disp_relax} we 
derive exact solutions to the simplified VSI model developed above (Eq.\ \ref{nearly_iso_explicit}).
In \S\ref{iso_vsi_beta_crit} we reach our main result, the maximum cooling time above which 
VSI growth is suppressed.


%Isothermal perturbations correspond to $\beta = 0$ or $\gamma=1$ so 
%that $\chi = 1$. This case has been investigated by \citetalias{nelson13},
%\citetalias{mcnally14} and \citetalias{barker15}.  
%
%In Appendix \ref{iso_discuss}, we collect some useful results for
%isothermal  perturbations. This includes a necessary condition for
%instability (there must be vertical shear) and an upper bound to the
%instability growth rate (it is limited by the maximum vertical shear
%rate in the disk). Neither of these actually require the thin-disk
%form of the density field described above.  
%We also develop solutions in the thin-disk
%limit, but without assuming low frequency from the onset. 
%The methods used in Appendix \ref{iso_discuss} and results obtained
%therein motivates the solution procedure employed in the following
%sections, and are provided for ease of reference. We refer the reader
%to \citetalias{barker15} for a more detailed treatment of isothermal 
%perturbations. %which also account for more general boundary
%               %conditions than that considered below 

\subsection{Introducing a small but finite
  cooling time}\label{relax_pert}
We are interested in finite thermal relaxation
timescales $\beta > 0$, but it is instructive to
first ask  the more analytically tractable question: how 
do the eigenfrequencies and eigenfunctions change when we change
$\beta$ from zero to a small but finite value? For 
sufficiently small $\beta$ we expect the solution to only differ
slightly from a case with $\beta\equiv 0$. We thus perturb a solution
for $\beta=0$ to see the effect of finite cooling.  

For definiteness, let us consider the simple solution 
\begin{align}
  \beta\equiv 0, \quad \delta v_z = 1,\quad \hat{\upsilon}^2 = \frac{1 +
  \ii h q \hat{k}}{1+\hat{k}^2}, \label{pert_basic} 
\end{align}
which solves Eq. \ref{nearly_iso_explicit} since $\delta
v_z=$constant and $\chi=1$ so that $B=C=0$. 
This is the lowest order VSI mode, the `fundamental corrugation mode',
where the vertical velocity is constant throughout the disk
column. Hereafter we shall simply call it the fundamental
mode. 

The fundamental mode has been observed to dominate numerical simulations
\citep[\citetalias{nelson13}]{stoll14}, and are the ones we find to typically 
dominate in numerical calculations with increasing $\beta$
(\S\ref{therm_relax_eff}) for moderately large    
values of $\khat$, as well as in PPDs 
with a realistic estimate for thermal timescales
(\S\ref{application}). We will see later that consideration of low
order modes in fact provides a useful way to characterize the effect
of increasing the thermal timescale on the VSI. 

We linearize Eq. \ref{nearly_iso_explicit} about the above
solution for $\beta\equiv0$ and write 
\begin{align}\label{nearly_iso_pert}
  \beta \to 0 + \delta\beta,\, \delta v_z\to \delta v_z+\delta
  v_{z1},\,\hat{\upsilon} \to \hat{\upsilon} + \delta\hat{\upsilon}, 
\end{align}
which implies 
\begin{align}
  \chi \to 1 + \delta\chi = 1 + \ii \hat{\upsilon}\left(\gamma-1\right)\delta\beta,
\end{align}
with $\delta v_z$ and $\hat{\upsilon}$ given by Eq. \ref{pert_basic}. We
may then seek  
\begin{align}
  \delta v_{z1} = a \zhat^2 + b,
\end{align}
where $a$, $b$ are constants. 

We insert Eq. \ref{nearly_iso_pert} into
Eq. \ref{nearly_iso_explicit}, keeping only first order terms, and
solve for $\delta\hat{\upsilon}$ using the above expressions for $\delta\chi$
and $\hat{\upsilon}$. We find imaginary part of $\delta\hat{\upsilon}$
is 
% yield the requirements, after
% balancing constants and coefficients of $\zhat^2$, 
% \begin{align}
%   0 &= 2a + 2\hat{\upsilon}\left(1+\khat^2\right)\delta\hat{\upsilon} +
%   \left(\hat{\upsilon}^2 -1\right)\delta\chi, \\
%   0&= 2a\left(1+\ii h q \khat\right) - \left(\khat^2 - \ii h
%     q \khat\right)\delta\chi.
% \end{align}
% This gives 
% \begin{align}\label{nearly_iso_dsig}
%   \delta \hat{\upsilon} =
%   -\frac{\ii\left(\gamma-1\right)\hat{k}^2\left(\ii h
%       q - \hat{k}\right)^2}{2\left(1+\ii h q \hat{k}\right)\left(1+\hat{k}^2\right)^2}\delta\beta.
% \end{align}
% The imaginary part of Eq. \ref{nearly_iso_dsig} is
\begin{align}
  \delta\hat{\sigma} =
  -\frac{\left(\gamma-1\right)\hat{k}^2 \left(\hat{k}^2 -
      2h^2q^2\hat{k}^2 - h^2q^2\right)}{2\left(1+h^2 q^2
      \hat{k}^2\right)\left(1+\hat{k}^2\right)^2}\delta\beta.  
\end{align}
Since $h \ll 1$, introducing finite cooling $\delta\beta>0$
implies $\delta\hat{\sigma} < 0$, i.e. stabilization, since $\gamma>1$. 
% Interestingly, for both
% $\hat{k}\to0$ and $\hat{k}\to\infty$ we find $\delta\hat{\nu}\to0$, so
% stabilization by small finite cooling is ineffective at very large or
% very small scales. 

A finite cooling time allow buoyancy
forces to stabilize vertical motions in sub-adiabatically stratified
disks. The $\zhat^2$ dependence in $\delta v_{z1}$ makes  
sense because it becomes significant at large $|\zhat|$, i.e. away from
the midplane where the effect of buoyancy first appears as $\beta$ is
increased from zero (since $N_z^2\propto z^2$ for a thin disk, see
Eq. \ref{nzsq_def}).   

\subsection{Explicit solutions and dispersion relation}\label{disp_relax}
We now solve Eq. \ref{nearly_iso_explicit} explicitly. We first write  
\begin{align}
  \delta v_z(\zhat) =
  g(\zhat)\exp\left(\frac{\lambda\zhat^2}{2}\right), \label{adia_ansatz}
\end{align}
where the constant $\lambda$ is a constant to be chosen for convenience. Inserting
Eq. \ref{adia_ansatz} into Eq. \ref{nearly_iso_explicit} gives
\begin{align}
  0 = g^{\prime\prime} - \hat{z}\left(A - 2\lambda\right)g^\prime + \left(B +
    \lambda\right)g
  +\left(\lambda^2 - \lambda A - C\right)\zhat^2 g.
\end{align}
We choose $\lambda$ to make the coefficient of $\zhat^2g$
vanish, and impose the vertical kinetic energy density
$\rho|\delta v_z|^2\propto |g|^2 \exp{\left(\real\lambda -
    1/2\right)\zhat^2}$ to remain finite as $|\zhat|\to\infty$. 
Then assuming $g(\zhat)$ is a polynomial, we require  
\begin{align}
  \real\lambda < \frac{1}{2}, 
\end{align}
which amounts to choosing 
\begin{align}
  \lambda = \frac{1}{2}\left(A - \sqrt{A^2 + 4C}\right).  
\end{align} 


Eq. \ref{nearly_iso_explicit} becomes 
\begin{align}
  0 = g^{\prime\prime} - \hat{z}\left(A - 2\lambda\right)g^\prime +
  \left(B + \lambda\right)g.
\end{align}
We seek polynomial solutions 
\begin{align}
  g(\zhat) = \sum_{m=0}^M b_m \zhat^m,
\end{align}
which requires
\begin{align}
  B(\hat{\upsilon}) + \lambda(\hat{\upsilon}) =
  M\left[A-2\lambda(\hat{\upsilon})\right].\label{adia_disp0} 
\end{align}
For ease of analysis, we re-arrange Eq. \ref{adia_disp0} and square it
to obtain a polynomial in $\hat{\upsilon}$,  
\begin{align}
  0 = \sum_{l=0}^{6}c_l\hat{\upsilon}^l,\label{relax_disp}
\end{align}
where the coefficients $c_l$ are given in Appendix \ref{relax_coeff}.
The dispersion relation $\hat{\upsilon}=\hat{\upsilon}(\khat)$ is
complicated and generally requires a numerical solution. However,
simple results may be obtained in limiting cases which we consider
below.  

\subsection{An upper limit to the cooling time}\label{iso_vsi_beta_crit}
Here we estimate the maximum thermal relaxation timescale 
$\beta_c$ that allows growth the VSI.  In Appendix \ref{disp_neut_limit}
we derive the following relation between $\beta_c$ and the wave frequency,
$\hat{\omega}_c$  at marginal stability to the VSI:
\begin{subequations}\begin{align}
 \left(\hat{\omega}_c\khat\right)^4  = &  
% \left[1 + \gamma\beta^2\left(1-\gamma\right)(1+2M)^2\right]\left(\hat{\omega}_c\khat\right)^2 \label{relax_cond_simp1}\\ 
 \left(\hat{\omega}_c\khat\right)^2  + M(M+1)\left(1-h^2q^2\khat^2\right) , \label{relax_cond_simp1} \\
   \left(\hat{\omega}_c\khat\right)^2 = & \frac{\beta}{h q} (\gamma-1)(1+2M)^2
   \left(\hat{\omega}_c\khat\right) \label{relax_cond_simp2}\\
   &- 2M(M+1) . \notag
\end{align}\end{subequations}
These equations hold for $\beta_c \ll 1$ and $\khat \sim 1/\hat{\omega}_c \gg1$,
as explained in Appendix \ref{disp_neut_limit}.


Eq. \ref{relax_cond_simp1} gives an approximate result for the frequency
\begin{align}
% \left(\hat{\omega}_c\khat\right)^2 \simeq (M+1)\left[1  + O( M|h q \khat|^2)\right]  \label{marg_freq_approx}
  \left(\hat{\omega}_c\khat\right)^2 \simeq (M+1)\left(1 -
    \frac{M|h q \khat|^2}{2M+1}\right).  \label{marg_freq_approx}
\end{align}
which holds for either (a) $M= 0$ and arbitrary $|h q \hat{k}|$ or (b) $|h q \hat{k}| \ll 1$ and arbirtary $M \sim O(1)$. 
These distinct conditions mean that our results are most robust for the fundamental vertical mode ($M = 0$) but also
hold for higher order vertical modes when radial wavelengths are not too short.
Note that Eq. \ref{marg_freq_approx} 
reduces (\emph{nearly}?)\footnote{\emph The correspondence is only exact 
if $L = M+1$.  Is this the case (in which case at least note in the appendix) or is there a discrepancy (in which case 
we can't say it reduces)?}
 to the dispersion relation for low-frequency inertial waves 
%in vertically isothermal disks without vertical shear 
(see Appendix\ref{stable_novshear} for details).

Inserting Eq. \ref{marg_freq_approx} into
Eq. \ref{relax_cond_simp2}, we obtain 
\begin{align}\label{bcrit_gen}
  \beta_c &\simeq  \frac{h|q|}{(\gamma - 1)}
  \frac{\sqrt{1+M}}{(1+2M)}
  \left[1 + \frac{M(2M-1)|h q   \khat|^2}{2(2M+1)^2}\right].
  &\lesssim \beta_\mathrm{crit}%\notag
\end{align}
where the above-mentioned approximations still hold, but the correction term ($O(M |h q \khat|^2)$) is now suppressed.
Eq. \ref{bcrit_gen} says that higher $M$ modes have lower $\beta_c$, i.e.\ require faster cooling for VSI growth.
This trend is confirmed numerically in \S\ref{therm_relax_eff} (and the trend holds even when the restrictive assumption
on $|h q \hat{k}|$ is dropped). 

We are most interested in the longest cooling time which allows growth for any $M$.  
This criterion is clearly set by the fundamental, $M = 0$, mode.  Our cooling criterion for 
VSI growth is thus 
 \begin{align}\label{iso_vsi_cond}
  \beta\leq \beta_\mathrm{crit}  \equiv
  \frac{h|q|}{\gamma-1} . 
\end{align}
The thin disk approximation, $h \ll 1$, justifies the $\beta_c \ll 1$ approximation.
 We heuristically explain $\beta_\mathrm{crit}$ in
\S\ref{vsi_require}, and numerically test its validity in \S\ref{bcrit_num_test}. 

%Eq. \ref{iso_vsi_cond} suggests that as the thermal timescale $\beta$
%is increased from zero, higher order modes with $M>0$
%will be stabilized before the fundamental mode. We indeed observe this
%trend in numerical calculations (\S\ref{therm_relax_eff}), despite
%adopting somewhat restrictive assumptions for $M>0$ in order to arrive
%at the simple expression Eq. \ref{bcrit_gen}. 

%We can interpret  $\beta_\mathrm{crit}$ as the thermal timescale below
%which buoyancy becomes ineffective and perturbations are effectively
%isothermal so the VSI can operate.  We see that in a thin disk, a very short thermal timescale
%is required ($\beta_\mathrm{crit}\ll 1$) because
%$\beta_\mathrm{crit}\propto h$. Physically, this is because
%vertical shear is weak in a thin disk, so even a small entropy
%gradient can significantly reduce VSI growth rates.  


