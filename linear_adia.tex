\subsection{Thin-disk approximation}\label{analytic_relax}
Our final approximation apply to thin disks for 
which $\epsilon\ll 1$. Then we may write $\rho/\rho_0\simeq
\exp{(-\zhat^2/2)}$. The governing equation,
Eq. \ref{vertiso_gov_nondim}, is then   
\begin{align}
  \delta v_z^{\prime\prime} - z A\delta v_z^\prime +
  (B - C\zhat^2)\delta v_z = 0,\label{nearly_iso_explicit}
\end{align}
with
\begin{align}
  &A \equiv 1 + \ii \epsilon q \hat{k},\\
  &B \equiv \hat{\sigma}^2\left(\chi + \hat{k}^2\right) -
  \left(\chi + \ii \epsilon q \hat{k}\right),\\
  &C \equiv \left(1-\chi\right)\left(\hat{k}^2 - \ii
    \epsilon q\hat{k}\right), 
\end{align}
and $\chi =
\left(1-\ii\hat{\sigma}\beta\right)/\left(1-\ii\hat{\sigma}\beta\gamma\right)
$ in terms of dimensionless variables. 

We remark that Eq. \ref{nearly_iso_explicit} is in the same 
form as Eq. 41 in \cite{lubow93}, which describe adiabatic axisymmetric waves in
a vertically isothermal disk without vertical shear. The two equations
become identical after  setting $q=0$ (no vertical shear) and
$\chi=1/\gamma$ (adiabatic flow) in our Eq. \ref{nearly_iso_explicit},
and making the low-frequency, nearly-Keplerian approximation in
\citeauthor{lubow93} after a change of variables.    

% The thin disk approximation allows us to develop explicit solutions,
% but some general results may be established without it (see,
% e.g. Appendix \ref{global_corr} and \S\ref{iso_pert}).     

\subsection{Isothermal perturbations}\label{iso_pert}
Isothermal perturbations correspond to $\beta = 0$ or $\chi = 1$. This
case has been investigated numerically by \cite{nelson13} and 
\cite{mcnally14}, the latter of which employed the VGSB framework.  

In Appendix \ref{iso_discuss}, we give an analytical discussion of the
linear problem with $\beta=0$. There, we establish more general
results that do not require the thin disk approximation, namely 
a necessary condition for instability (there must be vertical shear)
and an upper bound to the instability growth rate (it is limited by
the maximum vertical shear rate in the disk). We also develop analytic
solutions in the thin disk limit. The methods used in Appendix
\ref{iso_discuss} and results obtained therein motivates the solution
procedure employed in the following sections.      

\subsection{The effect of introducing a small but finite
  thermal relaxation time}\label{relax_pert}
We are interested in the case $\beta\neq 0$, but it is instructive to
first ask  the more analytically tractable question: how 
do the eigenfrequencies and eigenfunctions change when we change
$t_c=\beta\Omega_k^{-1}$ from zero to a small but finite value? For 
sufficiently small $\beta$ % , for which the perturbations may be 
% considered nearly-isothermal,
we expect the solution to only differ slightly from a case with
$\beta\equiv 0$. We thus perturb a solution for $\beta=0$ to see the
effect of thermal relaxation.  

For definiteness, let us consider the simple solution 
\begin{align}
  \beta\equiv 0, \quad \delta v_z = 1,\quad \hat{\sigma}^2 = \frac{1 +
  \ii\epsilon q \hat{k}}{1+\hat{k}^2}, \label{pert_basic} 
\end{align}
which solves Eq. \ref{nearly_iso_explicit} since $\delta
v_z=$constant and $\chi=1$ so that $B=C=0$. 
This is the fundamental or `corrugation' mode and represents
the entire disk column moving vertically. These modes have been
observed to dominate numerical simulations \citep{nelson13,stoll14},
and are the ones we find to develop in protoplanetary disks
with a realistic estimate for thermal timescales
(\S\ref{application}). 

%To determine the (assumed) small effect of introducing a short thermal
%relaxation, 
We linearize Eq. \ref{nearly_iso_explicit} about the above
solution for $\beta\equiv0$ and write 
\begin{align}\label{nearly_iso_pert}
  \beta \to 0 + \delta\beta,\, \delta v_z\to \delta v_z+\delta
  v_{z1},\,\hat{\sigma} \to \hat{\sigma} + \delta\hat{\sigma}, 
\end{align}
which implies 
\begin{align}
  \chi \to 1 + \delta\chi = 1 + \ii \hat{\sigma}\left(\gamma-1\right)\delta\beta,
\end{align}
with $\delta v_z$ and $\hat{\sigma}$ given by Eq. \ref{pert_basic}. We
may then seek  
\begin{align}
  \delta v_{z1} = a \zhat^2 + b,
\end{align}
where $a$, $b$ are constants. 

Inserting Eq. \ref{nearly_iso_pert} into Eq. \ref{nearly_iso_explicit}
and keeping only first order terms yield the requirements, after
balancing constants and coefficients of $\zhat^2$, 

\begin{align}
  0 &= 2a + 2\hat{\sigma}\left(1+\khat^2\right)\delta\hat{\sigma} +
  \left(\hat{\sigma}^2 -1\right)\delta\chi,\notag\\
  0&= 2a\left(1+\ii\epsilon q \khat\right) - \left(\khat^2 - \ii\epsilon
    q \khat\right)\delta\chi.
\end{align} 

Using the expressions for $\delta\chi$ and $\hat{\sigma}$ we can solve
for $\delta\hat{\sigma}$. This gives 
\begin{align}\label{nearly_iso_dsig}
  \delta \hat{\sigma} =
  -\frac{\ii\left(\gamma-1\right)\hat{k}^2\left(\ii\epsilon
      q - \hat{k}\right)^2}{2\left(1+\ii\epsilon q \hat{k}\right)\left(1+\hat{k}^2\right)^2}\delta\beta.
\end{align}
The imaginary part of Eq. \ref{nearly_iso_dsig} is
\begin{align}
  \delta\hat{\nu} =
  -\frac{\left(\gamma-1\right)\hat{k}^2 \left(\hat{k}^2 -
      2\epsilon^2q^2\hat{k}^2 - \epsilon^2q^2\right)}{2\left(1+\epsilon^2 q^2
      \hat{k}^2\right)\left(1+\hat{k}^2\right)^2}\delta\beta.  
\end{align}
Since $\epsilon \ll 1$, introducing finite cooling $\delta\beta>0$
implies $\delta\hat{\nu} < 0$, i.e. stabilization. Interestingly, for both
$\hat{k}\to0$ and $\hat{k}\to\infty$ we find $\delta\hat{\nu}\to0$, so
stabilization by small finite cooling is ineffective at very large or
very small scales. 


A finite thermal relaxation time introduces buoyancy
forces, which is stabilizing for sub-adiabatically stratified disks
($\gamma > 1$). The $\zhat^2$ dependence in $\delta v_{z1}$ then makes
sense because it becomes significant for large $|\zhat|$, i.e. away from
the midplane where the effect of buoyancy first appears as $\beta$ is
increased from zero. 

\subsection{Explicit solutions and dispersion relation}\label{disp_relax}
%To obtain a dispersion relation similar to that for 
%isothermal perturbations (Eq. \ref{sig2_iso}), 
We now solve Eq. \ref{nearly_iso_explicit} explicitly. We first write  
\begin{align}
  \delta v_z(\zhat) =
  g(\zhat)\exp\left(\frac{\alpha\zhat^2}{2}\right), \label{adia_ansatz}
\end{align}
where $\alpha$ is a constant to be chosen for convenience. Inserting
Eq. \ref{adia_ansatz} into Eq. \ref{nearly_iso_explicit} gives
\begin{align}
  0 = g^{\prime\prime} - \hat{z}\left(A - 2\alpha\right)g^\prime + \left(B +
    \alpha\right)g
  +\left(\alpha^2 - \alpha A - C\right)\zhat^2 g.
\end{align}
We choose $\alpha$ to make the coefficient of $\zhat^2g$
vanish, and impose the vertical kinetic energy density
$\rho|\delta v_z|^2\propto |g|^2 \exp{\left(\real\alpha -
    1/2\right)\zhat^2}$ to remain finite as $|\zhat|\to\infty$. 
Then assuming $g(\zhat)$ is a polynomial, we require  
\begin{align}
  \real\alpha < \frac{1}{2}, 
\end{align}
which amounts to choosing 
\begin{align}
  \alpha = \frac{1}{2}\left(A - \sqrt{A^2 + 4C}\right).  
\end{align} 
%by considering the case $q=0$ and $\beta\to\infty$ as examined in
%\cite{lubow93}.   

Eq. \ref{nearly_iso_explicit} becomes 
\begin{align}
  0 = g^{\prime\prime} - \hat{z}\left(A - 2\alpha\right)g^\prime +
  \left(B + \alpha\right)g.
\end{align}
We seek polynomial solutions 
\begin{align}
  g(\zhat) = \sum_{m=0}^M b_m \zhat^m,
\end{align}
which requires
\begin{align}
  B(\hat{\sigma}) + \alpha(\hat{\sigma}) =
  M\left[A-2\alpha(\hat{\sigma})\right].\label{adia_disp0} 
\end{align}
For ease of analysis, we square Eq. \ref{adia_disp0} to obtain a 
a polynomial in $\hat{\sigma}$,  
\begin{align}
  0 = \sum_{l=0}^{6}c_l\hat{\sigma}^l,\label{relax_disp}
\end{align}
where the coefficients $c_l$ are given in Appendix \ref{relax_coeff}.
The dispersion relation $\hat{\sigma}=\hat{\sigma}(\khat)$ generally
requires a numerical solution. However, simple solutions may be
obtained in limiting cases which we consider below. 

% However, for the fundamental mode
% ($M=0$) some analytic progress can be made, which we consider next.  

\subsection{Upper limit on the thermal relaxation timescale for the
  VSI}\label{iso_vsi_beta_crit}
Here we estimate the maximum thermal relaxation timescale
$\beta$ that allows the VSI. We consider wavenumbers such that
$\khat^2\gg M+1$ which, as we will find, correspond to low frequency
modes. 

%At marginal stability, the eigenfrequency $\hat{\sigma}=\hat{\omega}$ is real.  
%This is a pair of simultaneous
%equations for $\hat{\omega}$ and $\beta$. To proceed, we
%consider further limits. 

Our starting point is Eq. \ref{relax_cond1}---\ref{relax_cond2} which
approximates the dispersion relation Eq. \ref{relax_disp} at marginal
stability with $\khat\gg 1$. We consider modes with $M=0$ (the
fundamental mode) or  $M=O(1)$ and low frequency $|\hat{\omega}|\ll
1$ such that $|\hat{\omega}\khat|$ is $O(1)$ because $\khat\gg 1$. We
also assume $|q|$ and $\beta$ are $O(1)$ or less. For a thin disk
($\epsilon \ll 1$), we have approximately
\begin{align}
 0 = &M(M+1) + \left[1 +
    \gamma\beta^2\left(1-\gamma\right)(1+2M)^2\right]\left(\hat{\omega}\khat\right)^2\notag\\
   &-  \left(\hat{\omega}\khat\right)^4,\label{relax_cond_simp1}\\
   0= &M(M+1) + \frac{\beta}{\epsilon q} (1-\gamma)(1+2M)^2
   \left(\hat{\omega}\khat\right) + \left(\hat{\omega}\khat\right)^2. \label{relax_cond_simp2}
\end{align}
(We explain this in Appendix \ref{disp_neut_limit}.)

Eq. \ref{relax_cond_simp1}---\ref{relax_cond_simp2} is a pair of
simultaneous equations for $\hat{\omega}$ and the critical thermal
timescale $\beta$ for marginal stability. If $\beta\ll 1$,
then a simple solution to Eq. \ref{relax_cond_simp1} is 
\begin{align}
  \left(\hat{\omega}\khat\right)^2 \simeq M+1, 
\end{align}
which is the dispersion relation for low-frequency waves in vertically
isothermal disks without vertical shear (see Appendix
\ref{stable_novshear}). Then  
\begin{align}\label{bcrit_gen}
  \beta = \frac{\epsilon|q|}{(\gamma - 1)}
  \frac{(1+M)^{3/2}}{(1+2M)^2} \leq \beta_\mathrm{crit} 
\end{align}
where 
\begin{align}\label{iso_vsi_cond}
  \beta_\mathrm{crit}  \equiv \frac{\epsilon|q|}{\gamma-1}
\end{align}
is the critical thermal timescale for the fundamental mode
($M=0$). However, according to Eq. \ref{bcrit_gen}  the critical
thermal timescale decreases with increasing  $M$. We thus expect
higher order modes to be stabilized more rapidly by finite thermal
relaxation. When $\beta>\beta_\mathrm{crit}$, all modes
(satisfying the above assumptions) are stabilized. We can interpret
$\beta_\mathrm{crit}$ as the thermal timescale below which
perturbations are effectively isothermal to allow the VSI.  We will confirm
Eq. \ref{iso_vsi_cond} numerically in \S\ref{bcrit_num_test}.  

(We remark that the critical thermal timescale $\beta_\mathrm{crit}$
for the fundamental mode may be obtained more readily with fewer
assumptions by setting $M=0$ in
Eq. \ref{relax_cond1}---\ref{relax_cond2} from the onset. ) 

Rapid thermal relaxation is required to overcome a stabilizing
vertical entropy gradient \citep{goldreich67,urpin98,urpin03}. This 
conclusion was reached in these studies by considering
vertically-localized disturbances. % , although a definitive timescale
                                % was not presented.  
By contrast, our Eq. \ref{iso_vsi_cond} quantifies this requirement
for vertically global disturbances. We see that in a thin disk, a very
short thermal timescale is required ($\beta_\mathrm{crit}\ll 1$)
because $\beta_\mathrm{crit}\propto \epsilon$. 
This suggests that finite thermal relaxation rapidly stabilizes the
VSI.   






% Here we estimate the maximum thermal relaxation timescale
% $\beta$ that allows the fundamental VSI mode for $\khat\gg1$.   

% At marginal stability, the eigenfrequency $\hat{\sigma}=\hat{\omega}$
% is real. The dispersion relation for the fundamental mode is then
% \begin{align}\label{relax_disp_fund}
%   0 = \sum_{l=1}^{6}c_l\hat{\omega}^l \quad \text{for $M=\hat{\nu}=0$}.  
% \end{align} 
% Taking the real and imaginary parts of Eq. \ref{relax_disp_fund} and considering 
% $\khat\gg1$, we find
% \begin{align}
%   0 =& \left[1 + \gamma\beta^2\left(1-\gamma\right)\right] + 2\epsilon q
%   \gamma\beta (\hat{\omega}\khat) -  (\hat{\omega}\khat)^2 \notag\\
%   &+ \beta^2\gamma^2\hat{\omega}^2 (\hat{\omega}\khat)^2,\label{relax_cond1}\\
%   0=& \beta(1-\gamma)\khat^2 + \epsilon q \khat^2 (\hat{\omega}\khat)
%   - 2\beta (\hat{\omega}\khat)^2 - \epsilon q \gamma^2\beta^2 (\hat{\omega}\khat)^3 \notag\\
%     &+ 2\beta\gamma(\hat{\omega}\khat)^4.  
% \label{relax_cond2}
% \end{align}
% This is a pair of simultaneous equations for
% $\hat{\omega}$ and $\beta$. Now, low-frequency axisymmetric waves in
% the absence of vertical shear are stable and  satisfy
% $\hat{\omega}\propto \khat^{-1}$ (Appendix \ref
% {stable_novshear}, Eq. \ref{iso_disp_full}).  We are thus motivated to
% seek solutions to Eq. \ref{relax_cond1}---\ref{relax_cond2} with  
% % we wish to seek a criterion independent of k 
% % \begin{align}
% %   \khat \to \infty, \quad \hat{\omega}\to 0 ,\quad \hat{\omega} \khat\text{ finite}.
% %\end{align}
% \begin{align}
%   \khat \gg1, \quad |\hat{\omega}|\ll 1 ,\quad
%   |\hat{\omega}\khat|=O(1). 
% \end{align}
% Then
% \begin{align}
%   \hat{\omega}\khat = \frac{(\gamma-1)\beta}{\epsilon q}
% \end{align}
% from Eq. \ref{relax_cond2}, which implies, from Eq. \ref{relax_cond1}
% that
% \begin{align}\label{beta_crit0}
%   \beta = \frac{1}{(\gamma-1)}\left[\frac{1}{\left(\epsilon
%         q\right)^2} - \frac{\gamma}{(\gamma-1)}\right]^{-1/2} 
% \end{align}
% at marginal stability for perturbations with large $\khat$. 

% From our discussion in \S\ref{relax_pert}, which indicate introducing
% a finite thermal relaxation is stabilizing, we expect
% Eq. \ref{beta_crit0} is an upper limit. For a thin  
% disk $\epsilon\ll 1$, we thus infer 
% \begin{align}\label{iso_vsi_cond}
%   \beta \lesssim \frac{|\epsilon q|}{\gamma-1} \equiv
%   \beta_\mathrm{crit} 
% \end{align}
% is required for the fundamental VSI to operate. That is, when
% $\beta<\beta_\mathrm{crit}$, perturbations are  effectively isothermal. 
% We will confirm Eq. \ref{iso_vsi_cond} numerically in
% \S\ref{bcrit_num_test}. 