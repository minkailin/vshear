\section{Adiabatic disks} 
We now examine the VSI in adiabatic disks ($t_c\to\infty$). For 
analytical discussion we consider the initially isothermal case with 
$\Gamma=1$. We eliminate variables in favor of $\delta v_z$ and
$\Delta \equiv \nabla\cdot\delta\bm{v}$ to obtain
\begin{align}
&\Delta\left(1 + \frac{\gamma c_s^2 k_x^2}{D}\right) = \frac{d\delta
  v_z}{dz} + \frac{\ii k_x^2}{D}\left(\ii c_s^2 \frac{d\ln\rho}{dz}
  - \frac{r}{k_x}\frac{d\Omega^2}{dz}\right)\delta v_z,\label{adia_iso1}\\
& -\sigma^2\delta v_z = \gamma c_s^2 \frac{d\Delta}{dz} +
 c_s^2\frac{d}{dz}\left(\frac{d\ln\rho}{dz}\delta v_z\right) +
 \left(\gamma-1\right)\frac{d\ln\rho}{dz} \Delta.\label{adia_iso2}
\end{align}
{\bf note: incompressible mode, set delta to zero. only works for
  constant gravity.}

We again make the low-frequency approximation
and ignore the vertical dependence of $\Omega$ and $\kappa$ except
where their vertical derivatives appear explicitly in the above
equations. We thus make the replacement $D\to \Omega_k^2$, as before
for isothermal perturbations. Here, we also use
the fact that in the global disk,
\begin{align}
  r\frac{\p \Omega^2}{\p z} = - \frac{\p\ln\rho}{\p z}\frac{\p
    c_s^2}{\p r} = -\frac{\p \ln\rho}{\p z} \frac{qc_s^2}{r},
\end{align}
where the second equality follows from the power-law prescription of
the sound-speed for $\Gamma=1$. 

Combining Eq. \ref{adia_iso1} and Eq. \ref{adia_iso2} with the above
substitutions we obtain, in terms of dimensionless variables defined previously, an equation for $\delta v_z$,
\begin{align}
  0 =& \dd v_z^{\prime\prime} + \left(1 + \ii \epsilon q
    \hat{k}\right)\left(\ln\rho^{\prime}\delta v_z\right)^\prime\notag\\
  &+\left\{\hat{\sigma}^2\left(\frac{1}{\gamma}+\hat{k}^2\right)\right.\notag\\
  &\phantom{0=}\left.-\left(\frac{\gamma-1}{\gamma}\right)\left[\ln\rho^{\prime\prime}+\hat{k}^2\left(1-\frac{\ii\epsilon
          q}{\hat{k}}\right)\ln\rho^{\prime 2}\right]\right\}\delta v_z.\label{adia_iso3}
\end{align}
We multiply Eq. \ref{adia_iso3} by $\rho\delta v_z^*$ and
integrate vertically, assume boundary terms vanish when integrating by
parts, to obtain
\begin{align}
  &\hat{\sigma}^2\left(\frac{1}{\gamma} +
    \hat{k}^2\right)\int_{\zhat_1}^{\zhat_2}\rho|\delta
  v_z|^2 d\zhat \notag\\
  &=  \left(\frac{\gamma-1}{\gamma}\right)
  \int_{\zhat_1}^{\zhat_2}\rho|\delta v_z^\prime|^2 d\zhat
  +\frac{1}{\gamma}\int_{\zhat_1}^{\zhat_2}\frac{1}{\rho}|(\rho\delta
  v_z)^\prime|^2 d\zhat\notag\\
  &\phantom{=} +
  \left(\frac{\gamma-1}{\gamma}\right)\hat{k}^2\left(1-\frac{\ii\epsilon
      q}{\hat{k}}\right) \int_{\zhat_1}^{\zhat_2}\rho\ln\rho^{\prime
    2}|\delta v_z|^2 d\zhat\notag\\
  &\phantom{=} + \ii\epsilon q \hat{k}
  \int_{\zhat_1}^{\zhat_2}\ln\rho^\prime(\rho\delta v_z^*)^\prime
  \delta v_z d\zhat.\label{adia_integral}
\end{align}
When $q\equiv0$, all the terms on the right-hand-side (RHS) are real. Then
$\hat{\sigma}^2$ is real, and  $\hat{\sigma}^2>0$ if $\gamma>1$. As
expected, a sub-adiabatically stratified disk is stable in the absense
of vertical shear. In the presence of vertical shear $q\neq0$ and
$\hat{\sigma}$ is generally complex, which implies the possibility of
instability.  

Note that the term with $\ln\rho^{\prime 2}$ in the integrand (third
term on the RHS) represents stabilization by the vertical entropy
gradient when there is no vertical shear. It gives an imaginary
contribution to $\hat{\sigma}$ when $q\neq0$, but this is a factor
$\hat{k}$ smaller than the real (stabilizing) contribution, since we
consider $\hat{k}\gg1$. Thus, instability is expected to be due to the
last term on the RHS of Eq. \ref{adia_integral}. 
  
\subsection{Explicit solutions in the thin-disk limit}
To simplify further, we consider the problem in the thin-disk limit so
that $\ln \rho \simeq -\zhat^2/2$. Eq. \ref{adia_iso3} becomes 
\begin{align}
  0 = \delta v_z^{\prime\prime} - \zhat A \delta v_z^\prime + \left(B
    - C \zhat^2\right)\delta v_z,
\end{align}
with
\begin{align}
  &A \equiv 1 + \ii \epsilon q \hat{k},\\
  &B \equiv \hat{\sigma}^2\left(\frac{1}{\gamma} + \hat{k}^2\right) -
  \left(\frac{1}{\gamma} + \ii \hat{k} \epsilon q\right)\\
  &C \equiv \frac{\left(\gamma-1\right)}{\gamma}\hat{k}^2\left(1 - \frac{\ii
      \epsilon q}{\hat{k}}\right).\label{adia_thin}
\end{align}
Let
\begin{align}
  \delta v_z(\zhat) =
  g(\zhat)\exp\left(\frac{\alpha\zhat^2}{2}\right), \label{adia_ansatz}
\end{align}
where $\alpha$ is a constant to be chosen for convenience. We require
the vertical kinetic energy density $\rho|\delta v_z|^2$ to remain
finite.  Then assuming
$g(\zhat)$ is a polynomial, we require 
\begin{align}
  \real\alpha < \frac{1}{2}. 
\end{align}

Inserting the ansatz Eq. \ref{adia_ansatz} into Eq. \ref{adia_thin}
gives
\begin{align}
  0 = g^{\prime\prime} - \hat{z}\left(A - 2\alpha\right)g^\prime + \left(B +
      \alpha\right)g
    +\underbrace{\left(\alpha^2 - \alpha A - C\right)}_\text{set to zero}\zhat^2 g.
\end{align}
We choose $\alpha$ to make the coefficient of $\zhat^2g$ vanish. That
is,
\begin{align}
  \alpha = \frac{1}{2}\left(A - \sqrt{A^2 + 4C}\right).
\end{align} 
We now have an equation in the same form as Eq. \ref{iso_ode3}:
\begin{align}
  0 = g^{\prime\prime} - \hat{z}\left(A - 2\alpha\right)g^\prime +
  \left(B + \alpha\right)g.
\end{align}
We seek polynomial solutions 
\begin{align}
  g(\zhat) = \sum_{m=0}^M b_m \zhat^m,
\end{align}
% We again seek power-series solutions 
% \begin{align}
%   g(\zhat) = \sum_{m=0}^\infty b_m \zhat^m,
% \end{align}
% which requires
% \begin{align}
%   (m+2)(m+1)b_{m+2} + \left[\left(2\alpha - A\right)m + B + \alpha\right]=0.
% \end{align}
which requires
\begin{align}
  B + \alpha = M\left(A-2\alpha\right).\label{adia_disp}
\end{align}
Inserting the definition of $B$ we get
\begin{align}
  &\hat{\sigma}^2\left(\frac{1}{\gamma}+\hat{k}^2\right) \notag\\ &=
  \left(\frac{1}{\gamma}-\frac{1}{2}\right) +\frac{\ii}{2}\epsilon q
  \hat{k} + \frac{1}{2}\left(2M+1\right)\left(A^2 + 4C\right)^{1/2}. 
\end{align}
%Recalling that $B$ depends $\hat{\sigma}$, Eq. \ref{adia_disp} gives
%the eigenfrequency as a function of disk paramters. 

% \subsection{Special case of a thin-disk with $\gamma=2$} 
% In principle, we can solve Eq. \ref{adia_disp} for the mode
% frequencies and growth rates. This is complicated in the
% general case, but simplifies considerably for $\gamma=2$. 