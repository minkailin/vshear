
% \section{Alternative inference of instability in the presence of weak
%   vertical shear}\label{pert_theory}
% We can also investigate the destabilizing effect of vertical shear
% without explicitly solving the full governing equation,
% Eq. \ref{iso_ode3}. We first
% consider a system with $q\equiv0$, for which the eigenfunctions are
% Hermite polynomials and the real eigenfrequencies are known. We then
% perturb this system by introducing a weak vertical shear ($|q|\ll1$)
% and linearize the governing equation. This procedure is
% \begin{align}   
%   q \to 0 + \delta q,\quad
%   W \to \He_n + \delta W,\quad
%   \hat{\sigma} \to \hat{\sigma} + \delta\hat{\sigma}. 
% \end{align}
% %where for this exercise we restore $\sigma$ as the eigenfrequency. 

% Linearizing the integral relation Eq. \ref{integral_relation1} in the
% thin-disk limit, and taking the
% imaginary part, we have
% \begin{align}
%   2\hat{\sigma}\imag(\delta\hat{\sigma})
%   \left(1+\hat{k}^2\right) \int_{-\infty}^{\infty} w(\zhat)
%   \He_n^2(\hat{z}) d\hat{z}
%   = \delta q \epsilon \hat{k} 
%   \int_{-\infty}^{\infty}
%   w(\zhat)\hat{z}\He_n(\hat{z})\He_n^\prime(\zhat) d\hat{z}
% \end{align}
% Recognizing $\hat{z} = \He_1(\hat{z})$ and using $\He_n^\prime = n
% \He_{n-1}$ we have
%  \begin{align}
%    2\hat{\sigma}\imag(\delta\hat{\sigma})
%    \left(1+\hat{k}^2\right) \int_{-\infty}^{\infty} w(\zhat)
%    \He_n^2(\hat{z}) d\hat{z}
%    =\delta q \epsilon \hat{k} 
%    \int_{-\infty}^{\infty}
%    w(\zhat)\He_1(\hat{z})\He_n(\hat{z})n\He_{n-1}(\hat{z})d\hat{z}. 
%  \end{align}

% Finally, we evaluate the integrals using the results
% \begin{align}
%   \int_{-\infty}^{\infty}
%   w(\xi) \He_k(\xi)\He_l(\xi) d\xi = \sqrt{2\pi}k!\delta_{kl} \quad
% % \end{align}
% \text{and} \quad
% % \begin{align}
%   \int_{-\infty}^{\infty}
%   w(\xi) \He_m(\xi)\He_k(\xi)\He_l(\xi) d\xi =
%   \frac{\sqrt{2\pi}m!k!l!}{(j-m)!(j-k)!(j-l)!}, 
% \end{align}
% where $j = (m+k+l)/2$. We then obtain
%  \begin{align}
%    2\hat{\sigma}\imag(\delta\hat{\sigma})
%   \left(1+\hat{k}^2\right) = \delta q \epsilon \hat{k} n.
%  \end{align}
% Inserting the original eigenfrequency $\hat{\sigma} = \pm
% \sqrt{n}(1+\hat{k}^2)^{-1/2}$ gives
% \begin{align}
%   \imag(\delta\hat{\sigma})= \pm \frac{1}{2}\delta q \epsilon
%   \sqrt{n} \frac{\hat{k}}{\sqrt{1+\hat{k}^2}}. 
% \end{align}
% This result agrees with Eq. \ref{simple_growth} in the limit
% $|q|\to0$. 
% %for $n=1$, since the
% %function $\He_1$ solves the governing equation exactly. 
% For $\hat{k}\gg 1$ we have
% \begin{align}
%   \imag(\delta\hat{\sigma})\simeq \pm \frac{1}{2}\delta q \epsilon
%   \sqrt{n} \sgn{\hat{k}}. 
% \end{align}




% Differentiating Eq. \ref{adia_iso1} properly gives
% \begin{align}
%   \left(D + \gamma k_x^2 c_s^2\right)\frac{d\Delta}{dz}=& D
%   \frac{d^2\delta v_z}{dz^2} + k_x^2c_s^2\left[\frac{\gamma}{\left(D + \gamma k_x^2 c_s^2\right)}\frac{dD}{dz} + \ii
%     \frac{d\ln\rho}{dz}\left(\ii +
%       \frac{q}{k_xr}\right)\right]\frac{d\delta v_z}{dz}\notag\\
%   &+ \ii k_x^2 c_s^2 \left(\ii +
%       \frac{q}{k_xr}\right)\left[\frac{d^2\ln\rho}{dz^2} -
%       \frac{1}{\left(D + \gamma
%           k_x^2c_s^2\right)}\frac{dD}{dz}\frac{d\ln\rho}{dz}\right]\delta
%     v_z. \label{adia_diso1}
% \end{align}
% The terms that were ignored in deriving Eq. \ref{adia_iso3} are those
% proportional to $dD/dz$ in Eq. \ref{adia_diso1}. Now, in the global
% disk with $\Gamma=1$ we have
% \begin{align}\label{dkappa2}
%   \frac{\p\kappa^2}{\p z} = 4 \frac{\p\Omega^2}{\p z} + r\frac{\p}{\p
%     r}\frac{\p\Omega^2}{\p z} = -
%   \frac{\p\ln\rho}{\p z}\frac{qc_s^2}{r^2} \left(2 + q +
%     \frac{z^2/r^2-2}{z^2/r^2+1}\right). 
% \end{align}
% For the local problem we may then write
% \begin{align}
%   \frac{dD}{dz}  = - \frac{d\ln\rho}{dz}\frac{qc_s^2}{r^2}F(z;q),
% \end{align}
% where the function $F$ corresponds to the bracket in
% Eq. \ref{dkappa2}. This function varies from $F=q$ at $z=0$ to $F\to
% 3+q$ as $|z|\to\infty$, i.e. its magnitude is of order unity. 
% % We estimate the
% % importance of these terms by noting that 
% % \begin{align}
% %   \frac{dD}{dz}\equiv \frac{d\kappa^2}{dz} \simeq \frac{d\Omega^2}{dz}
% %   = - \frac{d\ln\rho}{dz}\frac{qc_s^2}{r^2},
% % \end{align}
% % for thin disks. 
% Eq. \ref{adia_diso1} becomes
% \begin{align}
% \left(D + \gamma k_x^2 c_s^2\right)\frac{d\Delta}{dz}=& D
%   \frac{d^2\delta v_z}{dz^2} - k_x^2c_s^2\frac{d\ln\rho}{dz}\left[ 1 - 
%       \frac{\ii q}{k_xr}  +  \frac{\gamma q c_s^2F}{r^2\left(D +
%         \gamma k_x^2 c_s^2\right)}\right]\frac{d\delta
%     v_z}{dz}\notag\\ 
%   &- k_x^2 c_s^2 \left(1 - 
%       \frac{\ii q}{k_xr}\right)\left[\frac{d^2\ln\rho}{dz^2} + 
%       \frac{qc_s^2F}{r^2\left(D + \gamma
%           k_x^2c_s^2\right)}\left(\frac{d\ln\rho}{dz}\right)^2\right]\delta
%     v_z. \label{adia_diso2}
% \end{align}
% Since $D\sim \Omega_k^2$ in the low-frequency limit, we 
% see from Eq. \ref{adia_diso2} that the neglected terms are
% $O(\epsilon^2)$. We can make this explicit by combining
% Eq. \ref{adia_diso1}---\ref{adia_diso2} and Eq. \ref{adia_iso2}, then
% set $D\to\Omega_k^2$. In terms of non-dimensional variables, the result
% is  
% \begin{align}
%    &\delta v_z ^{\prime\prime} + \left(1 + \ii\epsilon q\hat{k} -
%     \frac{\gamma q \epsilon^2\hat{k}^2F}{1+\gamma
%       \hat{k}^2}\right)\ln\rho^{\prime}\delta v_z^\prime +
%   \left[\left(\frac{1}{\gamma} + \ii \epsilon q
%       \hat{k}\right)\ln\rho^{\prime\prime} - \hat{k}^2\left(1 -
%       \frac{\ii\epsilon
%         q}{\hat{k}}\right)\left(\frac{\gamma-1}{\gamma} +
%       \frac{q\epsilon^2F}{1+\gamma\hat{k}^2}\right)\ln\rho^{\prime
%       2}\right]\delta v_z \notag\\&=
%   -\hat{\sigma}^2\left(\frac{1}{\gamma} + \hat{k}^2\right)\delta v_z\label{adia_diso3}.
% \end{align}   
% Eq. \ref{adia_diso3} differs from Eq. \ref{adia_iso3} by terms
% proportional to $\epsilon^2$. For a thin disk, $\epsilon\ll1$, so
% neglecting these terms has no qualitative effect on the discussion in
% \S\ref{analytic_adia}. 

% \section{Axisymmetric instability in vertically 
%   isothermal disks in global cylindrical geometry}\label{global_vertiso}
% Consider a locally isothermal disk in 3D such that $P=c_s^2\rho$ at
% all times, so in the perturbed state $\delta P =
% c_s^2\delta\rho$, with $c_s$ begin a prescribed function of position. 
% The axisymmetric linearized fluid equations in cylindrical
% geometry is then 
% \begin{align}
%  \ii\sigma \delta v_r + 2\Omega\delta v_\phi &= c_s^2\frac{\p}{\p r}\left(\frac{\delta\rho}{\rho}\right),\\
%  \ii\sigma \delta v_\phi &= \frac{1}{r}\delta\bm{v}\cdot\nabla j,\\
%  \ii\sigma \delta v_z &= c_s^2\frac{\p}{\p
%    z}\left(\frac{\delta\rho}{\rho}\right),\\
%  \ii\sigma\delta\rho &= \nabla\cdot\left(\rho\delta\bm{v}\right),
% \end{align}
% and we recall $j=r^2\Omega$. We can eliminate $\delta v_\phi$ from these equations to obtain
% \begin{align}
%   -\sigma^2\rho\left(|\delta v_r|^2 + |\delta v_z|^2\right) +
%   \kappa^2\rho|\delta v_r|^2 + \frac{2\Omega}{r}\frac{\p j}{\p z} \rho
%   \delta v_r^*\delta v_z =
%   c_s^2\rho\delta\bm{v}^*\cdot\nabla\left(\frac{\ii\sigma\delta\rho}{\rho}\right)
%   =c_s^2\rho\delta\bm{v}^*\cdot\nabla\left[\frac{1}{\rho}\nabla\cdot\left(\rho\delta\bm{v}\right)\right].\label{global_integral} 
% \end{align}
% At this point if we make the anelastic approximation by assuming
% $\nabla\cdot\left(\rho\delta\bm{v}\right)$ is negligible in this
% equation \citep[as argued by][]{nelson13}, 
% then the only contribution to instability comes from the third term on
% the left hand side, i.e. the explicit vertical shear term
% $\p_zj$. This is the same conclusion reached in
% \S\ref{integral_relation}. 

% The term proportional to $\kappa^2$ is stabilizing. Because radial shear
% is much larger than vertical shear, this implies that significant
% instability can only be achieved with large vertical
% motions (i.e. $|\delta v_z|\gg|\delta v_r|$). 

% If we integrate Eq. \ref{global_integral} over the fluid volume 
% and assume boundary terms vanish when integrating by parts, we obtain 
% \begin{align}
%   -\sigma^2\int \rho\left(|\delta v_r|^2 + |\delta v_z|^2\right) dV +
%   \int \kappa^2\rho|\delta v_r|^2 dV + \int \frac{2\Omega}{r}\frac{\p j}{\p z} \rho
%   \delta v_r^*\delta v_z dV = -\int
%   \frac{c_s^2}{\rho}\left|\nabla\cdot\left(\rho\delta\bm{v}\right)\right|^2
%   dV - \int \nabla\cdot\left(\rho\delta\bm{v}\right) \delta
%   \bm{v}^*\cdot\nabla c_s^2.
% \end{align}
% Note that the treatment so far is in fact valid for any prescribed
% $c_s(r,z)$. However, to proceed further we specialize to the case
% $c_s=c_s(r)$ for simplicity, i.e. vertically isothermal
% disks. Recalling $r\p_z\Omega^2 = -\p_z\ln{\rho}\p_r
% c_s^2$  for a vertically isothermal disk (Eq. \ref{vertical_shear}), we obtain 
% \begin{align}
%   \sigma^2\int \rho\left(|\delta v_r|^2 + |\delta v_z|^2\right) dV =
%   \int \kappa^2\rho|\delta v_r|^2 dV + \int
%   \frac{c_s^2(r)}{\rho}\left|\nabla\cdot\left(\rho\delta\bm{v}\right)\right|^2
%   dV + \int \left[ \rho \left(\nabla\cdot\delta\bm{v}\right)\delta
%     v_r^* + |\delta v_r|^2\frac{\p \rho}{\p r}\right]\frac{dc_s^2(r)}{dr}  dV.
% \end{align}
% We see that instability is only possible if
% $dc_s^2/dr\neq0$. Furthermore, since density and temperature typically
% decrease outwards in astrophysical disks, instability is associated
% with the term proportional to $\nabla\cdot\delta\bm{v}$. It is also
% clear that $\sigma^2$ will be complex in general.
