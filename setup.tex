\section{Baroclinic disk equilibra}
We consider  an inviscid, non-self-gravitating three-dimensional (3D)
disk orbiting a central star of mass $M_*$. 
We adopt cylindrical
co-ordinates $(r,\phi, z)$ centered on the star to describe the global
disk. The equilibrium disk is steady and axisymmetric with density,
pressure and rotation profiles $\rho(r,z)$, $P(r,z)$ and
$\Omega(r,z)$, respectively. 

We examine disks with power-law profiles for the midplane density
$\rho_0(r)\equiv\rho(r,0)$: 
\begin{align}
  \rho_0(r) = \rho_{00}\left(\frac{r}{r_0}\right)^p,
\end{align}
where $\rho_{00}$ is the midplane density at the fiducial radius
$r=r_0$ and $p$ is the power-law index. The equilibrium pressure $P$
and density $\rho$ are related by 
\begin{align}
  P(r,z) &= 
  \frac{c_{s0}^2(r)}{\Gamma}\rho_0(r)^{1-\Gamma}\rho^\Gamma(r,z),\notag\\
  & \equiv K(r)\rho^{\Gamma}(r,z). \label{eqm_press}
\end{align}
where the polytropic index $\Gamma$ is a parameter and 
the midplane sound-speed $c_{s0}(r)$ is taken to be another power-law: 
\begin{align}
  c_{s0}^2(r)=c_{s00}^2\left(\frac{r}{r_0}\right)^q, 
\end{align}
where $c_{s00}\equiv c_{s0}(r_0)$  and $q$ is the power-law index for
the midplane temperature. We define
\begin{align}
  \epsilon(r) \equiv \frac{c_{s0}(r)}{v_k(r)}
\end{align}
to parametrize the magnitude of the sound-speed, where
$v_k=r\Omega_k=\sqrt{GM_*/r}$ is the Keplerian azimuthal velocity and
$G$ is the gravitational constant. 
The global sound-speed $c_s(r,z)$ is given by
\begin{align}
  c_s^2\equiv \frac{dP}{d\rho} =
  c_{s0}^2(r)\left(\frac{\rho}{\rho_0}\right)^{\Gamma-1}. 
\end{align}

The equilibrium disk satisfies
\begin{align}
  0 &= \frac{1}{\rho}\frac{\p P}{\p z} + \frac{\p \Phi_*}{\p z},\label{vert_equilibrium}\\
  r\Omega^2&= \frac{1}{\rho}\frac{\p P}{\p r} + \frac{\p \Phi_*}{\p
    r},\label{hori_equilibrium} 
\end{align} 
where $\Phi_* = -GM_*(r^2 + z^2)^{-1/2}$ is the gravitational
potential of the central star. Note that the disk is in general
baroclinic with $\Omega = \Omega(r,z)$, since
Eq. \ref{vert_equilibrium}---\ref{hori_equilibrium} imply, with
Eq. \ref{eqm_press},  
\begin{align}
  r\frac{\p\Omega^2}{\p z} = - \rho^{\Gamma-1}\frac{\p\ln\rho}{\p z}\frac{dK}{dr}.
\end{align}


The explicit expression for the density field for
$\Gamma\neq1$ is given  by
\begin{align}\label{eqm_dens}
  &\left(\frac{\rho}{\rho_0}\right)^{\Gamma-1} = 1 +
  \frac{\left(\Gamma-1\right)}{\epsilon^2}\left(\frac{1}{\sqrt{1+z^2/r^2}}-1\right),
\end{align}
Note that for $\Gamma\neq1$ the disk has a surface $z=H(r)$ such that
$\rho(r,H)=0$. If we parametrize
\begin{align}
  H(r)\equiv h r,
\end{align}
then the imposed sound-speed and disk thickness are related by 
\begin{align}
  h^2(r) = \left[1-\frac{\epsilon^2(r)}{\Gamma-1}\right]^{-2}-1. 
\end{align}

Similarly, the equilibrium rotation is 
\begin{align}\label{eqm_rot}
  \frac{\Omega^2(r,z)}{\Omega_k^2(r)}=1 +
  \left(p+\frac{s}{\Gamma}\right)\epsilon^2(r) 
  +\frac{s}{\Gamma} \left(1-\frac{1}{\sqrt{1+z^2/r^2}}\right), 
\end{align}
where $s\equiv q+p(1-\Gamma)$. We also define the radial shear rate  
\begin{align}
  S(r,z) \equiv r\frac{\p\Omega}{\p r},  
\end{align}
and the square of the epicyclic frequency $\kappa$ 
\begin{align}
  \kappa^2(r,z) \equiv 2\Omega(2\Omega + S). 
\end{align}

% {\bf need to mention: generally consider sub-keplerian disks -> caveat is that it's entirely possble for angular 
% velocity to reach zero at finite height. give example later for isothermal disk. this limits domain size, but this critical height is usually
% very large for thin disks}

\subsection{Vertically isothermal disks}
For the special case $\Gamma=1$ the disk is isothermal in the
vertical direction. The equilibrium density $\rho(r,z)$ and rotation
$\Omega(r,z)$ is then 
\begin{align}
  \frac{\rho(r,z)}{\rho_0(r)} &=
  \exp{\left[\frac{1}{\epsilon^2}\left(\frac{1}{\sqrt{1+z^2/r^2}}-1\right)\right]}\\    
  \frac{\Omega^2(r,z)}{\Omega_k^2(r)}& =1+ (p+q)\epsilon^2(r) + q\left(1 -
    \frac{1}{\sqrt{1+z^2/r^2}}\right),
\end{align}
which may be obtained from Eq. \ref{eqm_dens} and Eq. \ref{eqm_rot} by
taking the limit $\Gamma\to 1$. The vertically isothermal disk has no
surface, but we can define a characteristic scale-height 
\begin{align}
  \Hiso \equiv \epsilon r.% \quad\quad(\Gamma \equiv 1),   
\end{align} 
and in this case the characteristic aspect-ratio $h = \epsilon$. 

\section{The vertically global shearing box} 
We are interested in the axisymmetric stability of the above system,
on radial length-scales much smaller than the local radius $r$, but
may be global in the vertical direction. As such, we work within the
framework of the \emph{vertically global shearing 
  box} (VGSB) developed by \cite{mcnally14}. In this approximation
one sets up a local Cartesian co-ordinate  system $(x,y,z)$ centered
on a fiducial point $(r,\phi,z)=(r_0,\phi_0,0)$ in the global disk, with $\phi_0(t) =
\Omega(r_0,0)t$. The local $x,y,z$ axis correspond to the
global radial, azimuthal and vertical directions.  

In the this shearing box the equilibrium density, pressure and
sound-speed are assumed to depend only on $z$, by evaluating these
quantities at $r=r_0$ in the global disk. The background rotation is
approximated by a linear shear in $x$, but left unaltered in $z$.     

The axisymmetric VGSB equations read 
\begin{align}
  & \frac{\p\rho}{\p t} + \frac{\p}{\p x}\left(\rho v_x\right) +
  \frac{\p}{\p z}\left(\rho v_z\right)=0,\\
  & \frac{\p\bm{v}}{\p t} + \bm{v}\cdot\nabla\bm{v} +
  v_zr_0\frac{d\Omega_0}{dz} \hat{\bm{y}} = - 
  2\Omega_0\hat{\bm{z}}\times\bm{v} - S_0v_x\hat{\bm{y}} \notag\\ 
  &\phantom{\frac{\p\bm{v}}{\p t} + \bm{v}\cdot\nabla\bm{v} +
    v_zr_0\frac{d\Omega_0}{dz} \hat{\bm{y}}=}-\frac{1}{\rho}\nabla P - \frac{\p\Phi_0}{\p z}\hat{\bm{z}},\\
  & \frac{\p P}{\p t} + \bm{v}\cdot\nabla P = -\gamma P
  \nabla\cdot\bm{v} -
  \frac{\rho}{t_c}\left(\frac{P}{\rho}-\left.\frac{P}{\rho}\right|_{t=0}\right), 
\end{align}
where $\bm{v}$ is the velocity field relative to the bulk flow
(i.e. $\bm{v}=\bm{0}$ in equilibrium),  
subscript $0$ indicates evaluation at $r=r_0$. In the energy equation
the ratio of specific heats $\gamma$ is assumed constant, 
and we have included a thermal relaxation term with timescale $t_c$.  

\subsection{Linearization}
We perturb the system such that
\begin{align}
  \rho \to \rho(z) + \delta\rho(z)\exp{\left(\ii k_x x - \ii\sigma
      t\right)},    
\end{align}
and similarly for other fluid variables, where $k_x$ is a horizontal
wavenumber and $\sigma = \omega + \ii \nu$ is a (generally) complex
frequency, with $\omega$ amd $\nu$ being the real frequency and growth
rate, respectively. For clarity we drop the subscript $0$ on the
frequencies and shear rates below. Then writing $\delta P /\rho \equiv W$
and $c_s^2\delta\rho/\rho\equiv Q$, the linearized system of equations
is   
 \begin{align}
   -&\ii\sigma \frac{Q}{c_s^2} + \ii k_x \delta v_x + \frac{d\dd
     v_z}{dz} + \frac{d\ln{\rho}}{dz}\delta v_z = 0,\label{lin_mass}\\
   &\ii\sigma \delta v_x + 2\Omega\delta v_y = \ii k_x W,\label{lin_vx}\\
   -&\frac{\kappa^2}{2\Omega}\delta v_x + \ii \sigma\delta v_y =
   r\frac{d\Omega}{dz}\delta v_z, \label{lin_vy}\\
   & \ii\sigma\delta v_z = \frac{dW}{dz} +
   \frac{d\ln{\rho}}{dz}\left(W-Q\right), \label{lin_vz}\\
   -&\ii\sigma W + \frac{1}{t_c}\left(W-\frac{Q}{\Gamma}\right) = -\ii\sigma\frac{\gamma}{\Gamma} Q +
   c_s^2\frac{d\ln{\rho}}{dz}\left(\frac{\gamma}{\Gamma}-1\right)\delta v_z.\label{lin_energy}
 \end{align}
It is understood that the quantities $\rho,\,c_s^2,\,\Omega,\kappa$
are evaluated at $r=r_0$ using their global expressions above.  

We can eliminate the perturbed velocities between the linearized
equations to obtain a pair or ordinary differential equations
for $W,Q$ as
\begin{align}
  &\frac{\sigma^2}{c_s^2}Q + \frac{\sigma^2k_x^2}{D}W \notag\\ 
  &=\left(\frac{\ii
    k_x r}{D}\frac{d\Omega^2}{dz} -
  \frac{d\ln{\rho}}{dz}\right)\left[\frac{dW}{dz} +
  \frac{d\ln{\rho}}{dz}\left(W-Q\right)\right] \notag\\
&- \frac{d^2W}{dz^2} - \frac{d\ln{\rho}}{dz}\left(\frac{dW}{dz} -
  \frac{dQ}{dz}\right) - \frac{d^2\ln{\rho}}{dz^2}\left(W-Q\right),\label{ode_w}\\
&\sigma^2W - \frac{\gamma}{\Gamma}\sigma^2 Q +
\frac{\ii\sigma}{t_c}\left(W-\frac{Q}{\Gamma}\right)\notag\\
&=c_s^2\frac{d\ln{\rho}}{dz}\left(\frac{\gamma}{\Gamma} - 1\right) 
\left[\frac{dW}{dz} + \frac{d\ln{\rho}}{dz}\left(W-Q\right)\right],\label{ode_Q} 
\end{align}
 % It is convenient to express $\delta v_x$ as a function of $W$ and
 % $\delta v_z$,
 % \begin{align}
 %   \delta v_x = -\frac{1}{D}\left(\sigma k_x W +
 %     r\frac{d\Omega^2}{dz}\delta v_z\right),
 % \end{align}
where
\begin{align}
  D \equiv \kappa^2 - \sigma^2.
\end{align}
Eq. \ref{ode_w}---\ref{ode_Q} are convenient for numerical work, 
but alternative forms are more suitable for analytical discussion, and 
will be derived later.  

\subsection{Boundary conditions}
In numerical calculations a finite vertical domain $z\in[-1,1]\zmax$
is adopted. For $\Gamma\neq1$ we take $\zmax$ to be slightly less than
$H$ to avoid the zero-density surface. At $z=\pm\zmax$ we impose
\begin{align}
  \delta v_z = 0,
\end{align}
for a solid boundary, or 
\begin{align}
  \Delta P \equiv \delta P + \bm{\xi}\cdot\nabla P= 0,  
\end{align}
for a free surface, where $\bm{\xi}$ is the Lagrangian
displacement.

%These boundary conditions are also numerically
%convenient because they can be recast to contain the eigenfrequency
%$\sigma$. This means that the the full set of ODEs, including boundary
%conditions can be written in the form $L(\bm{U}) = \sigma R (\bm{U})$  
