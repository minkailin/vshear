\section{Governing equations and disk models}\label{setup}
An inviscid, non-self-gravitating disk orbiting a central star of mass $M_*$
obeys the three-dimensional fluid equations:
\begin{align}
  &\frac{\p\rho}{\p t} + \nabla\cdot{\left(\rho\bm{v}\right)} = 0,\label{full_mass}\\
  &\frac{\p\bm{v}}{\p t} + \bm{v}\cdot\nabla\bm{v} =
  -\frac{1}{\rho}\nabla P - \nabla\Phi_*,\\
  &\frac{\p P}{\p t} + \bm{v}\cdot\nabla P  = - \gamma P
  \nabla\cdot\bm{v} - \Lambda % \frac{\rho}{t_c}\left(\frac{P}{\rho} -
    % \left.\frac{P}{\rho}\right|_{t=0}\right)
  , \label{full_energy}
\end{align}
where $\rho$ is the mass density, $\bm{v}$ is the velocity field (the
rotation frequency being $\Omega=v_\phi/r$), $P$
is the pressure, $\gamma$ is the constant adiabatic index, and $\Phi_*
= -GM_*(r^2 + z^2)^{-1/2}$ is the gravitational potential of the
central star with $G$ as the gravitational constant. The cylindrical
co-ordinates $(r,\phi, z)$ are centered on the star. 
In the energy equation (Eq.\ \ref{full_energy}) the sink term
$\Lambda$ includes non-adiabatic cooling and (negative) heating. 
The gas temperature $T$ obeys the ideal gas 
equation of state,  
\begin{align}
P = \frac{\mathcal{R}}{\mu}\rho T,
\end{align}
where $\mathcal{R}$ is the gas constant and $\mu$ is the mean molecular
weight.    

\subsection{Thermal relaxation}
We model radiative effects as thermal
relaxation with a timescale $t_c$, i.e.,
\begin{align}\label{thermal_relax}
  \Lambda  \equiv \frac{\rho\mathcal{R}}{\mu}\frac{T - T_\mathrm{eq}}{t_c},
\end{align}
where $T_\mathrm{eq}=T_\mathrm{eq}(r,z)$ is the equilibrium temperature.  We 
define the dimensionless cooling time 
\begin{align}\label{beta_def}
  \beta \equiv \OmK t_c
\end{align}
relative to the Keplerian frequency, $\OmK=\sqrt{GM_*/r^3}$.  We 
use the terms `thermal relaxation' and `cooling' interchangeably
throughout this paper.   

For most of this paper we will take $\beta$ as a constant input
parameter, so that we can vary the thermodynamic response of the
disk from isothermal ($\beta\ll 1$) to adiabatic ($\beta \gg 1$) in a
controlled manner. However, in \S\ref{application} 
we will consider a more realistic model for $\beta$ in
PPDs with a dust opacity (as in Fig.\  \ref{bcrit_mmsn2d}).  
 
% using Eq.\(\ref{beta_mmsn_simp}), which   
% Fig. \ref{bcrit_mmsn2d} shows that in this model $\beta >> 1$ in the 
% dense inner disk and near the midplane, but that rapid cooling ($\beta \ll 1$)
%is possible elsewhere.  Furthermore short wavelength perturbations can cool 
%quicker ...
 
% \emph{TODO: $k_x$ is in Fig. \ref{bcrit_mmsn2d} but not explained yet...}

% based on physical thermal
%timescales, in which case $\beta$ is non-constant and is determined by
%the equilibrium disk properties and the perturbation lengthscales of
%interest (e.g. Fig. \ref{bcrit_mmsn2d}). 



\subsection{Baroclinic disk equilibria}\label{eqm}
The equilibrium disk is steady and axisymmetric with density,
pressure and rotation profiles $\rho(r,z)$, $P(r,z)$ and
$\Omega(r,z)$, respectively. These functions satisfy  
\begin{subequations}\begin{align}
  0 &= \frac{1}{\rho}\frac{\p P}{\p z} + \frac{\p \Phi_*}{\p z},\label{vert_equilibrium}\\
  r\Omega^2&= \frac{1}{\rho}\frac{\p P}{\p r} + \frac{\p \Phi_*}{\p
    r}.\label{hori_equilibrium} 
\end{align} \end{subequations}
A unique solution requires additional assumptions about disk structure.  We choose
\begin{subequations}\begin{align}
 \rho(r,0) &\equiv \rho_0(r) = \rho_{00}\left(\frac{r}{r_0}\right)^p, \\
 T(r,0) &\equiv T_0(r) = T_{00}\left(\frac{r}{r_0}\right)^q, \\
 P(r,z) & = K(r)\rho^{\Gamma}(r,z). \label{eqm_press}
\end{align} \end{subequations}
where $r_0$ is a fiducial radius and $p$ and $q$ are the standard powerlaw indices
for midplane density and temperature, respectively.  The polytropic index $\Gamma$ 
parametrizes the vertical stratification with $\Gamma = 1$ ($\Gamma = \gamma$)
describing vertically isothermal (adiabatic) disks, respectively.  The ideal gas
law requires $K = \rho_0^{1-\Gamma}\mathcal{R} T_0/\mu$.

%These equations allows us to choose any midplane density 
%distribution $\rho_0(r)\equiv \rho(r,0)$. In this work we assume a
%power-law profile, 
%\begin{align}
%  \rho_0(r) = \rho_{00}\left(\frac{r}{r_0}\right)^p,
%\end{align}
%where $\rho_{00}$ is the midplane density at the fiducial radius
%$r=r_0$ and $p$ is the power-law index.  

%We consider disk equilibria in which the pressure $P$ and density
%$\rho$ are related by   
%\begin{align}
%  P(r,z) &= 
%  \frac{c_{s0}^2(r)}{\Gamma}\rho_0(r)^{1-\Gamma}\rho^\Gamma(r,z),\notag\\
%  & \equiv K(r)\rho^{\Gamma}(r,z). \label{eqm_press}
%\end{align}
%where the polytropic index $\Gamma$ is a  constant parameter  and the
%function $c_{s0}(r)$ is prescribed below. Choosing $\Gamma=1$ gives  a 
%locally vertically isothermal disk and $c_{s0}(r)$ is the mid-plane
%sound-speed. 
%It shall be convenient to define the function $c_s(r,z)$ 

We further define a modified sound speed
\begin{align}
	c_s \equiv \sqrt{\Gamma P /\rho } , 
% c_s^2\equiv % \frac{dP}{d\rho} =
%  c_{s0}^2(r)\left(\frac{\rho}{\rho_0}\right)^{\Gamma-1} =
%  \frac{\Gamma P}{\rho},  
\end{align}
which in general differs from the  isothermal, $c_\mathrm{iso} =
c_s/\sqrt{\Gamma}$, and adiabatic $c_\mathrm{ad} =
c_s\sqrt{\gamma/\Gamma}$, sound speeds. 
We also introduce the characteristic scale-height, $H = c_s(r,0)/\OmK$
and disk aspect-ratio
%We take $c_{s0}(r)$ to be another power-law:  
%\begin{align}
%  c_{s0}^2(r)=c_{s00}^2\left(\frac{r}{r_0}\right)^q, 
%\end{align}
%where $c_{s00}\equiv c_{s0}(r_0)$  and $q$ is the temperature power-law index, and
%define   
\begin{align}
   h(r) \equiv %\frac{c_{s0}(r)}{v_k(r)} \equiv
  \frac{H}{r} \propto r^{(1+q)/2}
\end{align}
We are interested in thin disks with $ h \ll 1$.


\subsubsection{Density structure}
The equilibrium density field follows from vertical hydrostatic equilibrium (Eq.\ \ref{vert_equilibrium}).
The solution for $\Gamma\neq1$ is
\begin{align}\label{eqm_dens}
  &\left(\frac{\rho}{\rho_0}\right)^{\Gamma-1} = 1 +
  \frac{\left(\Gamma-1\right)}{ h^2}\left(\frac{1}{\sqrt{1+z^2/r^2}}-1\right).
\end{align}
Note that for $\Gamma > 1 +  h^2$ there is a disk surface $H_s$
where $\rho(r,H_s)=0$; and $H_s \simeq \sqrt{2/(\Gamma-1)}H$ for $h\ll 1$.
%For $h \ll 1$, this surface is located at $H_s \simeq \sqrt{2/(\Gamma-1)}H$. 
% Note that for $\Gamma\neq1$ the disk has a surface $z=H(r)$ such that
% $\rho(r,H)=0$. If we parametrize
% \begin{align}
%   H(r)\equiv h r,
% \end{align}
% then the imposed sound-speed and disk thickness are related by 
% \begin{align}
%   h^2(r) = \left[1-\frac{ h^2(r)}{\Gamma-1}\right]^{-2}-1. 
% \end{align}

The vertically isothermal case, $\Gamma = 1$, can be calculated either 
as a special case or by taking the limit of Eq. \ref{eqm_dens},    
\begin{subequations}\begin{align}
\lim_{\Gamma\to1}  \frac{\rho(r,z)}{\rho_0(r)} &=
  \exp{\left[\frac{1}{ h^2}\left(\frac{1}{\sqrt{1+z^2/r^2}}-1\right)\right]},   \\
  & \simeq \exp {\left( - {z^2 \over 2 H^2}\right)} \quad\text{for } |z|\ll r \label{rhoisothin}. 
  \end{align}\end{subequations}
Eq.\ \ref{rhoisothin} is the approximate form of the density field in the thin-disk limit.  
We will primarily focus on 
vertically isothermal disks, the relevant case for passively irradiated
PPDs \citep{chiang97}. 

 
 \subsubsection{Rotation and vertical shear profiles}\label{vshear_def}
The equilibrium rotation field, $\Omega(r,z)$ follows from the density field and centrifugal balance (Eq.\ \ref{hori_equilibrium}),
giving 
\begin{align}\label{eqm_rot}
  \frac{\Omega^2(r,z)}{\OmK^2(r)}=1 +
  \frac{p+q}{\Gamma} h^2(r) 
  +\frac{s}{\Gamma} \left(1-\frac{1}{\sqrt{1+z^2/r^2}}\right), 
\end{align}
for all $\Gamma$, where $s\equiv q+p(1-\Gamma)$.  We also refer to the
epicyclic frequency $\kappa$ defined via
\begin{align}\label{kap2_def}
\kappa^2 \equiv \frac{1}{r^3}\frac{\p}{\p r}\left(r^4\Omega^2\right) = \frac{1}{r^3}\frac{\p j^2}{\p r}, 
\end{align}
where $j$ is the specific angular momentum. 

The vertical shear rate follows from Eq. \ref{eqm_rot}, 
\begin{align}
  r\frac{\p\Omega^2}{\p z} = \frac{sz}{\Gamma r}  \frac{\OmK^2}{
    \left(1+z^2/r^2\right)^{3/2}} . \label{vertical_shear_ex} 
\end{align}
For $\Gamma = 1$, notice that $s/\Gamma = q$ is the only disk parameter that affects vertical shear.
Vertical shear increases linearly with height near the midplane, and $|\p_z \Omega^2|$ is maximized at
$\mathrm{min}(r/\sqrt{2},H_s)$, which is expected to limit the maximum VSI growth rate. 
%the smaller of $r/\sqrt{2}$ or $H_s$. 

A more general expression for vertical shear holds for vertical polytropes 
(which satisfy Eq.\ \ref{eqm_press}), with no assumed radial structure:
\begin{align}
  r\frac{\p\Omega^2}{\p z} = - \rho^{\Gamma-1}\frac{\p\ln\rho}{\p
    z}\frac{dK}{dr} = - \frac{g_z}{\Gamma}\frac{d\ln K}{dr}, \label{vertical_shear}
\end{align}
where $g_z = \p_z P/\rho$.

%\subsubsection{Vertically isothermal disks}
%For $\Gamma=1$ the equilibrium density and rotation profiles are
%given, respectivley, by 
%\begin{align}
%  \frac{\rho(r,z)}{\rho_0(r)} &=
%  \exp{\left[\frac{1}{ h^2}\left(\frac{1}{\sqrt{1+z^2/r^2}}-1\right)\right]},\\    
%  \frac{\Omega^2(r,z)}{\OmK^2(r)}& =1+ (p+q) h^2(r) + q\left(1 -
%    \frac{1}{\sqrt{1+z^2/r^2}}\right),\label{vertiso_eqm}
%\end{align}
%which may be obtained from Eq. \ref{eqm_dens} and Eq. \ref{eqm_rot} by
%taking the limit $\Gamma\to 1$. We will, in fact, primarily focus on
%vertically isothermal disks which is the relevant case for passively irradiated
%PPDs \citep{chiang97}. 


%\subsection{Vertical shear}\label{vshear_def}
%Note that the disk is in general baroclinic with $\Omega =
%\Omega(r,z)$. This is explicit from Eq. \ref{eqm_rot} and Eq. \ref{vertiso_eqm}, but is also 
%implied by Eq. \ref{vert_equilibrium}---\ref{hori_equilibrium} and 
%Eq. \ref{eqm_press},  
%\begin{align}
%  r\frac{\p\Omega^2}{\p z} = - \rho^{\Gamma-1}\frac{\p\ln\rho}{\p
%    z}\frac{dK}{dr} = - \frac{g_z}{\Gamma}\frac{d\ln K}{dr}.\label{vertical_shear}
%\end{align}
%For the above disk models,
%\begin{align}
%  r\frac{\p\Omega^2}{\p z} = \frac{sz\OmK^2}{\Gamma
%    r\left(1+z^2/r^2\right)^{3/2}}. \label{vertical_shear_ex} 
%\end{align}
%Note that at fixed $r$ there is a maximum vertical shear rate at $|z|=r/\sqrt{2}$,
%which is expected to limit the growth rate of instabilities associated with vertical shear. 

\subsection{Entropy gradients and vertical buoyancy}
The gradients of specific entropy, $S\equiv
C_P\ln{P^{1/\gamma}/\rho}$, in our disk models are
\begin{subequations}
\begin{align}
  &\frac{\p S}{\p r} = C_P\left[\frac{s}{\gamma r} +
    \left(\frac{\Gamma}{\gamma}-1\right)\frac{\p \ln{\rho}}{\p
       r}\right], \label{dSdr}\\
  &\frac{\p S}{\p z} = C_P
  \left(\frac{\Gamma}{\gamma}-1\right)\frac{\p \ln{\rho}}{\p z} = {C_P g_z \over c_s^2} {\Gamma - \gamma \over \gamma},\label{dSdz}
\end{align} 
\end{subequations}
 where $C_P$ is the heat capacity at
constant pressure. 

The vertical buoyancy frequency $N_z$ is 
\begin{align}
  N_z^2 \equiv - C_P^{-1}g_z \frac{\p S}{\p z} = {g_z^2 \over c_s^2}{\gamma - \Gamma \over \gamma}. %c_s^2 \left(\frac{\gamma -
   %   \Gamma}{\gamma}\right)\left(\frac{\p\ln\rho}{\p z}\right)^2,  
    \label{nzsq_def}
\end{align}
We only consider convectively stable disks with with $\Gamma <
\gamma$ and  $N_z^2\geq0$. % , and the disk is 
% convectively stable in the vertical direction. 

\subsection{Stability without cooling}\label{solberg}
In the absence of cooling, axisymmetric stability is ensured
if both Solberg-Hoiland criteria  are satisfied:
\begin{subequations}\begin{align}
  \kappa^2 - \frac{1}{\rho C_P}\nabla P \cdot \nabla S &> 0,\\
  -\frac{\p P}{\p z}\left(\frac{\p j^2}{\p r}\frac{\p S}{ \p z} -
    \frac{\p j^2}{\p z}\frac{\p S}{\p r} \right)&>0 \label{second_SH} 
\end{align}\end{subequations}
\citep{tassoul78}. %, where $j=r^2\Omega$ is the specific angular
%momentum. %and $\kappa = r^{-3} \p_r j^2 $ is the epicyclic frequency.  
For rotationally-supported, thin disks %  pressure gradients are small 
% compared to rotation, so 
the first criterion is easily satisfied. 
Thus, we consider the second
criterion. 
Without loss of generality (since Eq. \ref{second_SH} is even in $z$), we consider $z>0$
so that $\p_zP<0$. With $\p_r j^2\simeq r^3\OmK^2$ and
using Eq. \ref{vertical_shear} for $\p_zj^2$  we find that
\begin{align}\label{solberg2}
  % \gamma - \Gamma > \frac{s h^2}{\Gamma}
  % \left(\frac{\rho}{\rho_0}\right)^{\Gamma-1} \left[s
  %   -\left(\gamma-\Gamma\right)\frac{\p\ln{\rho}}{\p\ln{r}} \right]. 
  \gamma - \Gamma > \frac{ h^2}{\Gamma}
  \left(\frac{\rho}{\rho_0}\right)^{\Gamma-1} \left[s^2
    -s\left(\gamma-\Gamma\right)\frac{\p\ln{\rho}}{\p\ln{r}} \right]
\end{align} 
implies stability. For typical model parameters, the right hand
side (RHS) is $O( h^2) \ll 1$.  The left hand side is positive and order unity ($O(0.1)$ at least) if the disk is convectively stable, and not very close to convective instability.  This stable stratification is expected for irradiated disks  \citep{chiang97}.
Thus adiabatic disturbances are stable to a disk's vertical shear, explaining why the VSI requires cooling. 

\section{Linear problem}\label{linear} 
This section presents the two sets of equations we use to study the
linear development of the VSI.  Both sets are radially local and vertically global 
with a finite cooling time.
The first set, presented in \S\ref{sec:radlocal}, is more general and used 
for precise numerical calculations.  The second set, in \S\ref{sec:simplified}, makes
additional approximations  about disk structure and wave frequency.  This simplified
set is used to obtain analytic results.


\subsection{Radially local approximation}\label{sec:radlocal}
We consider axisymmetric perturbations to the above disk equilibria.    
The growth of the VSI is strongest for short radial wavelengths, as compared
 to the disk radius \citepalias{nelson13,barker15}.  We thus perform a
 standard two step process to obtain linearized equations in the radially local 
 approximation (e.g., \citealp{goldreich67}, who also consider vertically localized perturbations).  
 
 We first expand all fluid variables 
 into the equilibrium value plus an Eulerian perturbation denoted by $\delta$, e.g.\ 
 $\delta \rho$ for the perturbed density, and drop all non-linear perturbations.  Second, we perform
 a Taylor expansion about a fiducial radius $r_0$ with the local radial coordinate
 $x = r - r_0$, keeping only the leading order terms in $x/r_0$.  We also relabel 
 (trivially for axisymmetric perturbations) the azimuthal direction as $y$.  
 
 Perturbations take the form of a radial plane wave with arbitrary vertical dependence, 
 e.g.\
 \begin{align}
  \delta\rho \rightarrow \delta\rho(z)\exp{\left(\ii k_x x - \ii\upsilon
      t\right)},    
\end{align}
where $\upsilon = \omega + \ii \sigma$ is a (generally) complex frequency, 
with $\omega$ and $\sigma$ being
the real frequency and growth rate, respectively, and $k_x$ is a real 
radial wavenumber.  We take $k_x>0$ without loss of generality, 
assume $k_xr_0 \gg 1$ and neglect curvature terms. 
Henceforth all unperturbed fluid variables, including 
gradients such as $\p_z \Omega$, refer to  equilibrium values at $(r_0, z)$.

We further define the perturbation variables  $W \equiv \delta P /\rho$ 
and $Q \equiv c_s^2\delta\rho/\rho$.  With this procedure the linearized
system of equations are
\begin{subequations}\label{lin_all}\begin{align}
  \ii\upsilon \frac{Q}{c_s^2}  &=  \ii k_x \delta v_x + \frac{d\dd
    v_z}{dz} + \frac{\p\ln{\rho}}{\p z}\delta v_z + \zeta
  \frac{\p\ln{\rho}}{\p r} \delta v_x,\label{lin_mass}\\
  \ii\upsilon \delta v_x  &= \ii k_x W - 2\Omega\delta v_y -
  \zeta\frac{1}{\rho}\frac{\p P}{\p r}\frac{Q}{c_s^2}\label{lin_vx}\\
   \ii \upsilon\delta v_y &= r_0\frac{\p \Omega}{\p z}\delta v_z +
  \frac{\kappa^2}{2\Omega}\delta v_x, \label{lin_vy}\\
   \ii\upsilon\delta v_z &= \frac{dW}{dz} +
  \frac{\p\ln{\rho}}{\p z}\left(W-Q\right), \label{lin_vz}\\
  \ii \upsilon W &= c_s^2\frac{\p\ln{\rho}}{\p z}\delta v_z +
  c_s^2\frac{\gamma}{\Gamma} \left(\ii k_x\delta v_x + \frac{d\delta
      v_z}{dz}\right) \notag\\
  &+ \frac{1}{t_c}\left(W - \frac{Q}{\Gamma}\right) +
  \zeta\frac{1}{\rho}\frac{\p P}{\p r}\delta v_x.\label{lin_energy}
  % -&\ii\upsilon W + \frac{1}{t_c}\left(W-\frac{Q}{\Gamma}\right) = -\ii\upsilon\frac{\gamma}{\Gamma} Q +
  % c_s^2\frac{d\ln{\rho}}{dz}\left(\frac{\gamma}{\Gamma}-1\right)\delta v_z.\label{lin_energy}
\end{align}\end{subequations}
This is a set of ordinary differential equations (ODEs) in
$z$.  %boundary conditions for well posed problem 
Solutions to these 
are presented in \S\ref{numerical} and \S\ref{application}.   The coefficient $ \zeta = 1$ is 
introduced simply to label terms containing radial gradients of the equilibrium state, 
which again are evaluated at $(r_0, z)$. For clarity, hereafter we drop the subcript $0$.    
We will consider the effects of ignoring these radial gradients below. 


%We consider axisymmetric perturbations to the above equilibria in the
%form 
%\begin{align}
%  \rho \to \rho(r, z) + \delta\rho(r,z)\exp{\left(\ii k_rr - \ii\upsilon
%      t\right)},    
%\end{align}
%and similarly for other fluid variables, where $\upsilon = \omega + \ii
%\sigma$ is a (generally) complex frequency, with $\omega$ and $\sigma$ being
%the real frequency and growth rate, respectively, and $k_r$ is a real 
%radial wavenumber. We take $k_r>0$ without loss of generality. 

%The VSI is expected to operate for perturbations with small radial
%wavelengths \citepalias{nelson13,barker15}. We thus linearize 
%Eq. \ref{full_mass}---\ref{full_energy} assuming perturbations 
%in the above form with radially slowly-varying
%amplitudes and $k_r r \gg 1$, so that the radial dependence of $\delta\rho(r,z)$, etc.,
%may be neglected, as well as curvature terms. Henceforth we are
%considering solutions at some fiducial radius $r=r_0$.  
%\citetalias{barker15} have validated this 
% approximation for the VSI by comparison with the linear problem in
% global cylindrical disk geometry. 

%In order to emphasize the radially local approximation, we introduce
%the local Cartesian co-ordinates $(x,y,z)$ corresponding to the the
%local $(r,\phi,z)$ directions, respectively.  
%We thus linearize 
%and replace  
% The linearized fluid equations in cylindrical co-ordinates are  
% \begin{align}
%   \ii\upsilon\delta\rho &= \frac{\rho}{r}\frac{\p}{\p r}\left(r\delta v_r\right) + \frac{\p}{\p
%     z}\left(\rho\delta v_z\right)  + \zeta\frac{\p\rho}{\p r}\delta v_r,\label{full_mass1}\\
%   \ii\upsilon\delta v_r &= \frac{1}{\rho}\frac{\p\delta P}{\p r} - 2\Omega
%   \delta v_\phi - \zeta\frac{\delta\rho}{\rho^2}\frac{\p P}{\p r},\\ 
%   \ii\upsilon\delta v_\phi  &= \frac{\kappa^2}{2\Omega}\delta v_r +
%   r\frac{\p\Omega}{\p z}\delta v_z,\\
%   \ii \upsilon\delta v_z &= \frac{1}{\rho}\frac{\p\delta P}{\p z} -
%   \frac{\delta \rho}{\rho^2}\frac{\p P}{\p z},\\
%   \ii\upsilon\delta P &=\frac{\p P}{\p z}\delta v_z + \gamma P
%   \left[\frac{1}{r}\frac{\p}{\p r}\left(r\delta v_r\right) + \frac{\p
%       \delta v_z}{\p z}\right]\notag \\
%   &+ \frac{1}{t_c}\left(\delta P -  
%     \frac{P}{\rho}\delta\rho\right) + \zeta \frac{\p P}{\p r}\delta v_r,\label{full_energy1}
% \end{align}
% where the coefficients $\zeta=1$ are inserted to label terms
% associated with the background radial disk structure. 
% This approach has been adopted
% in recent linear calculations of the VSI
% \cite{nelson13,mcnally14,barker15} because unstable modes have short
% radial lengthscales. 
%We consider perturbations with small radial length-scales 
%so that we can ignore the radial variations in the background
%structure and radial gradients. 
%We thus evaluate all coefficients in
%Eq. \ref{full_mass1}---\ref{full_energy1} at the fiducial radius
%$r=r_0$. However, the full vertical dependence of the coefficients are
%retained. % Then we can assume perturbations have spatial dependence in
% the form 
% \begin{align}
%   \delta \rho(r,z) \to \delta\rho(z)\exp{\ii k_r r}.
% \end{align}
% We take $k_r>0$ without loss of generality, and further assume $k_rr\gg1$ so that curvature terms can also be 
% ignored.       
%Then writing $\delta P /\rho \equiv W$, $c_s^2\delta\rho/\rho\equiv
%Q$ and making the notation change $k_r\to k_x$,  the
%linearized system of equations are 
%\begin{align}
%  \ii\upsilon \frac{Q}{c_s^2}  &=  \ii k_x \delta v_x + \frac{d\dd
%    v_z}{dz} + \frac{\p\ln{\rho}}{\p z}\delta v_z + \zeta
%  \frac{\p\ln{\rho}}{\p r} \delta v_x,\label{lin_mass}\\
%  \ii\upsilon \delta v_x  &= \ii k_x W - 2\Omega\delta v_y -
%  \zeta\frac{1}{\rho}\frac{\p P}{\p r}\frac{Q}{c_s^2}\label{lin_vx}\\
%   \ii \upsilon\delta v_y &= r\frac{\p \Omega}{\p z}\delta v_z +
%  \frac{\kappa^2}{2\Omega}\delta v_x, \label{lin_vy}\\
%   \ii\upsilon\delta v_z &= \frac{dW}{dz} +
%  \frac{\p\ln{\rho}}{\p z}\left(W-Q\right), \label{lin_vz}\\
%  \ii \upsilon W &= c_s^2\frac{\p\ln{\rho}}{\p z}\delta v_z +
%  c_s^2\frac{\gamma}{\Gamma} \left(\ii k_x\delta v_x + \frac{d\delta
%      v_z}{dz}\right) \notag\\
%  &+ \frac{1}{t_c}\left(W - \frac{Q}{\Gamma}\right) +
%  \zeta\frac{1}{\rho}\frac{\p P}{\p r}\delta v_x,\label{lin_energy}
%  % -&\ii\upsilon W + \frac{1}{t_c}\left(W-\frac{Q}{\Gamma}\right) = -\ii\upsilon\frac{\gamma}{\Gamma} Q +
%  % c_s^2\frac{d\ln{\rho}}{dz}\left(\frac{\gamma}{\Gamma}-1\right)\delta v_z.\label{lin_energy}
%\end{align}
%where the coefficients $\zeta=1$ are inserted to label terms
%associated with the background radial disk structure. 
%It is understood that the quantities $\rho,\,c_s^2,\,\Omega,\kappa,\, 
%\p_z\Omega,\, \p_r\ln{\rho} $ and $ \p_r P$ are evaluated at $r=r_0$
%using their expressions in the global disk given in \S\ref{setup}. Thus, they are functions of $z$ only. 
%Eq. \ref{lin_mass}---\ref{lin_energy} will form the basis of our numerical
%study, but for analytical discussion
%we will make further simplifications, as discussed next.
