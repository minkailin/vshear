\section{Baroclinic disk equilibra}
We consider  an inviscid, non-self-gravitating three-dimensional (3D)
disk orbiting a central star of mass $M_*$. 
We adopt cylindrical
co-ordinates $(r,\phi, z)$ centered on the star to describe the global
disk. The equilibrium disk is steady and axisymmetric with density,
pressure and rotation profiles $\rho(r,z)$, $P(r,z)$ and
$\Omega(r,z)$, respectively. The equilibrium disk satisfies
\begin{align}
  0 &= \frac{1}{\rho}\frac{\p P}{\p z} + \frac{\p \Phi_*}{\p z},\label{vert_equilibrium}\\
  r\Omega^2&= \frac{1}{\rho}\frac{\p P}{\p r} + \frac{\p \Phi_*}{\p
    r},\label{hori_equilibrium} 
\end{align} 
where $\Phi_* = -GM_*(r^2 + z^2)^{-1/2}$ is the gravitational
potential of the central star.

We examine disks with power-law profiles for the midplane density
$\rho_0(r)\equiv\rho(r,0)$: 
\begin{align}
  \rho_0(r) = \rho_{00}\left(\frac{r}{r_0}\right)^p,
\end{align}
where $\rho_{00}$ is the midplane density at the fiducial radius
$r=r_0$ and $p$ is the power-law index. The equilibrium pressure $P$
and density $\rho$ are related by 
\begin{align}
  P(r,z) &= 
  \frac{c_{s0}^2(r)}{\Gamma}\rho_0(r)^{1-\Gamma}\rho^\Gamma(r,z),\notag\\
  & \equiv K(r)\rho^{\Gamma}(r,z). \label{eqm_press}
\end{align}
where the polytropic index $\Gamma$ is a parameter and 
the midplane sound-speed $c_{s0}(r)$ is taken to be another power-law: 
\begin{align}
  c_{s0}^2(r)=c_{s00}^2\left(\frac{r}{r_0}\right)^q, 
\end{align}
where $c_{s00}\equiv c_{s0}(r_0)$  and $q$ is the power-law index for
the midplane temperature. We define
\begin{align}
  \epsilon(r) \equiv \frac{c_{s0}(r)}{v_k(r)} \equiv
  \frac{H_\mathrm{iso}}{r} 
\end{align}
to parametrize the magnitude of the sound-speed, where
$v_k=r\Omega_k=\sqrt{GM_*/r}$ is the Keplerian azimuthal velocity and
$G$ is the gravitational constant, and $H_\mathrm{iso}$ is a
characteristic scale-height.  Thus, $\epsilon$ is also the
characteristic disk aspect-ratio.  

The global sound-speed $c_s(r,z)$ is given by
\begin{align}
  c_s^2\equiv \frac{dP}{d\rho} =
  c_{s0}^2(r)\left(\frac{\rho}{\rho_0}\right)^{\Gamma-1} = \frac{\Gamma P}{\rho}. 
\end{align}
For discussion purposes it is useful to define the vertical buoyancy frequency $N_z$ via
\begin{align}
  N_z^2 = c_s^2 \left(\frac{\gamma -
      \Gamma}{\gamma}\right)\left(\frac{\p\ln\rho}{\p z}\right)^2,   
\end{align}
where $\gamma$ is the constant adiabatic index. We are interested in
neutrally or sub-adiabatically stratified 
disks with $\Gamma\leq \gamma$ so that $N_z^2\geq0$.  

The explicit expression for the density field for
$\Gamma\neq1$ is given  by
\begin{align}\label{eqm_dens}
  &\left(\frac{\rho}{\rho_0}\right)^{\Gamma-1} = 1 +
  \frac{\left(\Gamma-1\right)}{\epsilon^2}\left(\frac{1}{\sqrt{1+z^2/r^2}}-1\right),
\end{align}
Note that for $\Gamma\neq1$ the disk has a surface $z=H(r)$ such that
$\rho(r,H)=0$. If we parametrize
\begin{align}
  H(r)\equiv h r,
\end{align}
then the imposed sound-speed and disk thickness are related by 
\begin{align}
  h^2(r) = \left[1-\frac{\epsilon^2(r)}{\Gamma-1}\right]^{-2}-1. 
\end{align}
Similarly, the equilibrium rotation is 
\begin{align}\label{eqm_rot}
  \frac{\Omega^2(r,z)}{\Omega_k^2(r)}=1 +
  \left(p+\frac{s}{\Gamma}\right)\epsilon^2(r) 
  +\frac{s}{\Gamma} \left(1-\frac{1}{\sqrt{1+z^2/r^2}}\right), 
\end{align}
where $s\equiv q+p(1-\Gamma)$. We also define the radial shear rate  
\begin{align}
  S(r,z) \equiv r\frac{\p\Omega}{\p r},  
\end{align}
and the square of the epicyclic frequency $\kappa$ 
\begin{align}
  \kappa^2(r,z) \equiv 2\Omega(2\Omega + S). 
\end{align}

Note that the disk is in general baroclinic with $\Omega =
\Omega(r,z)$. This is explicit from Eq. \ref{eqm_rot}, but is also
implied by Eq. \ref{vert_equilibrium}---\ref{hori_equilibrium} and 
Eq. \ref{eqm_press},  
\begin{align}
  r\frac{\p\Omega^2}{\p z} = - \rho^{\Gamma-1}\frac{\p\ln\rho}{\p
    z}\frac{dK}{dr}.\label{vertical_shear}
\end{align}

\subsection{Vertically isothermal disks}
For the special case $\Gamma=1$ the disk is isothermal in the
vertical direction. The equilibrium density $\rho(r,z)$ and rotation
$\Omega(r,z)$ is then 
\begin{align}
  \frac{\rho(r,z)}{\rho_0(r)} &=
  \exp{\left[\frac{1}{\epsilon^2}\left(\frac{1}{\sqrt{1+z^2/r^2}}-1\right)\right]}\\    
  \frac{\Omega^2(r,z)}{\Omega_k^2(r)}& =1+ (p+q)\epsilon^2(r) + q\left(1 -
    \frac{1}{\sqrt{1+z^2/r^2}}\right),\label{vertiso_eqm}
\end{align}
which may be obtained from Eq. \ref{eqm_dens} and Eq. \ref{eqm_rot} by
taking the limit $\Gamma\to 1$. We will, in fact, primarily focus on
vertically isothermal disks for simplicitly. 


\subsection{Stability condition against axisymmetric adiabatic
  perturbations}\label{solberg}
A rotating flow is stable against axisymmetric adiabatic perturbations
if both Solberg-Hoiland criteria  are satisfied:
\begin{align}
  \kappa^2 - \frac{1}{\rho c_p}\nabla P \cdot \nabla S &> 0,\\
  -\frac{\p P}{\p z}\left(\frac{\p j^2}{\p r}\frac{\p S}{ \p z} -
    \frac{\p j^2}{\p z}\frac{\p S}{\p R} \right)&>0.  
\end{align}
\citep{tassoul78}, where $S = c_p\ln{P^{1/\gamma}/\rho}$ is the
specific entropy, $c_p$ is the heat capacity at constant
pressure (assumed to be constant), and $j\equiv r^2\Omega$ is the
specific angular momentum.   

For rotationally-supported, thin disks, pressure gradients are small 
compared to rotation, so the first criterion is generally
satisfied. Thus, to assess stability, we evaluate the second criterion
for the above disk models in $z>0$ where $\p_zP<0$, without loss of
generality,   
\begin{align}\label{solberg2}
  \gamma - \Gamma > \frac{s\epsilon^2}{\Gamma}
  \left(\frac{\rho}{\rho_0}\right)^{\Gamma-1} \left[s
    -\left(\gamma-\Gamma\right)\frac{\p\ln{P}}{\p\ln{r}} \right]. 
\end{align} 
For typical model parameters the right hand side is
$O(\epsilon^2)$. Hence, to have instability in the adiabatic limit 
requires $\gamma < \Gamma + O(\epsilon^2)$. Since we consider
$\epsilon\ll1$, this implies the equilibrium disk must have a small
entropy gradient in order to be unstable to axisymmetric adiabatic
perturbations. 

\section{The vertically global shearing box} 
We are interested in the axisymmetric stability of the above system,
on radial length-scales much smaller than the local radius $r$, but
may be global in the vertical direction. As such, we work within the
framework of the \emph{vertically global shearing 
  box} (VGSB) developed by \cite{mcnally14}. In this approximation
one sets up a local Cartesian co-ordinate  system $(x,y,z)$ centered
on a fiducial point $(r,\phi,z)=(r_0,\phi_0,0)$ in the global disk, with $\phi_0(t) =
\Omega(r_0,0)t$. The local $x,y,z$ axis correspond to the
global radial, azimuthal and vertical directions.  

In this shearing box the equilibrium density, pressure and
sound-speed are assumed to depend only on $z$, by evaluating these
quantities at $r=r_0$ in the global disk. The background rotation
$\Omega$ is approximated by a linear shear in $x$, but left unaltered
in $z$. However, the vertical shear $\p_z\Omega$ is also assumed to
only depend on $z$.     

The axisymmetric VGSB equations read 
\begin{align}
  & \frac{\p\rho}{\p t} + \frac{\p}{\p x}\left(\rho v_x\right) +
  \frac{\p}{\p z}\left(\rho v_z\right)=0,\\
  & \frac{\p\bm{v}}{\p t} + \bm{v}\cdot\nabla\bm{v} +
  v_zr_0\frac{d\Omega_0}{dz} \hat{\bm{y}} = - 
  2\Omega_0\hat{\bm{z}}\times\bm{v} - S_0v_x\hat{\bm{y}} \notag\\ 
  &\phantom{\frac{\p\bm{v}}{\p t} + \bm{v}\cdot\nabla\bm{v} +
    v_zr_0\frac{d\Omega_0}{dz} \hat{\bm{y}}=}-\frac{1}{\rho}\nabla P - \frac{\p\Phi_0}{\p z}\hat{\bm{z}},\\
  & \frac{\p P}{\p t} + \bm{v}\cdot\nabla P = -\gamma P
  \nabla\cdot\bm{v} -
  \frac{\rho}{t_c}\left(\frac{P}{\rho}-\left.\frac{P}{\rho}\right|_{t=0}\right), \label{energy_eq}
\end{align}
where $\bm{v}$ is the velocity field relative to the bulk flow
(i.e. $\bm{v}=\bm{0}$ in equilibrium),  
subscript $0$ indicates evaluation at $r=r_0$. In the energy equation 
we have added a thermal relaxation term with timescale
$t_c=\beta\Omega_k^{-1}$ with $\beta$  being a dimensionless input
parameter.  

\subsection{Linearization}
We perturb the system such that
\begin{align}
  \rho \to \rho(z) + \delta\rho(z)\exp{\left(\ii k_x x - \ii\sigma
      t\right)},    
\end{align}
and similarly for other fluid variables, where $k_x$ is a horizontal
wavenumber and $\sigma = \omega + \ii \nu$ is a (generally) complex
frequency, with $\omega$ amd $\nu$ being the real frequency and growth
rate, respectively. For clarity we drop the subscript $0$ on the
frequencies and shear rates below. Then writing $\delta P /\rho \equiv W$
and $c_s^2\delta\rho/\rho\equiv Q$, the linearized system of equations
is   
 \begin{align}
   -&\ii\sigma \frac{Q}{c_s^2} + \ii k_x \delta v_x + \frac{d\dd
     v_z}{dz} + \frac{d\ln{\rho}}{dz}\delta v_z = 0,\label{lin_mass}\\
   &\ii\sigma \delta v_x + 2\Omega\delta v_y = \ii k_x W,\label{lin_vx}\\
   -&\frac{\kappa^2}{2\Omega}\delta v_x + \ii \sigma\delta v_y =
   r\frac{d\Omega}{dz}\delta v_z, \label{lin_vy}\\
   & \ii\sigma\delta v_z = \frac{dW}{dz} +
   \frac{d\ln{\rho}}{dz}\left(W-Q\right), \label{lin_vz}\\
   -&\ii\sigma W + \frac{1}{t_c}\left(W-\frac{Q}{\Gamma}\right) = -\ii\sigma\frac{\gamma}{\Gamma} Q +
   c_s^2\frac{d\ln{\rho}}{dz}\left(\frac{\gamma}{\Gamma}-1\right)\delta v_z.\label{lin_energy}
 \end{align}
It is understood that the quantities $\rho,\,c_s^2,\,\Omega,\kappa$
are evaluated at $r=r_0$ using their global expressions above.  

We can eliminate the perturbed velocities between the linearized
equations to obtain a pair or ordinary differential equations
for $W,Q$ as
\begin{align}
  &\frac{\sigma^2}{c_s^2}Q + \frac{\sigma^2k_x^2}{D}W \notag\\ 
  &=\left(\frac{\ii
    k_x r}{D}\frac{d\Omega^2}{dz} -
  \frac{d\ln{\rho}}{dz}\right)\left[\frac{dW}{dz} +
  \frac{d\ln{\rho}}{dz}\left(W-Q\right)\right] \notag\\
&- \frac{d^2W}{dz^2} - \frac{d\ln{\rho}}{dz}\left(\frac{dW}{dz} -
  \frac{dQ}{dz}\right) - \frac{d^2\ln{\rho}}{dz^2}\left(W-Q\right),\label{ode_w}\\
&\sigma^2W - \frac{\gamma}{\Gamma}\sigma^2 Q +
\frac{\ii\sigma}{t_c}\left(W-\frac{Q}{\Gamma}\right)\notag\\
&=c_s^2\frac{d\ln{\rho}}{dz}\left(\frac{\gamma}{\Gamma} - 1\right) 
\left[\frac{dW}{dz} + \frac{d\ln{\rho}}{dz}\left(W-Q\right)\right],\label{ode_Q} 
\end{align}
where
\begin{align}
  D \equiv \kappa^2 - \sigma^2.
\end{align}

\subsection{Limitations}
Before using the above equations to study the VSI, it is very important to be
aware of several limitations of the VGSB model. The VGSB momentum  
equations have the same form as that for the standard shearing box 
\citep{goldreich65}, except for the appearence of the term 
$v_zr_0d\Omega_0/dz$. This term appears in the global stability
problem and is the only term associated with vertical shear that is 
ported from the global disk into the local model.   

Note that, within the VGSB framework, the vertical shear 
$d\Omega_0/dz$ is a free parameter (as are $\Omega_0$ and $S_0$). 
This is different to the global problem because in that case, the
vertical shear is a result of the disk radial structure, which is
ignored in the VGSB equilibria. 
In effect, we are studying dynamics in the standard shearing box
subject to an additional azimuthal force $-v_zr_0 d \Omega_0/dz$. This
force is inspired by the vertical shear in the global disk, and its
functional form is taken from global equilibrium. 

%horizontal structure that gives rises 
%vertical shear is an input force
%only porting vertical shear  
%vertical shear doesn't depend on x 
%not all results can be extrapolated to the global context 

In Appendix \ref{global_corr} we make the connection between the
linear problem in the VGSB to that in the global disk model. We
emphasize that not all calculations in the local model can be 
consistently interpreted in the global context. However, for
perturbations with small radial lengthscales (large $|k_x|$) and fast
thermal relaxation ($\beta\ll1$), which characterizes the VSI, the
VGSB framework is adequate.    
