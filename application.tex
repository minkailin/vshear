\section{Application to protoplanetary disks}
As an application of our results, we estimate where in a
protoplanetary disk do the thermodynamic conditions allow the VSI to 
operate. Specifically, we consider the fundamental isothermal VSI, 
for which we can use the simple criterion Eq. \ref{iso_vsi_cond}. The
condition derived below may be considered as a sufficient condition for the VSI.  

In our disk models we have adopted a simple thermal relaxation
function in the energy equation. To connect our model with a realistic
disk, we consider the more physically-motivated form of the energy
equation with radiative cooling 
\begin{align}\label{real_energy}
\rho T \frac{DS}{Dt} = - \left(\nabla\cdot\bm{F} - Q_+\right), 
\end{align}
where the temperature $T$ is defined such that $P=\mathcal{R}\rho
T/\mu$, where $\mathcal{R}$ is the ideal gas constant and $\mu$ is
the mean molecular weight. In Eq. \ref{real_energy} $S$ is the
specific entropy,
\begin{align}
  S \equiv c_v\ln{\left(\frac{p}{\rho^{\gamma}}\right)} 
\end{align}
where $c_v \equiv \left(\gamma-1\right)^{-1}\mathcal{R}/\mu$ is the specific heat capcity at
constant volume, $D/Dt$ is the convective derivative, $\bm{F}$ is the
radiative flux, and $Q_+$ is a heating term such that an equilibrium
solution exists.  

\subsection{Newtonian cooling}
Our aim is to extract from Eq. \ref{real_energy} the dimensionless
cooling time $\beta$ as used in our linear calculations. To do so, we
consider perturbations with small radial lengthscales $L\equiv
2\pi/k_x<l_\mathrm{rad}$ where 
\begin{align}
l_\mathrm{rad} \equiv \frac{1}{\kappa_d\rho} 
\end{align} 
is the photon mean-free-path and $\kappa_d$ is the (dust) opacity. For
$L< l_\mathrm{rad}$ the radiative cooling is given by {\bf (reference?)}
\begin{align}
\nabla\cdot\bm{F} = 4 \rho \kappa_d \sigma_s T^4,
\end{align}   
where $\sigma_s$ is the Stefan-Boltzmann constant. We choose the
heating term 
\begin{align}
  Q_+ = 4\rho\kappa_d\sigma_s T^{4}_{t=0}. 
\end{align}
Then Eq. \ref{real_energy} becomes
\begin{align}
  \frac{DP}{Dt} = -\gamma P \nabla\cdot\bm{v} -
  \frac{4P\kappa_d\sigma_s}{Tc_v}\left(T^4 - T_{t=0}^4\right). 
\end{align}
When linearized, the last term on the right hand side becomes
\begin{align}
  -\frac{16\sigma_s}{c_v}\kappa_d T^3\left(\delta P -
    \frac{P}{\rho}\delta\rho\right). 
\end{align}
By comparing this expression with the linearized version of the energy
equation used in our calculations (Eq. \ref{energy_eq}), we identify 
\begin{align}
  t_c = \frac{c_v}{16\sigma_s\kappa_dT^3}.
\end{align}
For the fundamental isothermal VSI to operate, we need $\beta =
t_c\Omega_k< \beta_\mathrm{crit}$ as given by
Eq. \ref{iso_vsi_cond}. This translates to
\begin{align} 
  \frac{\mathcal{R}\Omega_k}{16\mu\sigma_s\kappa_dT^3} < \epsilon|q|. 
\end{align}




the Minimum Mass Solar Nebula (MMSN) described in \cite{chiang10}

