\section{Application to protoplanetary disks}\label{application} 

\subsection{Numerical example}






\subsection{Connection with radiative cooling}
As an application of our results, we estimate where in a 
protoplanetary disk do the thermodynamic conditions allow the VSI to 
operate. Specifically, we consider the fundamental isothermal VSI, 
for which we can use the simple criterion Eq. \ref{iso_vsi_cond}. The
condition derived below may be considered as a sufficient condition for the VSI.  

In our disk models we have adopted a simple thermal relaxation
function in the energy equation. To connect our model with a realistic
disk, we consider the more physically-motivated form of the energy
equation with radiative cooling 
\begin{align}\label{real_energy}
\rho T \frac{DS}{Dt} = - \left(\nabla\cdot\bm{F} - Q_+\right), 
\end{align}
where the temperature $T$ is defined such that $P=\mathcal{R}\rho
T/\mu$, where $\mathcal{R}$ is the ideal gas constant and $\mu$ is
the mean molecular weight. In Eq. \ref{real_energy} $S$ is the
specific entropy,
\begin{align}
  S \equiv c_v\ln{\left(\frac{p}{\rho^{\gamma}}\right)} 
\end{align}
where $c_v \equiv \left(\gamma-1\right)^{-1}\mathcal{R}/\mu$ is the specific heat capcity at
constant volume, $D/Dt$ is the convective derivative, $\bm{F}$ is the
radiative flux, and $Q_+$ is a heating term such that an equilibrium
solution exists.  

\subsubsection{Newtonian cooling}
Our aim is to extract from Eq. \ref{real_energy} the dimensionless
cooling time $\beta$ as used in our linear calculations. To do so, we
consider perturbations with small radial lengthscales $L\equiv
2\pi/k_x<l_\mathrm{rad}$ where 
\begin{align}
l_\mathrm{rad} \equiv \frac{1}{\kappa_d\rho} 
\end{align} 
is the photon mean-free-path and $\kappa_d$ is the (dust) opacity. For
$L< l_\mathrm{rad}$ the radiative cooling is given by {\bf (reference?)}
\begin{align}
\nabla\cdot\bm{F} = 4 \rho \kappa_d \sigma_s T^4,
\end{align}   
where $\sigma_s$ is the Stefan-Boltzmann constant. We choose the
heating term 
\begin{align}
  Q_+ = 4\rho\kappa_d\sigma_s T^{4}_{t=0}. 
\end{align}
Then Eq. \ref{real_energy} becomes
\begin{align}
  \frac{DP}{Dt} = -\gamma P \nabla\cdot\bm{v} -
  \frac{4P\kappa_d\sigma_s}{Tc_v}\left(T^4 - T_{t=0}^4\right). 
\end{align}
When linearized, the last term on the right hand side becomes
\begin{align}
  -\frac{16\sigma_s}{c_v}\kappa_d T^3\left(\delta P -
    \frac{P}{\rho}\delta\rho\right). 
\end{align}
By comparing this expression with the linearized version of the energy
equation used in our calculations (Eq. \ref{energy_eq}), we identify 
\begin{align}
  t_c = \frac{c_v}{16\sigma_s\kappa_dT^3}.
\end{align}
For the fundamental isothermal VSI to operate, we need $\beta =
t_c\Omega_k< \beta_\mathrm{crit}$ as given by
Eq. \ref{iso_vsi_cond}. This translates to
\begin{align}\label{newton_vsi_app} 
  \frac{\mathcal{R}\Omega_k}{16\mu\sigma_s\kappa_dT^3} < \epsilon|q|. 
\end{align}


We evaluate Eq. \ref{newton_vsi_app} for the Minimum Mass Solar Nebula
(MMSN) described in \cite{chiang10} and summarized in Appendix
\ref{mmsn} to find
\begin{align}
  \frac{r}{\mathrm{AU}}\lesssim 14.24\left(\mu\hat{\kappa}_d
  \right)^{14/5}\hat{T}^{14}.
\end{align}
For the MMSN with $\mu=2$ this is
\begin{align*}
r\lesssim 100\mathrm{AU}. 
\end{align*}

On the other hand, we have assumed the perturbation lengthscale
$L<l_\mathrm{rad}$, or
\begin{align} 
  \frac{2\pi}{\khat} < \frac{l_\mathrm{rad}}{H_\mathrm{iso}} =
  \frac{1}{\kappa_d\rho H_\mathrm{iso}} \sim
  \frac{1}{\kappa_d\Sigma}. 
\end{align}
For the fiducial disk model, this becomes 
% \begin{align}
%   \khat \gtrsim 3.98\times10^4
%   \hat{\kappa_d}\hat{\Sigma}\hat{T}^2\left(\frac{r}{\mathrm{AU}}\right)^{-33/14}. 
% \end{align}
\begin{align}
  \left(\frac{r}{\mathrm{AU}}\right) \gtrsim 
  89.4\left(\hat{\kappa}_d\hat{\Sigma}\right)^{14/33}\hat{T}^{28/33}\khat^{-14/33}.  
\end{align}
For the MMSN and $\khat=10$, say, this is
\begin{align*}
  r \gtrsim 34 \mathrm{AU}. 
\end{align*}
% These rough estimates suggest that the VSI can operate in the outer
% parts of protoplanetary disks. 

\subsubsection{Radiative diffusion}
For perturbation lengthscales $L>l_\mathrm{rad}$,  
let us assume that temperature fluctuations are smoothed out by
radiative diffusion. To model this in linear theory we can redefine
the cooling time as the operator 
\begin{align}
  t_c^{-1} \to -\eta \nabla^2, 
\end{align}
where $\eta$ is the diffusion coefficient. This, of course,
fundamentally changes the linear problem by introducing higher order
derivatives. We defer a proper analysis of this problem to a follow-up
study. Here, we are interested in order-of-magnitude estimates. We
thus proceed by assuming vertical derivatives can be neglected in the
Laplacian. Then
\begin{align}
  t_c^{-1} \sim \eta k_x^2 = \hat{\eta} \Omega_k \khat^2,
\end{align}
where $\hat{\eta} = \eta/H_\mathrm{iso}^2\Omega_k$ is the
dimensionless diffusion coefficient, so that
\begin{align}
  \beta = \frac{1}{\hat{\eta}\khat^2}. 
\end{align}

Our linearized equations remain valid for thermal relaxation by
radiative diffusion with the above approximation, provided we replace
the dimensionless cooling timescale $\beta$ by
$1/\hat{\eta}\khat^2$. Then the criterion for the fundamental
isothermal VSI, Eq. \ref{iso_vsi_cond} becomes
\begin{align}\label{iso_vsi_cond_diff}
  \frac{(\gamma-1)}{\epsilon |q|\khat^2}< \hat{\eta}. 
\end{align}

Now, the dimensional diffusion cofficient is {\bf(reference needed?)}
\begin{align}
  \eta = \frac{16\sigma_s T^3}{3\kappa_d\rho^2 c_v}, 
\end{align}
so that
\begin{align}
  \hat{\eta} \simeq  \frac{16\sigma_s T^3}{3\kappa_d\Sigma^2
    c_v\Omega_k}. 
\end{align}
The criterion Eq. \ref{iso_vsi_cond_diff} becomes
\begin{align}
\frac{3\kappa_d\Sigma^2\mathcal{R}
  \Omega_k}{16\mu\sigma_s T^3} < \epsilon |q| \khat^2. 
\end{align}
For the fiducial disk model this translates to
\begin{align}
  \left(\frac{r}{\mathrm{AU}}\right) \gtrsim
  76.5\left(\frac{\hat{\kappa}_d\hat{\Sigma}^2}{\hat{T}}\right)^{14/57}\mu^{-14/57}\khat^{-28/57}. 
\end{align} 
Inserting $\mu=2$ and $\khat=10$ we find, for the MMSN, that
\begin{align*}
  r\gtrsim 21 \mathrm{AU}. 
\end{align*}

We have, however, assumed that perturbation lengthscales
$L>l_\mathrm{rad}$, which is the opposite limit to the previous
section. Thus, for the MMSN and $\khat=10$ the above estimate is
restricted to $r\lesssim 34\mathrm{AU}$. This suggests that the VSI
can operate between $\sim 20$---$30\mathrm{AU}$ in the MMSN. 

Finally, we need to verify the radiative diffusion approximation. This
requires an optical depth greater than unity, or
\begin{align}
  l_\mathrm{rad}< H_\mathrm{iso}. 
\end{align}
For the fiducial disk model this is
\begin{align}
  \left(\frac{r}{\mathrm{AU}}\right) \lesssim 41
  \left(\hat{\kappa}_d\hat{T}^2\hat{\Sigma}\right)^{14/33}, 
\end{align}
so the above estimate is valid for the MMSN. 
